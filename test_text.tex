oṃ namaścidātmaparamārthavapuṣe  //

atha

paramārthasāraḥ

śrīmanmahāmāheśvarācāryavaryaśrīmadabhinavaguptācāryaviracitaḥ  /

śrīmadyogarājācāryakṛtavivṛtyupetaḥ  /



cidghano 'pi jaganmūrtyā śyāno yaḥ sa jayatyajaḥ  /
svātmapracchādanakrīḍāvidagdhaḥ parameśvaraḥ  // 1  //

yo 'yaṃ vyadhāyi guruṇā yuktyā paramārthasārasaṃkṣepaḥ  /
vivṛtiṃ karomi laghvīmasminvidvajjanārthito yogaḥ  // 2  //

iha śivādvayaśāsane dehādipramātṛtāprādhānyasvasaṃkalpasamutthaśaṅkātaṅkālasyasaṃśayādirūpavighnaughaprasarapradhvaṃsapūrvikāṃ śāstraniṣpattiṃ manyamānaḥ parimitapramātṛtādhaspadīkāreṇa cidānandaikaghanasvātmadevatāsamāveśaśālinīṃ samastaśāstrārthasaṃkṣepagarbhāṃ prathamatastāvatparameśvarapravaṇatāṃ parāmṛśati  /

paraṃ parasthaṃ gahanādanādimekaṃ niviṣṭaṃ bahudhā guhāsu  /
sarvālayaṃ sarvacarācarasthaṃ tvāmeva śaṃbhuṃ śaraṇaṃ prapadye  // 1  //

tvām sarvapramātṛsphurattāsāraṃ svātmadevatārūpameva śaṃbhum anuttaraśreyaḥsvabhāvaṃ sattātmakam śaraṇam trātāraṃ tvatsamāveśasamāpattaye saṃśraye  / evakāraḥ śaṃbhuṃ svātmadevatākārameva prapadye na punarmāyāntaścāriṇaṃ kañcidbhinnaṃ devamityanyayogaṃ vyavacchinatti  / anyacca kiṃbhūtam param pūrṇaṃ cidānandecchājñānakriyāśaktinirbharamanuttarasvarūpaṃ tathā parasthaṃ gahanāt iti gahanānmāyābhidhānāttattvātparasminpūrṇa eva śivādividyātattvaparyante śuddhādhvani svarūpe tiṣṭhantaṃ na punastattadavasthāvaicitryeṇāpi sphuratastataḥ parasmātpūrṇātsvarūpāttasya pracyāvo bhavati  / yaduktam

jāgradādivibhede 'pi tadabhinne prasarpati  /
nivartate nijānnaiva svarūpādupalabdhṛtaḥ  //

iti spandaśāstre  / anādim purāṇaṃ sarvapratītīnāmanubhavitṛtayā pramātṛtvenādisiddhatvāt ekam ityasāhāyaṃ cidaikyena sphuraṇādbhedasyānupapatteḥ  / tathā niviṣṭam ityādi  / evañjātīyakamapi svasvātantryeṇa bahudhā nānāprakārairbhedaiḥ guhāsu rudrakṣetrajñarūpāsu hṛdguhāsvantarāviṣṭaṃ caitanyarūpo 'pi svayaṃ jaḍājaḍātmatāmābhāsya naṭavannānāpramātṛtayā sthita iti yāvat  / ata eva sarvālayam iti sarvasya rudrakṣetrajñādipramātṛprameyarūpasya jagataḥ ālayam viśrāntisthānaṃ  / sarvamidaṃ kila pūrṇapramātari sthitaṃ sadgrāhyagrāhakayugalakāpekṣayonmagnamiva bhedena prakāśamānaṃ nānārupairvyapadiśyate 'nyathaitasyaprakāśādbhinnasya sattaiva na syātkuta idaṃ viśvamiti sarvanāmapratyavamarśaḥ  / naitāvatā bhagavataḥ samuttīrṇaṃ svarūpamityāha sarvacarācarastham iti  / sarvamidaṃ yajjaḍājaḍasvabhāvaṃ viśvaṃ tadrūpatayā tiṣṭhantaṃ

kartāsi sarvasya yataḥ svayaṃ vai vibho tataḥ sarvamidaṃ tvameva  /

iti nyāyena hi vyatiriktasyānyasyāprakāśamānasya kāryatvānupapatteḥ  /

bhoktaiva bhogyabhāvena sadā sarvatra saṃsthitaḥ  /

iti bhagavāneva tathā tathā cakāstīti  / evaṃvidhaṃ tvāmanuttaraṃ sarvasya svātmadevatāsvarūpaṃ parāhantācamatkārasāramapi gṛhītanānātvamatha cātyantākhaṇḍitasvasvātantryaparamādvayaprakāśasvabhāvaṃ bhagavantaṃ śaṃbhum prapadye śarīrādikṛtrimāhaṅkāraguṇīkāreṇaitādṛśaṃ tvāmeva svātmānaṃ parāhantācamatkārasvarūpaṃ samāviśāmīti yāvat  / anena grahaṇakavākyena paramopādeyāṃ svasvabhāvasamāveśamayīṃ daśāmupadiśatā guruṇā vakṣyamāṇaheyopādeyatayā sakalagranthārthopakṣepaḥ kṛtaḥ  // 1  //

evaṃ prakaraṇatātparyamadvayasvarūpaṃ stutidvāreṇa pratipādyedānīṃ śāstrāvatāramabhidadhatsaṃbandhābhidheyādikaṃ granthakṛdāryādvayenāha

garbhādhivāsapūrvakamaraṇāntakaduḥkhacakravibhrāntaḥ  /
ādhāraṃ bhagavantaṃ śiṣyaḥ papraccha paramārtham  // 2  //

ādhārakārikābhistaṃ gururabhibhāṣate sma tatsāram  /
kathayatyabhinavaguptaḥ śivaśāsanadṛṣṭiyogena  // 3  //

kaścidbhagavatprasādātsamutpannavairāgyaḥ saṃsārādviratamatirguroḥ śāsanīyo 'smīti matvā sadgurum ādhāraṃ bhagavantam śeṣākhyaṃ muniṃ samyagārādhya paramārthopadeśasvarūpaṃ pṛṣṭavān  / tadā tadyogyatāparipākasvarūpapariśīlanakrameṇa tam śiṣyaṃ vigalitāntaḥkaraṇaṃ matvā so 'pyanantanātho niḥśeṣaśāstropadeśajñaḥ paramārthasāranāmnā ādhārakārikābhiḥ ityaparābhidhānagranthena sāṃkhyanayoktopadeśānusāreṇa prakṛtipuruṣavivekajñānātparabrahmāvāptirityevametaṃ śiṣyaṃ proktavān  / sa eva brahmopadeśaḥ paramādvayasvarūpasvasvātantryadṛṣṭyā pratipāditaḥ sanyuktiyukto bhavatīti matvā sarvaṃ janamanugrahītuṃ paramādvayaśaivanayayuktyā gururabhinavena alaukikena ciccamatkārasphāreṇa guptaḥ guhyaḥ sarahasya iti  / so 'yaṃ puṇyanāmākṣarāvaliḥ tatsāram tasya paramārthopadeśasya yatsāraṃ dadhno navanītamivopādeyaṃ parānugrahapravṛttaḥ sanpratipādayatīti  / evaṃ saṃbandhābhidheyābhidhānaprayojanādaya upapāditā neha punaḥ śāstragauravabhayātpratanyante  / kīdṛśaḥ sa śiṣya ityāha garbhādhivāsa iti  / garbhādhivāsaprārambhaṃ jāyate 'sti vardhate vipariṇamate 'pakṣīyate vinaśyatīti tattadavasthāvaicitryeṇa ṣaḍbhāvavikāranemiyuktaṃ yat maraṇāntakaduḥkharūpam āvirbhāvatirobhāvātsaṃsaraṇasvabhāvatayā cakramiva cakram tasmin vibhrāntaḥ viparivṛttaḥ  / anenāsya prāgjātismaraṇasvabhāvo bodhāvirbhāvo dyotito 'nyathā kathaṃ kāṣṭhāprāptilakṣaṇaṃ praśnakutūhalitvaṃ syāt  / evaṃ ca yaḥ samutpannavairāgyaḥ parameśvarānugrahaśaktividdhahṛdayaḥ samuditasamyagjñāna upadeśapātratāmavāpya parameśvarākāraṃ samucitamapi guruṃ samāsādya paramādvayajñānamabhilaṣate sa eva ca gurūpadeśabhājanaṃ syāditi  / etadeva cānyatroktam

śaktipātavaśāddevi nīyate sadguruṃ prati  /

iti  / iha ca purastādvakṣyate  // 3  //

adhunā samutpattikrameṇa pīṭhikābandhaṃ vidhāya viśvavaicitryacitre 'smiñjagati pārameśvaramanuttamaṃ svātantryamekameva saṃyojanaviyojanakartṛtvaheturiti tacchaktivikāsameva viśvamaṇḍacatuṣṭayamukhenāvedayangranthamavatārayati

nijaśaktivaibhavabharādaṇḍacatuṣṭayamidaṃ vibhāgena  /
śaktirmāyā prakṛtiḥ pṛthvī ceti prabhāvitaṃ prabhuṇā  // 4  //

cidānandaikaghanena prabhuṇā svatantreṇa bhagavatā maheśvareṇāṇḍacatuṣṭayaṃ viśvācchādakatvena kośarūpatayā vastupiṇḍabhūtam idam  / yaduktam

... vastupiṇḍo 'ṇḍamucyate  /

iti  / prabhāvitam prakāśitaṃ bhavanakartṛtayā vā prayuktam  / kasmādityāha nijaśaktivaibhavabharāt iti  / nijaḥ svātmīyo yo 'sāvasādhāraṇa icchādyaḥ śaktipracayaḥ tasya yat vaibhavam vicitraḥ prasarastasya yo 'sau bharaḥ samudrekastasmāditi  / bhagavataḥ kila svaśaktivikāsasphāra eva jagannirmāṇam  / yaduktaṃ śrīsarvamaṅgalāśāstre

śaktiśca śaktimāṃścaiva padārthadvayamucyate  /
śaktayo 'sya jagatsarvaṃ śaktimāṃstu maheśvaraḥ  //

iti  / kiṃrūpāṇḍacatuṣṭayasaṃkhyetyāha śaktirmāyā prakṛtiḥ pṛthvī ca iti  / viśvasya pramātṛprameyarūpasya parāhantācamatkārasārasyāpi svasvarūpāpohanātmākhyātimayī niṣedhavyāpārarūpā yā pārameśvarī śaktiḥ saivācchādakatvena bandhakatayā śaktyaṇḍamityucyate  / sadāśiveśvaraśuddhavidyātattvaparyantadalaṃ sadvakṣyamāṇamaṇḍatritayamantaḥ samantādgarbhīkṛtyāvatiṣṭhata iti kośarūpatayaiṣā śaktiranena śabdena saṃjñitā  / etasminnaṇḍe sadāśiveśvarāvevādhipatī  /

anyacca malatrayasvabhāvaṃ mohamayaṃ bhedaikapravaṇatayā sarvapramātṝṇāṃ bandharūpaṃ puṃstattvaparyantadalaṃ māyākhyamaṇḍamityucyate  / tacca vakṣyamāṇamaṇḍadvayamantaḥ samantātsvīkṛtya sthitam  / aṇḍādhipatiścātra gahanābhidhāno rudraḥ  /

tathā sattvarajastamomayī prakṛtiḥ kāryakaraṇātmanā pariṇatā satī paśupramātṝṇāṃ bhogyarūpā sukhaduḥkhamoharūpatayā bandhayitrī prakṛtyabhidhānamaṇḍamityucyate  / tatrāpi mahāvibhūtiḥ śrībhagavānviṣṇurbhedapradhāno 'ṇḍādhipatiḥ  /

tathā pṛthvī caivaṃ manuṣyasthāvarāntānāṃ pramātṝṇāṃ pratiprākārarūpatayā sthūlakañcukamayī bandhayitrīti kṛtvā pṛthvyaṇḍamityucyate  / tatrāpi caturdaśavidhe bhūtasarge pradhānatayāṇḍādhipo bhagavānbrahmeti  / evaṃ parameśvaravijṛmbhitamidamaṇḍacatuṣṭayaṃ bhagavatetthaṃ prakāśitaṃ parisphurati  // 4  //

evametadaṇḍacatuṣṭayaṃ pratipādyātraiva bhogyabhoktṛtvapratipādanaparatayā viśvasvarūpanirūpaṇāya kārikāmāha

tatrāntarviśvamidaṃ vicitratanukaraṇabhuvanasaṃtānam  /
bhoktā ca tatra dehī śiva eva gṛhītapaśubhāvaḥ  // 5  //

tatra teṣu caturṣvaṇḍeṣvāgamaprasiddheṣu viśvamidamantaḥ madhye vartate  / kīdṛśamityāha vicitra iti  / rudrakṣetrajñabhedabhinnā nānāmukhahastapādādiracanārūpāḥ tanavaḥ ākārā viśiṣṭasaṃsthānarūpeṇāścaryabhūtāḥ  / tathānyonyabhedena sātiśayāni karaṇāni cakṣurādīni  / tadyathā rudrapramātṝṇāṃ niratiśayāni sarvajñatvādiguṇagaṇayuktāni taiḥ kila sarvamidamekasminkṣaṇe yugapajjñāyate saṃpādyate ca  / kṣetrajñānāṃ punaretānyeva karaṇāni parameśvaraniyatiśaktiniyantritāni santi ghaṭādipadārthamātrajñānakaraṇasamarthānyeva na taiḥ sarvaṃ jñāyate nāpi kriyate  / tatrāpi yogināmatiśayaḥ karaṇānāṃ yanniyatiśaktisamullaṅghanāttadīyaiḥ karaṇaiḥ dūravyavahitaviprakṛṣṭamapi paricchidyate parapramātṛgataṃ ca sukhaduḥkhādi jñāyate  / evaṃ ca tiraścāmapi niyatiśaktyā saṃkucitānāmapi manuṣyebhyo 'pyatiśayaḥ karaṇānāṃ vidyate  / tadyathā gāvaḥ svagṛhaṃ vyavahitamapi paśyantyaśvā rātrāvapi mārgamīkṣante gṛdhrā yojanaśatagatamapyāmiṣamavalokayanti pakṣiṇo makṣikā maṣakaparyantā ākāśavihāriṇo dṛśyante sarīsṛpā urasā panthānaṃ gacchanti dṛśā ca śṛṇvanti śabdānuṣṭrā dūrādapi gartācchvāsamātreṇa sarpamākarṣayantīti  / evaṃ sarvatra karaṇavaicitryamūhyam  / tathā bhuvanāni āgamaprasiddhāni vartulatryaśracaturaśrārdhacandracchatrākāratayā sātiśayasaṃsthānānīti  / evam vicitraḥ nānātiśayādbhutasvabhāva eṣām tanukaraṇabhuvanānāṃ saṃtānaḥ aviratabandhapravāho yasminviśvasmiṃstadevaṃvidhaṃ viśvam  / evaṃvidhe cātra bhogyasvabhāve viśvasminbhoktrā bhāvyamityāha bhoktā ca tatra dehī iti  / malatrayāghrāto deho bhogāyatanaṃ vidyate yasyāṇoḥ sa dehī sukhaduḥkhādisvabhāvaḥ śarīrī sukhaduḥkhādimaye 'smin bhoktā sukhaduḥkhādyanubhavitā paśupramāteti kathyate  / nanu parapramātrapekṣayāṇumātrasyāpi na bhedo vidyate kutastadvyatirikto dehī nāma varākaḥ  / yaduktam

pradeśo 'pi brahmaṇaḥ sārvarūpyamanatikrāntaśca avikalpyaśca  /

iti  /

ekaikatra ca tattve 'pi ṣaṭtriṃśattattvarūpatā  /

iti nyāyāccaika eva svaśaktiyukto mahāprakāśavapureva parameśvaraḥ pramātā sarvato 'bhinna evāvabhāsate  / tato bhinnasyāprakāśamānasya dehino 'stitvābhyupagame 'pi prakāśamānatvānupapatterna sattā niścīyate  / prakāśate cettasminparabrahmātmani tarhi prakāśābhinna evaikaḥ pramāteti punarapi kiṃparatvenāyaṃ bhogyabhoktṛlakṣaṇaḥ sanbheda iti sarvaṃ samarthayamāna āha śiva eva gṛhītapaśubhāvaḥ iti  / yo 'yaṃ bhagavānsamanantaraṃ pratipāditaścidānandaikaghanaḥ svātantryasvabhāvaḥ śivaḥ sa eva svarūpagopanāsatattvaḥ sansvecchayā naṭa iva dehapramātṛbhūmikāṃ samāpannaḥ pālyatvātpaśutvātpaśusattālakṣaṇaśca sukhaduḥkhādimaye svayaṃnirmite 'sminbhogye bhoktā dehīti kathyate na punaḥ śivavyatiriktaṃ kiñcitpadārthajātamasti  / eṣa eva ca bhagavāñśivaḥ svātantryādbhoktṛbhogyalakṣaṇaṃ pramātṛprameyayugalakaṃ krīḍanakamiva samutthāpayati yadapekṣayāyaṃ bhedapradhāno vyavahāraḥ  / tasmādetadeva parameśvarasya svātantryaṃ niratiśayaṃ yatpūrṇasvarūpatāparityāgena bhoktṛbhogyasvabhāvaṃ paśubhāvamāpanno 'pi sarvapramātṝṇāmanubhavitṛtayā svātmani prasphurañcidānandaikaghanaḥ śiva eva  // 5  //

evamapyekaścitsvabhāvaḥ pramātā sa yadi māyādipramātṛprameyavaicitryeṇa nānātvādaneka iti kathaṃ viruddhayaikatayā vyavahriyata ekaścetkimiti nānārūpa iti cchāyātapavadvirodhādviruddhadharmādhyāsaḥ samāpatati na punaranekarūpa ekaśca padārthaḥ syāt  / yaduktam

ayameva bhedo bhedaheturbhāvānāṃ yadviruddhadharmādhyāsaḥ kāraṇabhedo vā  /

iti  / laukikamatra dṛṣṭāntaṃ pradarśayandārṣṭāntike codyaṃ samarthayate

nānāvidhavarṇānāṃ rūpaṃ dhatte yathāmalaḥ sphaṭikaḥ  /
suramānuṣapaśupādaparūpatvaṃ tadvadīśo 'pi  // 6  //

yathā eko 'pi sphaṭikamaṇiḥ tattallākṣānīlādyupādhivaicitryasahasreṇa tattadvaicitryaṃ svātmani dhārayaṃstathā vicitrito bhavati na punastasya sphaṭikatāhāniretāvatā samutpadyate  / etadeva sphaṭikamaṇermaṇitvaṃ yattattadviśeṣeṇācchurite 'pi tasminsphaṭikamaṇirayamityabādhitā sarvasya sarvadaiva pratītiḥ  / kevalamatrāmī lākṣādayaḥ sphurantīti vyavahriyate na punarlākṣādyupādhiḥ paṭamiva taṃ viśinaṣṭi yena svarūpavipralopo 'sya syāt  / tasmādetadevāmalatvaṃ maṇeryadupādhirūpānākārānbibharti svasvarūpatayā ca prathate  / tathaivāyamīśvaraḥ svatantraścidekaghana ekako 'pi svacche svātmadarpaṇe devamanuṣyapaśupakṣisthāvarāntānāṃ rudrakṣetrajñādipadārtharūpāṇāṃ viśeṣāṇāṃ svayaṃ nirmitānāṃ ca rūpatvam varṇavaicitryaṃ sphaṭikamaṇivatsvātmābhedena dhārayaṃstato 'pi samuttīrṇatvādahamityevamakhaṇḍacamatkāropabṛṃhitaṃ nānārūpamapyekaṃ svātmānaṃ pratyavamṛśati  / itthamapyasyaikatākhaṇḍanāmayo bhinnarūpo deśaḥ kālo vā na kaścidvidyate yadapekṣayaitasya svātmamaheśvarasya viruddhadharmādhyāsādidūṣaṇamucyetāpi  / sākṣātkāralakṣaṇaṃ citrajñānaṃ nānābhedasaṃbhinnamapi parairapyekameva tadabhyupagatam  / yathā

nīlādiścitravijñāne jñānopādhirananyabhāk  /
aśakyadarśanastaṃ hi patatyarthe vivecayan  //

iti pramāṇavārtike  / kiṃ punaḥ sarvataḥ pūrṇasya jñātuścidekavapuṣaḥ svatantrasya yāvimau deśakālau bhedakatayābhimatau mūrtivaicitryakriyāvaicitryābhyāṃ yasya samullāsakatayā sthitau kathaṃ tasyaiva bhagavato vyavacchedakau syātām  / yadi nāma deśakālayoḥ kadācitsaṃvido bhedena sthitirabhaviṣyattadā tatkṛto 'pi viruddhadharmādhyāsa udapatsyateti tatra saṃbhāvanā syāt  / yāvatā tayoḥ saṃvitprakāśenaiva svātmasattāsiddhiriti siddha evānekasvabhāvo 'pyeka eva maheśvaraścinmūrtirbhedadharme punarviruddhadharmādhyāso duruddhara eveti  // 6  //

nanveka eva saṃvitsatattvaḥ pramātābhyupagatastanukaraṇabhuvanatāṃ samāpannaḥ sansa evānekatāṃ yāta iti cettarhi tanvādivināśe sa eva vinaṣṭaḥ syāttadutpattau vaiṣa eva tadotpadyeta  / evaṃ pratipramātṛ sa eva jāyate 'stītyādiṣaḍbhāvavikāratayā vyavacchidyate puṇyapāpasvabhāvakarmavaicitryeṇaitasyaiva bhagavataḥ svarganarakādibhogaḥ prāpta iti kathamucyate svasvarūpa eva śiva iti  / dṛṣṭāntadvāreṇaitadapi samarthayate

gacchati gacchati jala iva himakarabimbaṃ sthite sthitiṃ yāti  /
tanukaraṇabhuvanavarge tathāyamātmā maheśānaḥ  // 7  //

yathā jalapravāhe yāti sati himakarabimbam candravapurvastuvṛttenākāśasthaṃ svayamacalattātmakaṃ jalapravāhāntaḥpatitamapi tat gacchati prayāti iva tathā tasminneva kṣaṇe 'nyatra jalāśaye niḥstimite sati tadeva himakarabimbaṃ sthitim gacchatīvetyubhayathā sarvapramātṛbhiretatsaṃbhāvyate na punaḥ paramārthena tattathaiva syāt  / nāpi jalagatau deśakālau bhedakatayā candramasaḥ svarūpaṃ gaganasthaṃ parāmṛśataḥ kevalaṃ jalameva tādṛśamatha ca tatpratibimbitasya candrabimbasya jalagatacalattācalattādiko bhedo vyavahriyata ityetāvatā gaṅgājalagatasya kardamapatitasya vā śaśinaḥ svasvarūpatāyāṃ na kācitkṣatiḥ  / tathaivāyamātmā caitanyasvabhāvaḥ svayaṃ nirmite tanukaraṇabhuvanasamūhe parikṣīṇe sati samutpanne vā prakṣīṇaḥ samutpannaśceti māyāvyāmohitānāṃ vyavahāramātrametajjalagatacandravanna punaḥ svātmā jāyate mriyate veti  / gītāsvevamevoktam

na jāyate mriyate vā kadacinnāyaṃ bhūtvā bhavitā vā na bhūyaḥ  /
ajo nityaḥ śāśvato 'yaṃ purāṇo na hanyate hanyamāne śarīre  //

iti  / tasmādayamātmā maheśānaḥ svatantraḥ sarvasvātmapratyavamarśasvabhāvaḥ sarvapramātṝṇāmanubhavitṛtayā prathamānastatastattadavasthāvipralope samutpattau vā svasvarūpa eva  / etadeva ca durghaṭakāri maheśānatvaṃ saṃvittattvasya yattathā tathā paśupramātṛtayā svarganarakādibhogabhoktāpi sarvānubhavitṛtayā saṃvitsvarūpa eva  / pratyuta puṇyapāpasvarganarakakṣutpipāsādiko yo 'yaṃ paśubhāvo bandhakatayā niyataḥ sa yadi bhagavatā svātmaprakāśena prakāśitaḥ parāmṛṣṭaśca syāttadā yathoktāṃ svātmani sattāṃ labhate 'nyathā niḥsvabhāva evaiṣa iti kathaṃ svātmanaḥ tasyaiva maheśasya svarūpavipralopāyocyate  / sarvathā nirmitameva vastu saṃhāryaṃ samutpādyaṃ vā dehādirūpaṃ syānna punaḥ nitye bhagavati caitanye samutpattivināśau kadācidbhavetām  / tasmādeka evātmā grāhyagrāhakatayā nānārūpasvabhāvaḥ sanpunarapi sarvānubhavitṛtayā sarvasyaikatayā prathata iti na kācidadvayavādakṣatiḥ  // 7  //

itthamapi sarveṣāmayamātmā viśvaprapañcasvabhāva eva saṃvinmātraparamārthaḥ sarvāvabhāsaḥ sarvatra saṃvidanugamāditi yuktyāgamābhyāṃ pratipāditaścetkimiti loṣṭādāvapyaviśeṣātsvātmatayā sa na pratīyate 'bhyupagame vā jaḍājaḍavyavastheyaṃ bhāsamānā na saṃgacchate lokavyavahāraśca jaḍājaḍarūpa iti kathametatsyādityāha

rāhuradṛśyo 'pi yathā śaśibimbasthaḥ prakāśate tadvat  /
sarvagato 'pyayamātmā viṣayāśrayaṇena dhīmukure  // 8  //

ākāśadeśe rāhuḥ sarvatra paribhramannapi nopalabhyate sa eva punargrahoparāgakāle candramūrtisthaḥ prathamāno 'yaṃ rāhuriti parīkṣyate 'nyathā sthito 'pi bhacakre 'sthita iva  / tathaivehāpi sarvāntaratamatvena sthito 'pi ayamātmā svānubhavaikasvarūpatayā pratyakṣaparidṛśyamānaḥ sarvasya tathā nopalakṣyate  / yadā punaḥ puryaṣṭakapramātṝṇāṃ buddhidarpaṇe pratibhāmukure grāhyavyavasthākāle śabdādiviṣayasvīkāreṇa śṛṇomītyevamahaṃpratītiviṣayo bhavati tadā grāhakasvabhāvatayā loṣṭādāvapi sthitaḥ sansphuṭarūpastatraiva svātmā prakāśate sarvaiśca svānubhavaikarūpaḥ pratīyate loṣṭādāvatyantatamomayatvātsthito 'pyasthitakalpo 'sau prathate rāhurākāśe yathā  / evamayaṃ bhagavānmāyāśaktyā svātmakalpe 'pi bhāvavarge kāṃścitpuryaṣṭakasvarūpānvedyakhaṇḍānapyahantāvyavasthārasābhiṣiktānvedakīkaroti kāṃścidvedyīkaroti yadapekṣayāyaṃ jaḍājaḍavyavasthāsvarūpo bhedavyavahāraḥ susthita evopapadyate  / tena loṣṭādirvedyatvājjaḍo vedakatvātpuryaṣṭakapramātāpyajaḍaḥ  / na punaḥ paramārthena parameśvarāpekṣayā jaḍājaḍavyavahāra iti  // 8  //

nanu sarvapramātṝṇāṃ buddhau cetsvātmano 'pyaviśeṣeṇa prasphuraṇaṃ tarhi te sarve kimiti svātmavido na syustajjñānavanto mā vā bhūvanviśeṣābhāvāt  / yatpunaḥ saṃsārāvasthāyāmapi kecana svātmajñānājjīvanmuktāḥ sarvajñatvasarvakartṛtvaśālinaḥ kecana svātmajñānayogyā ārurukṣavaśca dṛśyante 'pare svātmajñānarahitāḥ santo dharmādharmanimittaśubhāśubhakarmanigaḍaprabandhabaddhāḥ saṃsāriṇa eveti kathametatsaṃgacchata itthamantaḥ sarvaṃ kṛtvā pārameśvaraḥ śaktipāto viśṛṅkhala iti pratipādayati

ādarśe malarahite yadvadvadanaṃ vibhāti tadvadayam  /
śivaśaktipātavimale dhītattve bhāti bhārūpaḥ  // 9  //

darpaṇe mālinyarahite yathā nirviśeṣarūpādiguṇagaṇayuktaṃ mukhaṃ cakāsti  / na sa deśo 'sti yaṃ vinivṛttamala ādarśo na svīkurute  / samale darpaṇe tu mukhamananyātiśayamapi dhyāmalatvādvaiparītyena prakāśate  / nāpi malinastadīyānguṇānsvīkartumalam  / pratyuta tannyastamukhaḥ pumānmukhamanyathaiva dhyāmalatvādyupetamavalokya svātmano lajjāmāvahati vikṛtaṃ madvadanamiti  / tathaiva śivasya svātmano yāsāvanugrahākhyā śaktiḥ tasyāḥ pātaḥ svakiraṇavisphārastena saṃmārjite pratibhāmukura āṇavamāyīyakārmamalavāsanāprakṣayādviśadīkṛte keṣāñcidapaścimajanmanāṃ pramātṝṇām bhāḥ prakāśaḥ rūpam yasyeti saḥ bhārūpaḥ svātmā yāvatsarvajñatvādiguṇagaṇayuktastāvānapyavabhāsate yena kecanaiva te svātmasvarūpaprathanātsaṃsāramadhyapatitā api muktakalpāḥ sātiśayāśca  / keṣāñcideva parameśvaratirodhānaśaktyāṇavamāyīyakārmamalasamācchādite buddhitattve bhārūpo 'pyātmā mālinyādbhāto 'pyabhātakalpo yena te sāṃsārikāḥ paśava ityabhidhīyante 'nye 'pyubhayaśaktiyogātpramātāra ārurukṣava iti  / itthaṃ tīvramandamandatarādibhedena śaktipātavaicitryaṃ sarvatrāpyūhyam  / atra na māyāntaḥpātiniyatiśaktisamutthamaśvamedhādikaṃ japadhyānādi vānyadyatkiñcitkarma mocanaheturātmanastasya hi māyātaḥ samuttīrṇatvādbhedapradhānaṃ vastu tatsādhanāya na prakalpate  / yadgītam

nāhaṃ vedairna tapasā na dānena na cejyayā  /

iti  / tasmādekamevātra parameśvarānugrahaḥ kāraṇamakṛtrimaṃ bhavyabuddhīnām  / yaduktam

īśituḥ śaktipātāṃśe khyāpayitrī svatantratām  /
dhīḥ kāraṇakalāghrātā naiva kiñcidapekṣate  //

iti  / paśupramātṝṇāṃ tu parameśvaratirodhānaśaktiḥ saṃsaraṇahetureva yenaite svasvarūpāprathanācchubhāśubhakarmaniratāḥ sukhaduḥkhādibhogabhājaḥ punaḥ punarasminsaṃsaranti  / tasmātpramātṝṇāṃ sādhāraṇe 'pi svātmani prakāśāprakāśarūpe anugrahatirodhānaśaktī dve eva te mokṣabandhapravibhāgahetū  / yaduktam

badhnāti kācidapi śaktiranantaśakteḥ kṣetrajñamapratihatā bhavapāśajālaiḥ  /
jñānāsinā ca vinikṛtya guṇānaśeṣānanyā karotyabhimukhaṃ puruṣaṃ vimuktau  //

iti  // 9  //

evamidaṃ sarvamāgamānubhavayuktiyuktaṃ pratipādya yatprākśaktyādyaṇḍacatuṣṭayaṃ pratipāditaṃ tadantarālavarti samutpattikrameṇa ṣaṭtriṃśattattvātmakaṃ jagadyallagnaṃ bhāti tatprathamataḥ kāraṇakāraṇaṃ paramaśivasvarūpaṃ kārikādvayenāha

bhārūpaṃ paripūrṇaṃ svātmani viśrāntito mahānandam  /
icchāsaṃvitkaraṇairnirbharitamanantaśaktiparipūrṇam  // 10  //
sarvavikalpavihīnaṃ śuddhaṃ śāntaṃ layodayavihīnam  /
yatparatattvaṃ tasminvibhāti ṣaṭtriṃśadātma jagat  // 11  //

yat evaṃvidham param pūrṇaṃ śivatattvaṃ tatra śivādidharāparyantaṃ vakṣyamāṇaṃ viśvaṃ viśrāntaṃ satprakāśate tadabhinnameva cakāsadyuktyopapadyata iti yāvat  / nanu tanyate sarvaṃ tanvādi yatra tattattvaṃ tananādvā tadāpralayaṃ tasya bhāva iti vā tattvamityevamapi tattvavyapadeśo 'yaṃ jāḍyāpādakaḥ kathaṃ cidrūpe bhagavati paramaśive syāducyate  / upadeśyajanāpekṣayā yāvatā śabdena pratipādyate tāvatā tatra tattvavyapadeśo na vastutaḥ  / kīdṛśaṃ tatparaṃ tattvam  / bhāḥ prakāśaḥ rūpam svabhāvo yasya mahāprakāśavapurityarthaḥ  / tathā paripūrṇam nirākāṅkṣam  / nirākāṅkṣamapi sphaṭikamaṇidarpaṇādi jaḍaṃ vastu bhavatītyāha svātmani viśrāntito mahānandam iti  / svasminsvabhāve 'khaṇḍāhantācamatkārarase viśramānmahānānandaḥ parā nirvṛtiryasyeti  / tadevaṃ paramāhlādakasphurattāsāratvātprakāśyasphaṭikāderjaḍādvailakṣaṇyamuktaṃ bhavati  / ata evāha icchāsaṃvitkaraṇairnirbharitam iti  / icchājñānakriyāśaktisvabhāvameva na punaḥ śāntabrahmavādināmiva śaktivirahitaṃ jaḍakalpam  / anyacca anantaśaktiparipūrṇam iti  / anantāḥ niḥsaṃkhyā ghaṭapaṭādyā nāmarūpātmikāḥ śaktayaḥ icchājñānakriyāśaktīnāṃ pallavabhūtā brāhmyādyāḥ śaktayaḥ śabdarāśisamutthāstābhiḥ paritaḥ samantāt pūrṇam vyāptaṃ tata evollasantyastatraiva śāmyantīti  / evaṃ parāvāgrūpaṃ bhagavati svātantryamuktaṃ syāt  / nanu vāgrūpaṃ cetparaṃ tattvaṃ tarhi kālpanikaṃ śabdasaṃbhinnatvātkathaṃ śuddhaprakāśavapuṣi kalpanāyoga ityāśayenāha sarvavikalpavihīnam iti  / parapramātari yo 'yaṃ parāhantācamatkāraḥ sa vāgrūpo 'pi nirvikalpaḥ  / vikalpo hyanyāpohalakṣaṇo dvayaṃ ghaṭāghaṭarūpamākṣipannaghaṭādvyavacchinnaṃ ghaṭaṃ niścinoti  / prakāśasya punaḥ parāhantācamatkārasārasyāpi nāprakāśarūpaḥ prakāśādanyaḥ pratipakṣatayā vidyate yadvyavacchedāttasya vikalparūpatā syāt  / vyavacchedyo hyartho 'prakāśātmā prakāśavapuṣi prakāśate cettarhi

tatsaṃvedanarūpeṇa tādātmyapratipattitaḥ  /

ityādinyāyena yo 'pyarthaḥ prakāśasvabhāvatāṃ yātaḥ sansa kathaṃ svātmanastasyaiva vyavacchedakaḥ syādyena vikalparūpatāṃ tatra samāvahet  / atha pratipakṣatayā na prakāśata iti kathamihāprakāśamānaḥ padārthaḥ pratipakṣarūpo 'stīti paricchettumapi śakyeteti yatkiñcidetatsyāt  / yataḥ sarvaiḥ vyavacchedātmakaiḥ vikalpairvihīnam aparicchinnasvabhāvaṃ paraṃ tattvam  / ata evāha śuddham iti vimalaṃ vikalpamayyā aśuddhimaṣyā abhāvāt  / tathā śāntam iti  / grāhyagrāhakasamutthakṣobhābhāvācchaktisāmarasyena svasvarūpasthaṃ na punaraśmaśakalakalpam  / anyacca layodayavihīnam iti  /

sakṛdvibhāto 'yamātmā  /

iti kṛtvā sanātana eva  / ato bhūtabhaviṣyadvartamānavapuḥ kālastatra na kramate yataḥ kālasya tata eva samullāsa iti samutpattivināśabahiṣkṛte paratattve 'bhyupagate viśvasya viśvatvamupapannamiti pratipāditaṃ syāt  // 11  //

nanvevaṃvidhe paratattve jagadbhātīti yatpratipāditaṃ tatkathametatsyādyāvatā paratattvāpekṣayā na kiñcidbhedena bhātuṃ pragalbheta  / tato bhinnaṃ cejjagadbhāsate tadādvayavādakhaṇḍanābhinnaṃ cejjagatprakāśata iti kathaṃ vācoyuktiriti dṛṣṭāntadvāreṇa tadbhedābhedarūpaṃ tattvamupadarśayannetatsamarthanāyāha

darpaṇabimbe yadvannagaragrāmādi citramavibhāgi  /
bhāti vibhāgenaiva ca parasparaṃ darpaṇādapi ca  // 12  //
vimalatamaparamabhairavabodhāttadvadvibhāgaśūnyamapi  /
anyonyaṃ ca tato 'pi ca vibhaktamābhāti jagadetat  // 13  //

yathā nirmale mukurāntarāle nagaragrāmapuraprākārāṭṭasthalanadanadījvalanavṛkṣaparvatapaśupakṣistrīpuruṣādikaṃ sarvaṃ pratibimbatayā citram svālakṣaṇyena nānārūpaṃ bhāsate avibhāgi darpaṇādavibhaktaṃ sat bhāti tadabhedenaivāntarākāraṃ samarpayati tatrābhedena bhāsamānamapi bhāti vibhāgenaiva ca parasparam ityanyonyasvālakṣaṇyena ghaṭātpaṭo bhinnaḥ paṭātghaṭa iti vibhaktatayā sphurati  / tadantargatā hi bhāvā eva pṛthaktvena parāmṛśyante na punastaṃ darpaṇaṃ tyaktvā pṛthakkiñcidupalabhyate kintu darpaṇasāmarasyena sthitamapi sarvato bhinnaṃ jagatpratīyate  / evamapi ghaṭādipratibimbena darpaṇastarhyantarhitaḥ syādityetannetyāha darpaṇādapi ca iti  / na kevalaṃ svayaṃ bhāvā darpaṇāntargatā api bhinnāḥ prakāśante yāvaddarpaṇādapi vyatiricyante yato darpaṇastattatpratibimbamayo 'pi tebhyaḥ pratibimbebhyaḥ samuttīrṇasvarūpatayā cakāsti na punastanmayaḥ saṃpadyate yena ca na darpaṇa iti pratītaḥ syāt  / sarvasya punastattatpratibimbagrahaṇe 'pi darpaṇo 'yamityabādhitā pratipattiḥ  / nāpi ghaṭādirdarpaṇaṃ viśinaṣṭi yenāyaṃ ghaṭadarpaṇo 'yaṃ paṭadarpaṇa iti svasvarūpatāhāniratra jāyate  / deśakṛtaḥ kālakṛto vā bhedo na tatra svabhāvavipralopāya bhavati  / tasmāttattatpratibimbasahiṣṇuḥ sansvātmani darpaṇo darpaṇa eveti na kācitpratibimbavādakṣatiḥ  / athocyate bhrāntireṣā yaduta darpaṇe hastīti parāmṛśyate na tu punardarpaṇe sa kaścidvidyate tathātvenārthakriyāvirahādbhrāntyaivaiṣa niścaya iti  / etāvatā pratibimbavādena dṛṣṭāntastāvatsiddhaḥ  / bhrāntestu svarūpaṃ samanantaraṃ nirūpyate  / tadvat tathaiva darpaṇanagarādipratibimbadṛṣṭāntena vimalatamaparamabhairavabodhāt atiśayena vigalitakālikātpūrṇānandodriktātprakāśāt jagat viśvam vibhāgaśūnyamapi darpaṇapratibimbavattataḥ prakāśādavibhaktamapi parasparaṃ ca vibhaktatvena grāhyagrāhakāpekṣayā nānārūpaṃ prathate tato 'pi ca iti bodhādapyunmagnamiva ābhāti yato bodhastadrūpatayāpi prathamānastataḥ samuttīrṇaḥ prathate yathā pratibimbebhyo darpaṇaḥ
/ evamapi viśvabhāvapratibimbasahiṣṇuḥ prakāśo viśvabhāvebhyaḥ samuttīrṇaḥ sarvasyānubhavitṛtayā svasvarūpeṇa prathate  / bhāvagato 'pi deśakālākārabhedaḥ kevalamatra prakāśate darpaṇavanna punaḥ svaṃ rūpaṃ saṃbhinatti  / ata evaikānekasvarūpo 'pi bodha eka eva bodhābhyupagatacitrajñānavat  / kintu darpaṇaprakāśātsacamatkārasya citprakāśasyeyānviśeṣo yaddarpaṇe svacchatāmātrasanāthe bhinnaṃ bāhyameva nagarādi pratibimbatvenābhimataṃ bhāti na tu svanirmitamato darpaṇe 'yaṃ hastīti yo niścayaḥ sa bhrāntaḥ syātprakāśaḥ punaḥ svacamatkārasāraḥ svecchayā svātmabhittāvabhedena parāmṛśansvasaṃvidupādānameva viśvamābhāsayati  / viśvasyābhāsanameva nirmātṛtvaṃ bhagavata iti parāmarśa eva prakāśasya jaḍāddarpaṇaprakāśādervailakṣaṇyāpādakaṃ mukhyaṃ rūpamiti  / etadeva granthakṛtā vivṛtivimarśinyāmuktam

antarvibhāti sakalaṃ jagadātmanīha yadvadvicitraracanā makurāntarāle  /
bodhaḥ punarnijavimarśanasārayuktyā viśvaṃ parāmṛśati no makurastathā tu  //

iti  / itthaṃ parameśvarāpekṣayā svāṅganirmite bhāvarāśau na kācidbhedabhrāntirmāyāpramātrapekṣayā tu yo 'yaṃ bhedāvabhāsa eṣā pūrṇatvākhyātirūpā bhrāntiḥ pūrṇasyādvayātmano rūpasyākhyānamaprathā pūrṇaṃ na bhāsate kintvapūrṇaṃ dvayarūpaṃ bhāsate bheda eva pratīyata iti yāvat  / tasmānniravadyo 'yaṃ pratibimbavādaḥ  // 13  //

itthaṃ paratattvasvarūpanirūpaṇapūrvaṃ prakāśābhedena jagataḥ ṣaṭtriṃśattattvātmakasya sthitiṃ vidhāya punarapyetasya samutpattikrameṇa pratitattvaṃ svarūpaṃ kārikābhiḥ pratipādayati

śivaśaktisadāśivatāmīśvaravidyāmayīṃ ca tattvadaśām  /
śaktīnāṃ pañcānāṃ vibhaktabhāvena bhāsayati  // 14  //

yo 'yaṃ paramaśivaḥ paratattvanirūpaṇayā samanantarapratipāditasvarūpaḥ svasvarūparūpā yāḥ śaktayaścinnirvṛtīcchājñānakriyākhyāḥ pañcānantaśaktivrātahetubhūtāstāsāmeva pañcānāṃ śaktīnām bhinnatvenātadvyāvṛttyemām tattvadaśām pañcasaṃkhyāvacchinnāmeva bhāsayati svālakṣaṇyena prakaṭayatītyarthaḥ  / kīdṛśīṃ tāmityāha śiva ityādi  / śivaśca śaktiśca sadāśivaśca teṣāṃ bhāvo yasyāḥ sā tathoktā tāṃ tathā īśvaravidye prakṛtiryasyāṃ sā tatheti  / atra pratitattvaṃ svarūpaṃ pradarśyate  / tathā hi sarvapramātṝṇāmantaḥ pūrṇāhantācamatkāramayaṃ sarvatattvottīrṇaṃ mahāprakāśavapuryaccaitanyametadeva śivatattvam  / atra tattvanirūpaṇamupadeśyajanāpekṣayeti  / tasyaiva bhagavataścidrūpasyānandarūpā viśvaṃ bhavāmīti parāmṛśato viśvabhāvasvabhāvamayī saṃvideva kiñciducchūnatārūpā sarvabhāvānāṃ bījabhūmiriyaṃ śaktyavasthā  / eṣaiva viśvagatasṛṣṭisaṃhāropacārātkṛśapūrṇobhayarūpāpyekaiva sarvarahasyanayeṣu gīyate  / punarapyatraiva viśvasamutpattibījabhūmau mahāśūnyātiśūnyākhyāyāṃ maheśasyāhamidamityabhedena pūrṇāhantāmayo yaścamatkāro jñānaprādhānyātkriyābhāgasyāhantāviśrānteḥ seyaṃ sadāśivadaśā  / atra mantramaheśvarāḥ pramātārastiṣṭhanti  / tathātraivāhamidamityabhedenāhantedantayoḥ samadhṛtatulāpuṭanyāyena yaḥ svātmacamatkāraḥ saiṣā tasyeśvarāvasthā  / atrāpi mantreśvarāḥ pramātāraḥ  / atrāpīdantāprādhānyenāhantāguṇīkāreṇa yo 'hamahamidamidamityevaṃrūpaścamatkāraḥ sadyojātabālasyeva śiro 'ṅgulīnirdeśya etadeva bodhasāratvādbhagavataḥ śuddhavidyātattvam  / atra vidyeśvaraiḥ saha saptakoṭyastu mantrāṇāṃ vācakatayānugrahasvabhāvātpaśūnuddhartuṃ vācyānmantramaheśvaramantreśvarānpratyavatiṣṭhante  / atra vidyātattve vidyeśvarapramātṝṇāṃ bodharūpatvāviśeṣe 'pi yā bhedaprathā sā māyāśaktikṛtaivetyāgameṣu gīyate

māyopari mahāmāyā ...  /
iti  / yena tatrasthā mantrā mahāmāyānupraveśādaṇava ityucyante  / māyātattvopari śuddhavidyādhaśca vijñānākalāḥ pramātāra āṇavamalabhājanam  / evamekamevedaṃ śivasvarūpaṃ turyātītamapi turyarūpatayā tattvapañcakatayā gīyate  / tasmādeka evaiṣa svatantraḥ kartā prakāśate yasyāhamidamiti sadāśiveśvarabhūmau yaḥ prakāśa etadeva śuddhavedanarūpaṃ karaṇaṃ vakṣyamāṇo māyātattvādidharāntastattvasargaśca kāryamityeva kartṛkaraṇakriyārūpa eka eva svātmamaheśvarākhyaḥ paramapramātā vijṛmbhate  // 14  //

māyātattvasvarūpamāha

paramaṃ yatsvātantryaṃ durghaṭasaṃpādanaṃ maheśasya  /
devī māyāśaktiḥ svātmāvaraṇaṃ śivasyaitat  // 15  //

paramam ananyāpekṣam yat parameśituḥ svātantryam viśvanirmātṛtvaṃ saiveyam māyākhyā śaktiḥ tasya śaktimataḥ  / mīyate paricchidyate dharāntaḥ pramātṛprameyaprapañco yayā sā māyā viśvamohakatayā vā māyā  / eṣā devasya krīḍāśīlasya saṃbandhinīti kṛtvā devī na punarbrahmavādināmiva vyatiriktā kācinmāyopapadyata iti  / kīdṛśaṃ tatsvātantryam durghaṭasaṃpādanam iti  / duḥkhena ghaṭayituṃ śakyamiti durghaṭasya kāryasya pramātṛprameyarūpasya saṃpādanam prāptiprāpakam  / eṣaiva māyā svecchayā paśubhāvamāpannasya śivasya svātmāvaraṇam svarūpagopanākhyamāṇavādimalatritayam  // 15  //

vakṣyamāṇe ca prādhānike sukhādirūpe bhinne bhogye yadbhoktṛrūpaṃ puṃstattvaṃ tatsvarūpamāha

māyāparigrahavaśādbodho malinaḥ pumānpaśurbhavati  /
kālakalāniyativaśādrāgāvidyāvaśena saṃbaddhaḥ  // 16  //

māyā svīkārapāratantryātsarvajñatvasarvakartṛtvamayo 'pi bodhaḥ sarvajñatvādiguṇāpahastanenākhyātirūpamāṇavaṃ malamāpanno yena ghaṭākāśavatpūrṇasvarūpāccidākāśādavacchedya parimitīkṛtaḥ saṃstadeva puṃstattvamucyate  / ato māyāyāḥ pālyaḥ pāśyaśceti paśuḥ āṇavamāyīyakārmamalasvabhāvānāṃ pāśānāṃ bhājanam  / anyacca kālakalā iti vakṣyamāṇasvarūpaiḥ kālādibhiḥ otaprotatayā samyak baddhaḥ dṛbdha ityevaṃ tattvaṣaṭkaveṣṭitaṃ puṃstattvam  // 16  //

kālādīnāṃ tattvānāṃ caitadveṣṭanakrameṇa svarūpamāha

adhunaiva kiñcidevedameva sarvātmanaiva jānāmi  /
māyāsahitaṃ kañcukaṣaṭkamaṇorantaraṅgamidamuktam  // 17  //

itthaṃ svatantro 'pi bodhaḥ svamāyayā yathāṇutvaṃ prāptastathaiva tadīye jñānakriyāśaktī saṃkucite asya paśurūpasya vidyākale ityucyete  / yathā rājñāpahṛtasarvasvasyānukampayā jīvanārthaṃ kiñciddhanaṃ stokaṃ dīyate tathaivāṇutvamāpannasya bodhasyāpahṛtasarvajñatvādeḥ kiñcitkartṛtāparamārthaṃ jñatvamupodbalyata iti jñatvasyaiva prādhānyātkālādīnāṃ jānātinānvayo darśitaḥ  / idaṃ kañcukaṣaṭkam uktarūpayā māyayā yuktam aṇoḥ āṇavamalāpahastitasarvajñatvādeḥ puṃsaḥ svarūpācchādakaṃ svarṇasya kālikeva antaraṅgam nijaṃ kathitam  / kiṃrūpamityāha adhunā ityādi  / adhunaiva jānāmi iti so 'ṇurvartamānatayedaṃ prāṅmayā jñātaṃ jānāmi jñāsyāmītyevamapi kṛtaṃ karomi kariṣye veti jñānakriyāsvarūpeṇa bhāvānapi tathā kalayannavacchinatti cetyeṣo 'sya kālaḥ  / tathā kiñcideva ityavacchinnameva karoti sarvaṃ kartuṃ nālaṃ ghaṭamātrakaraṇāya prabhavati na paṭādāvityetadasyāṇoḥ kalātattvam  / idameva iti niyatātkāranānniyataṃ kāryaṃ yadarthayate yathā vahnereva dhūmo 'śvamedhādikarmaṇa eva svargādiphalaṃ na sarvasmādityevaṃ niyamena svasaṃkalpakṛtakarmaprabandhasamutthapuṇyāpuṇyairātmā niyamyate yena tadasya niyatitattvam  / tathā sarvātmanā iti yeyamapūrṇammanyatā sarvaṃ mamedamupayujyate bhūyāsaṃ mā kadācinna bhūvamityetatpaśo rāgatattvam  / buddhidharmo yo rāgaḥ sa ekatra kutrāpi kāntālakṣaṇe 'rthe 'nyadapohyātra me rāga ityabhiṣvaṅgamātraṃ na sarvākāṅkṣāmayasya rāgatattvasya samānaḥ  / tathā jānāmi iti kiñcideva purovarti ghaṭādikaṃ na punardūravyavahitaṃ vastviti vidyātattvam  / śuddhavidyāpekṣayā kārikāyāmavidyeti kathitaṃ na punarvedanābhāvāt  / māyāsahitam bhedaprathāyuktametatkañcukaṣaṭkaṃ paśoriti  // 17  //

kathametasya kañcukaṣaṭkasyāṇuṃ pratyantaraṅgatvamityāha

kambukamiva taṇḍulakaṇaviniviṣṭaṃ bhinnamapyabhidā  /
bhajate tattu viśuddhiṃ śivamārgaunmukhyayogena  // 18  //

vāstavena vṛttena bhinnamapi kambukam yathā abhidā taṇḍulakaṇaviniviṣṭam ityabhedena taṇḍulāntaścotaṃ bhāsate nipuṇairapi yatnataḥ prakṣipyamāṇaṃ taṇḍulasyāntaraṅgatvānna pṛthagavatiṣṭhate tathaitanmāyādikañcukamaṇostaṇḍulasthānīyasyāntaraṅgatvātkambukasthānīyaṃ vyatiriktamapyavyatiriktatayā pūrṇasaṃvitsvarūpamācchādya sthitamiti śeṣaḥ  / evamapi durnivāraṃ kathaṃ tadvilīyata ityāha bhajate ityādi  / tuḥ viśeṣe nānyo 'tropāyaḥ  / śivasya svātmamaheśvarasya yo 'sau mārgaḥ paramādvayacidānandaikaghano 'smi mamaivedaṃ viśvaṃ svaśaktivijṛmbhaṇamātramiti yā svātmasvarūpavibhūtipratyavamarśarūpā saraṇiḥ tadaunmukhyam dārḍhyena tannibhālanaparatvaṃ sa eva yogaḥ pūrṇarūpatayāṇoḥ svātmani svasvarūpatvena saṃbandhastena tat uktasvarūpaṃ kañcukaṃ viśeṣeṇa śuddhiṃ bhajate svayameva niḥśeṣeṇa vilayaṃ sevate  / etaduktaṃ syādyadā parameśvaraśaktipātaviśuddhahṛdayaḥ paśurbhavati tadāhameva maheśvara iti svātmajñānāvirbhāvātsvayameva paśutvāpādakaṃ kañcukāvaraṇaṃ vilīyate na punaratra svātmajñānaṃ vihāya māyīyaṃ kiñcinniyatiśaktisamutthaṃ karma pragalbhata iti  // 18  //

evaṃvidhasyāṇorbhoktuśca bhogyena bhāvyamiti prādhānikaṃ tattvasargamāha

sukhaduḥkhamohamātraṃ niścayasaṃkalpanābhimānācca  /
prakṛtirathāntaḥkaraṇaṃ buddhimano 'haṅkṛti kramaśaḥ  // 19  //

sattvarajastamasāṃ yat sukhaduḥkhamohātmakam sāmānyaṃ rūpamaṅgāṅgibhāvo yatra nopalabhyate sā mūlakāraṇam prakṛtiḥ  / prakṛteranantaraṃ kāryarūpamantaḥkaraṇamāha niścaya ityādi  / niścayaḥ idametādṛgiti saṃkalpanam mananam abhimānaḥ mamateti krameṇa buddhirmano 'haṅkāraḥ ityevaṃrūpaṃ tritayam antaḥkaraṇam aṅgāṅgibhāvena guṇānāṃ kāryaṃ bhūtendriyādyapekṣayā ca kāraṇamiti  // 19  //

bāhyakaraṇamāha

śrotraṃ tvagakṣi rasanā ghrāṇaṃ buddhīndriyāṇi śabdādau  /
vākpāṇipādapāyūpasthaṃ karmendriyāṇi punaḥ  // 20  //
vakṣyamāṇe śabdādau viṣaye jñānapradhānāni śrotrādīni pañcendriyāṇi kriyāpradhānāni cendriyāṇi pañca vāgādīni  / vacanādānaviharaṇavisargānandātmakāḥ karmendriyāṇāṃ viṣayāḥ  / ityubhayathā ca śṛṇomi kathayāmītyahaṅkārānugamādahaṅkārakāryāṇi  // 20  //

eṣāṃ śabdādiviṣayasvarūpaṃ kathayati

eṣāṃ grāhyo viṣayaḥ sūkṣmaḥ pravibhāgavarjito yaḥ syāt  /
tanmātrapañcakaṃ tacchabdaḥ sparśo maho raso gandhaḥ  // 21  //

jñeyakāryatayā svīkāryaḥ ya eṣām indriyāṇām viṣayaḥ gocaraḥ syāt  / sa kīdṛśaḥ pravibhāgavarjitaḥ viśeṣeṇa bahiṣkṛtaḥ sāmānyātmā sūkṣmaḥ yo 'rtho bhavettadeva śabdādi sāmānyarūpam tanmātram śabdasāmānyaṃ śabdatanmātramiti  / evamanyāni  / viṣayaviṣayiṇoḥ parasparāpekṣitvādindriyavadidamapi tanmātrapañcakam āhaṅkārikameveti  // 21  //

viṣayāṇāṃ parasparasāṃkaryeṇa pṛthivyādīni kāryamityāha

etatsaṃsargavaśātsthūlo viṣayastu bhūtapañcakatām  /
abhyeti nabhaḥ pavanastejaḥ salilaṃ ca pṛthvī ca  // 22  //

eteṣāṃ saṃsargavaśāt parasparasaṃgharṣasāmarthyādyo viśeṣaḥ sthūlo viṣayaḥ sa eva bhūtarūpatāṃ yāti  / tathā hi śabdatanmātrācchabdaviśeṣo nabho jāyate śabdasparśābhyāṃ pavano rūpasaṃyuktābhyāmetābhyāṃ tejaścaibhyo rasayuktebhyaścāpo gandhasaṃyuktebhyaśca pṛthvīti pañca mahābhūtāni kāryaṃ kāraṇānuguṇam iti kṛtvaikottaraguṇānīti  / evameṣā prakṛtiḥ kāryakāraṇātmā puruṣasya parameśvarasyecchayā bhogyatayā pravṛtteti ṣaṭtriṃśattattvātmakaṃ jagadvibhajya pratitattvaṃ nirūpitam  // 22  //

māyākañcukavatprakṛteḥ kañcukatāṃ puruṣaṃ pratyāha

tuṣa iva taṇḍulakaṇikāmāvṛṇute prakṛtipūrvakaḥ sargaḥ  /
pṛthvīparyanto 'yaṃ caitanyaṃ dehabhāvena  // 23  //

ayamapi prādhānikaḥ sargaḥ dharāparyantaḥ taṇḍulakaṇam yathā tuṣaḥ dhānyacarma āvṛṇute samācchādayati tathaiva māyākañcukena kambukasthānīyena samāvṛtam caitanyam punarapi tuṣasthānīyena dehabhāvena etatsamāvṛṇute tatpratiprākāratayā sthagayati  / atreme pramātāraḥ kalābhirindriyamātrābhirviśeṣāntābhirdehasvabhāvāḥ sakalā ityucyante viśeṣavarjitā videhāḥ pralayākalā iti ca  / evaṃ śivādisakalāntapramātṛsaptakasanāthaṃ rudrakṣetrajñādhiṣṭhitaṃ jagaditi  // 23  //

kañcukatritayasya parasūkṣmasthūlarūpatāmāha

paramāvaraṇaṃ mala iha sūkṣmaṃ māyādikañcukaṃ sthūlam  /
bāhyaṃ vigraharūpaṃ kośatrayaveṣṭito hyātmā  // 24  //

caitanyasya svasvarūpāpahastanasatattvākhyātirevāṇavaḥ malaḥ āntaraḥ svarṇasya kālikeva param antaraṅgam āvaraṇam chādanaṃ tādātmyena sthitatvāt  / māyādi vidyāntaṃ kañcukaṣaṭkam sūkṣmam ātmana āvaraṇaṃ taṇḍulasya kambukamivāvaraṇaṃ pṛṣṭhapātitvenāste yena bhedamayī jñatvakartṛtvādiprathā prathata ityeṣa māyīyo malaḥ  / etadapekṣayā bāhyam tuṣasthānīyaṃ prādhānikaṃ śarīrasattālakṣaṇamāvaraṇam sthūlam tvaṅmāṃsādirūpatvādeṣa tṛtīyaḥ kārmo malo yena pramātā śubhāśubhakarmasaṃcayabhājanaṃ bhavati  / evamanena parasūkṣmasthūlarūpeṇa kośatrayena veṣṭitaḥ vikasvaro 'pi ghaṭākāśavatsaṃkucitīkṛtaḥ ātmā ityaṇuriti paśurityucyate  // 24  //

etatsaṃbandhādupahata iva bhavatītyāha

ajñānatimirayogādekamapi svaṃ svabhāvamātmānam  /
grāhyagrāhakanānāvaicitryeṇāvabudhyeta  // 25  //

eṣa kośatrayasaṃbaddha ātmātmākhyātyandhakārasaṃbandhāt ekamapi advayasvabhāvamapi svam nijaṃ nānyasmādupanatam ātmasvabhāvam caitanyamātmasattālakṣaṇaṃ svarūpaṃ pramātṛpramāṇaprameyanānāracanāprapañcena jānātyabhedaviparītena bhedenābhimanyata iti yāvat  / yathā rekhātimiropahataḥ puruṣa ekamapi candraṃ paśyandvau candrāvimau nabhasi sta iti paricchindamṝÉllokamapi darśayati dvau candrāvimau paśyeti  / vastuvṛttenaika evāsau candra iti timiravaśāttathā bhāsate yenodvegalakṣaṇāmānandalakṣaṇāṃ vārthakriyāṃ sa taimirikaḥ prāpnoti  / tathaivājñānatimiraprāptabhedaprathaḥ sarvaṃ svātmano 'bhinnamapi bhedena vyavaharanbhinnaṃ karmaphalamarthayate yena bhūyobhūyaḥ svarganirayādibhogabhāgbhavati  / ataścājñānasya timireṇa rūpaṇā viparītābhāsanāt  // 25  //

ātmādvayaṃ dṛṣṭāntena nidarśayati

rasaphāṇitaśarkarikāguḍakhaṇḍādyā yathekṣurasa eva  /
tadvadavasthābhedāḥ sarve paramātmanaḥ śaṃbhoḥ  // 26  //

rasādayaḥ ikṣubhedāḥ yathā ekaḥ evekṣurasaḥ paramārthataḥ sarvatra mādhuryānugamāttathaiva jāgradādi avasthābhedāḥ grāhyagrāhakaprapañcarūpāḥ sarve viśeṣāḥ paramātmanaḥ svasvabhāvasya śaṃbhoḥ caitanyamaheśvarasyaiva  / yataḥ sa eva bhagavānsarvasya svātmabhūtaḥ svasvātantryāttāṃ tāmapi bhūmikāṃ samāpannastathā grāhyagrāhakādyavasthāviśiṣṭaḥ prathate yathekṣuraso na punaḥ svātmanastasmādbhinnaṃ kiñcidastīti sa eka eva sarvāvasthāsu saṃvidanugamāt  / itthaṃ sarvatraikarūpatādarśanātpramātā sarvadṛśvā bhavati  / yathāha śrīśaṃbhubhaṭṭārakaḥ

eko bhāvaḥ sarvabhāvasvabhāvaḥ sarve bhāvā ekabhāvasvabhāvāḥ  /
eko bhāvastattvato yena dṛṣṭaḥ sarve bhāvāstattvatastena dṛṣṭāḥ  //

iti  / bhagavadgītāsvapi

sarvabhūteṣu yenaikaṃ bhāvamakṣayamīkṣate  /
avibhaktaṃ vibhakteṣu tajjñānaṃ viddhi sāttvikam  //

iti  // 26  //

tīrthāntaraparikalpitastu bhedaḥ saṃvṛtyarthamabhyupagato 'pi na satyabhūmāvavakalpata ityāha

vijñānāntaryāmiprāṇavirāḍdehajātipiṇḍāntāḥ  /
vyavahāramātrametatparamārthena tu na santyeva  // 27  //

vijñānam iti bodhamātrameva kevalamanupādhi nāmarūparahitamapyanādivāsanāprabodhavaicitryasāmarthyānnīlasukhādirūpaṃ bāhyarūpatayā nānā prakāśata iti vijñānavādinaḥ  /

puruṣa evedaṃ sarvam  ...  /

neha nānāsti kiñcana  ...  /

iti nyāyena antaryāmi sarvasyeti paraṃ brahmaivānādyavidyāvaśādbhedena prakāśata iti brahmavādinaḥ  / atrobhayatrāpi vedanasya svātantryaṃ jīvitabhūtaṃ viśvanirmāṇaheturiti na cetitam  /

anye prāṇa brahmavādinastu yathāprāṇanameva viśvamāgūrya vartate nahi prāṇanādṛte 'nyatkiñcidbrahmaṇo rūpamiti savimarśaṃ śabdabrahmetyāhuḥ  /

apare pratipannā yathā virāḍdeham iti vairājamapi brahmaṇaḥ satyabhūtamiti  / yathoktam

yasyāgnirāsyaṃ dyaurmūrdhā khaṃ nābhiścaraṇau kṣitiḥ  /
sūryaścakṣurdiśaḥ śrotre tasmai lokātmane namaḥ  //

ityevamādi  /

jātiḥ iti mahāsattāsāmānyalakṣaṇaṃ sarvaguṇāśrayaṃ vastu paramārthasaditi vaiśeṣikādayo bruvate  /

anye piṇḍāḥ iti vyaktaya eva paramārthasatyo nahi sāmānyaṃ nāma kiñcidekamanekaguṇāśrayaṃ prakāśate nāpyupapadyate veti vyaktīnāmeva vyavahāraḥ parisamāptaḥ kiṃ sāmānyeneti nānāvṛttivikalpaiḥ sāmānyaṃ vivadamānā vyaktayo nānuyantyanyadanuyāyi na bhāsata ityevamādi bahu bruvanto jātirna paramārtha iti pratipannā ityeva vijñānādiḥ piṇḍo 'nte yeṣāṃ bhedānāṃ te tathoktāḥ vyavahāramātrametat ityasminsvātantryavāde prakāśamānasya vastuno 'napahnavanīyatvādete bhedāḥ saṃvṛtisatyatvena prakāśante paramārthena tu na santyeva iti na punaḥ satattvatayaite bhedāstīrthāntaraparikalpitabhedā vidyamānā eva  / tasmādeka eva paramaprakāśaparamārthaḥ svatantraścaitanyamaheśvara itthamitthaṃ cakāsti yato 'nyasyaitadvyatiriktasyāprakāśarūpasya prakāśamānatābhāvāt  / yaduktam

tīrthakriyāvyasaninaḥ svamanīṣikābhirutprekṣya tattvamiti yadyadamī vadanti  /
tattattvameva bhavato 'sti na kiñcidanyatsaṃjñāsu kevalamayaṃ viduṣāṃ vivādaḥ  //
iti  // 27  //

idānīṃ bhrānterasadarthapratipādanasāmarthyena nidarśanamāha

rajjvāṃ nāsti bhujaṅgastrāsaṃ kurute ca mṛtyuparyantam  /
bhrāntermahatī śaktirna vivektuṃ śakyate nāma  // 28  //

bhrānteḥ pūrṇatvākhyātirūpāyā atādrūpyapratibhāsane mahatī śaktiḥ uttamaṃ sāmarthyam na kenacit vivektuṃ śakyate na kenacidvicārayituṃ pāryata iti yāvat  / yathā vastuvṛttena rajjuḥ paridṛśyamānā dīrghatvakuṇḍalinīrūpatvabhramātsarpo 'yamityadhyavasātṝṇāṃ rajjudravye 'pyasadarthapratibhāso 'yaṃ sarpādhyavasāyaḥ sadarthapratibhāsādbhayaṃ maraṇāvasāyaṃ vidadhāti  / anubhavasiddhamapyetatsthāṇuṃ bhūtamiti matvā svayaṃ bhīṣaṇīyaṃ vākāraṃ samullikhya bhrāntāḥ santo hṛdbhaṅganāśaṃ ke nāma na yātā iti vibhrama evāpūrṇatvaprathane heturiti  // 28  //

etatprakṛte samarthayate

tadvaddharmādharmasvarnirayotpattimaraṇasukhaduḥkham  /
varṇāśramādi cātmanyasadapi vibhramabalādbhavati  // 29  //

evaṃ yathā rajjuḥ paramārthasatī bhrāntyā sarpatayā vimṛṣṭāpi sarpagatāmarthakriyāṃ karoti tathaiva dehātmamānināṃ cetasi dharmādyasadapi tattvato 'vidyamānam vibhrama vaśānmāyāvyāmohasāmarthyādeva bhavati etadeva tattvamiti bhrāntyā sattāṃ labhate  / dharmaḥ aśvamedhādiḥ adharmaḥ brahmahananādiḥ svaḥ niratiśayā prītiḥ nirayaḥ yātanā utpattiḥ janma maraṇam janmābhāvaḥ sukham āhlādaḥ duḥkham rājasaḥ kṣobhastathā varṇaḥ brāhmaṇo 'smītyādi āśramaḥ brahmacārītyādi ādi grahaṇāttapaḥpūjavratānīti sarvaṃ kalpanāmātrasāraṃ vibhramavijṛmbhitameva māyāśaktyā dehādyātmatayābhimanyate  / etatsarvaṃ bhrānteḥ prabhavati yayānavaratasvarganarakajanmamaraṇaprabandhabhājaḥ paśavo na punaḥ paramārthataḥ svātmano 'navacchinnacidānandaikaghanasya dharmādharmādikaṃ kiñcidvidyata iti  // 29  //

evamasadarthapratibhāsane bhrānteḥ sāmarthyaṃ vicārya tadutpattimāha

etattadandhakāraṃ yadbhāveṣu prakāśamānatayā  /
ātmānatirikteṣvapi bhavatyanātmābhimāno 'yam  // 30  //

etattadandhakāram ityeṣā sā samanantarapratipāditā viśvamohinī pūrṇatvākhyātirūpā bhrāntiḥ yadbhāveṣu pramātṛprameyarūpeṣu viśvavartiṣu padārtheṣu prakāśamānatayā iti

... nāprakāśaḥ prakāśate  /

iti prakāśamānatānyathānupapattyā prakāśaśarīrībhūteṣu ātmanaḥ caitanyamaheśvarādapṛthagbhūteṣu satsvapi yaḥ ayam atirekeṇāmī bhāvā grāhyā bāhyā matto bhinnāśceti anātmābhimānaḥ vāstavacidrūpatāpahastanena yatteṣvavāstavaṃ jaḍatvāpādanam  / ayamāśayo bhāvaprakāśane 'nyasyāprakāśarūpasya bāhyavāsanāderhetoranupapadyamānatvātsvātmaprakāśa eva svatantro 'rthānnīlasukhādinā prakāśate 'taḥ pramātṛprameyarūpatayā citsvarūpo 'hameva prakāśa iti yadvāstavaṃ rūpaṃ tanna prakāśate kevalamevāvāstavo bhedaḥ prathata iti tāttvikaprathanābhāvādbhrānterandhakāreṇa rūpaṇamiti  // 30  //

ātmanyanātmābhimānapūrvo 'nātmanyātmābhimāno bhavatīti pratipādayanbhrānteḥ sutarāṃ moharūpatāmāha

timirādapi timiramidaṃ gaṇḍasyopari mahānayaṃ sphoṭaḥ  /
yadanātmanyapi dehaprāṇādāvātmamānitvam  // 31  //

ādau tāvadekasaṃvitsatattveṣvapi bhāveṣu bhedamayaṃ jāḍyāpādanamakhyātitimireṇa kṛtaṃ yatsvātmano 'bhinnānāṃ bhāvānāṃ tato bhedena prathanamata eva timiramiva timiram akhyātiḥ  / yathaiko 'pi candraḥ cakṣusthena rekhātimireṇa dvidhā bhāsyate dvau candrāviti tathaivākhyātitimiramekamapi sarvaṃ svātmasvarūpaṃ vastu bhedenānātmarūpaṃ prakāśitavat  / evamavasthite timiram aparamāyātaṃ mohānmoho 'yamāpatitaḥ gaṇḍasyopari piṭakodbhavaśca yat akhyātyapahastitacitsvarūpeṣvapi viśvavartiṣu padārtheṣu jāḍyamāpāditeṣu madhyāduddhṛte vyatirikte jaḍe dehaprāṇādau vedyakhaṇḍe kṛśo 'haṃ sthūlo 'haṃ kṣudhito 'haṃ sukhyasmi na kiñcidahamiti pramātṛtayānātmabhūte ātmamānitvam atādrūpye tādrūpyapratipattiretadativaiśasam  / yadi tāvadātmābhimānena vinā vaiśasamasti tannīlasukhādiṣvapyastu mā vā kutrāpi bhūdyatpunaḥ katipaye jaḍe dehādau loṣṭaprāya ātmatayāhantārasābhiṣeko 'nyatra nīlasukhādāvidantayānātmapratipādanameṣa eva pūrṇaḥ saṃsāraḥ śocanīyo yadabhimānopanato dvandvābhighātaḥ karṣati paśūniti  / yaduktaṃ mārkaṇḍeyapurāṇe yoginyā madālasayā

yānaṃ kṣitau yānagataśca deho dehe 'pi cānyaḥ puruṣo niviṣṭaḥ  /
mamatvamurvyāṃ na tathā yathā sve dehe 'timātraṃ ca vimūḍhataiṣā  //

iti  // 31  //

evamakhyātivaśānmithyāvikalpairitthamātmānaṃ badhnātītyāha

dehaprāṇavimarśanadhījñānanabhaḥprapañcayogena  /
ātmānaṃ veṣṭayate citraṃ jālena jālakāra iva  // 32  //

akhyātyapahastitacaitanyaḥ sarvaḥ pramātā svotthairvikalpanigaḍairvyāpakamapi ātmānaṃ veṣṭayate  / kathamityāha deha ityādi  / dehaprāṇayorvimarśanaṃ dhiyo jñānam niścayaḥ nabhasāṃ prapañcaḥ vistāraḥ tadyogena dehādivikalpasaṃbandhena  / yathā kṛśaḥ sthūlo rūpavānpaṇḍitaścāsmīti bālāṅganāpāmarāḥ kārṣikā itthaṃ svavikalpena dehamevātmatvena pratipannāḥ kiñcidvivecakaṃmanyāḥ  /

dehastāvadihaiva pralīyate kuto 'syātmatvamato yaḥ kṣudhitaḥ pipāsitaḥ so 'hamiti prāṇātmamāninaśca vivecakaṃmanyatarāḥ  /

dehaprāṇau jaḍau loṣṭādivatkuto 'nayorātmabhāvastataḥ sukhyahaṃ duḥkhyahamiti yaḥ sukhaduḥkhādi cetate sa ātmeti puryaṣṭakābhimānino mīmāṃsakādayo 'pi vivecakatamāśca  /

etatsukhaduḥkhādyapi buddhidharmaḥ kathamātmatayā vaktuṃ śakyastato dehaprāṇadhīvikalpānāṃ yatrābhāvaḥ sa ātmeti śūnyābhimāninaḥ  / evaṃ yatkiñcididaṃ bhāti tannāhamityaprathārūpaṃ śūnyameva sarvāpohanasvabhāvamātmeti nabhaḥśabdenoktaḥ  / tadapi śūnyaṃ yadā samādhānāvasare vedyīkurvata etadapi śūnyaṃ vayaṃ na bhavāmastadā śūnyāntaramātmatvena vidadhānā neti neti brahmavādyabhyupagatatattacchūnyaparityāgena tāṃ tāṃ śūnyātmatāṃ parigṛhṇantīti nabhaḥprapañcaḥ kārikāyāṃ nirūpitaḥ  / itthaṃ saṃvitsvarūpāparyavasānācchūnyātmamānino yoginaḥ suṣuptaguhānimagnā jaḍātmāno bhrāntā evātmānaṃ saṃvitsvarūpamapi jāḍyenānubadhnanti  / citram ityāścaryametadyaduta vaiśasaṃ naitatsvayaṃ kartuṃ pāryata iti  / atra dṛṣṭāntamāha jālena ityādi  / yathā jālakāraḥ kaścitkṛmirvā svanirmitena phenena jālamāvaraṇaṃ nirmāya sarvato gatam ātmānaṃ veṣṭayate svaṃ svātmanidhanāya badhnāti yenottaratra tatraiva nidhanaṃ yāti tathā dehādyātmamānī tu svavikalpakalpitairahaṃ mameti vikalpaiḥ svātmānameva badhnāti  / tathā ca bauddhāḥ

satyātmani parasaṃjñā svaparavibhāgātparigrahadveṣau  /
anayoḥ saṃpratibaddhāḥ sarve doṣāḥ prajāyante  //

ityāhuḥ  // 32  //

kathameṣa durnivāro mahāmoho dehādipramātṛtāsamutthaḥ pralīyata iti bhagavatsvātantryamevātra heturityāha

svajñānavibhavabhāsanayogenodveṣṭayennijātmānam  /
iti bandhamokṣacitrāṃ krīḍāṃ pratanoti paramaśivaḥ  // 33  //

svasya ātmanaścaitanyalakṣaṇasya yat jñānam svasvātantryāvagamastasya vibhavaḥ dehādyabhimānavigalanena yaccitsvarūpe parāhantācamatkārarūpasya svasvātantryasya sphītatvaṃ cidānandaikaghanaḥ svatantro 'smīti tasya bodhasvātantryasvarūpasya vibhavasya bhāsanam prakāśaḥ sarvo mamāyaṃ vibhava iti bāhyatayābhimatasya sarvasya svātmanyeva svīkārastasya yogaḥ evaṃpariśīlanakrameṇātmani yadvimarśadārḍhyamevametena svajñānavibhavabhāsanayogena nijamātmānam nānyata upanataṃ caitanyasvabhāvam udveṣṭayate dehaprāṇapuryaṣṭakaśūnyaparāmarśanānigaḍairyo veṣṭita āsīttameva caitanyasvarūpaḥ svatantro 'smīti vimarśanena punarapi bhagavānevodveṣṭanaṃ vigataveṣṭanaṃ kuruta iti  / evamakhyātibalādāgataṃ svātmano dehādyāvaraṇaṃ tatpunarapi khyātibalādvinaśyatīti svavikalpakalpita iyāndoṣa iti śrīmadgranthakṛtā tantrasāre nirūpitam

yo niścayaḥ paśujanasya jaḍo 'smi karmasaṃpāśito 'smi malino 'smi parerito 'smi  /
ityetadanyadṛḍhaniścayalābhayuktyā sadyaḥ patirbhavati viśvavapuścidātmā  //

iti  / kimiti badhnāti muñcati ca bhagavānityāha iti bandha ityādi  / iti prākpratipāditena krameṇa bhagavānsvatantraḥ paramaśivaḥ pūrṇacidānandaikaghanalakṣaṇaḥ svarūpagopanasatattvakrīḍāśīlatvādakhyātyavabhāsanapūrvaṃ svātmānameva dehādipramātṛtāpannaṃ vidhāya svarūpaṃ pracchādya ca bandhaṃ vidadhāti tathaiva punaḥ svecchātaḥ svātmajñānaprakāśakrameṇa dehādipramātṛtābandhaṃ nivārya sa eva taṃ svātmānaṃ mocayatītyubhayathā bandhamokṣacitrām saṃsārāpavargasvarūpāścaryamayīm krīḍām khelām pratanoti vistārayatyekākī na ramāmyahamiti  / svabhāva evaiṣa devasya yattāṃ tāmapyavasthāmāpannaḥ svarūparūpaḥ sansarvatrānubhavitṛtayā prathata ityetadeva svātantryam  // 33  //

na kevalametadyāvadaparaḥ kaścidavasthāviśeṣaḥ svasminrūpe viśrānta evāvabhāsyate bhagavatetyāha

sṛṣṭisthitisaṃhārā jāgratsvapnau suṣuptamiti tasmin  /
bhānti turīye dhāmani tathāpi tairnāvṛtaṃ bhāti  // 34  //

viśvāpekṣayā ye sargādayo māyāpramātṛgatāśca ye jāgradādayo 'vasthāviśeṣāsta ubhayathaitasminbhagavatyānandaghane turīye dhāmani caturthe pūrṇāhantāmaye pade bhānti tadviśrāntāḥ santaḥ svarūpasattāṃ kalpitapramātrapekṣayā bāhyatayā labhante  / parameśvarabhittau yanna prakāśate tadbāhyatayāpi na prakāśate 'taḥ

triṣu caturthaṃ tailavadāsecyam  /

iti sarvāsvavasthāsu turīyaṃ rūpamanusyūtatvena sthitamiti paramārthaḥ  / etāvatā tatra taiḥ svarūpamācchāditaṃ syānna vetyāha tathāpi tairnāvṛtaṃ bhāti iti  / itthamapi taiḥ svarūpasattārthamāvṛtamapi tebhyaḥ samuttīrṇatayā sarvānubhavitṛtayā sarvatrāvabhāsata eva na punastadāvaraṇena pūrṇasvarūpatāṃ tatra tirodhatta iti śivadhāma sarvāvasthāsvapi sadaiva paripūrṇam  // 34  //

vedāntabhāṣābhirjāgradādīnāṃ trayāṇāṃ svarūpaṃ vyavaharaṃstadanusyūtamapi tataḥ paraṃ turīyamāvedayati

jāgradviśvaṃ bhedātsvapnastejaḥ prakāśamāhātmyāt  /
prājñaḥ suptāvasthā jñānaghanatvāttataḥ paraṃ turyam  // 35  //

jāgradavasthaiva viśvam brahmaṇo vairājaṃ svarūpaṃ kutaḥ bhedāt śabdādiviṣayapañcakasya bāhyatayā parameśvarasṛṣṭasyaiva sarvapramātṝṇāṃ cakṣurādīndriyapravṛtterityekasyaiva brahmaṇo viṣayaviṣayibhāvena sthitasya nānendriyajñānavaicitryam  / ata eva śivasūtreṣu

jñānaṃ jāgrat  /

iti  / eṣā brahmaṇo virāḍavasthā gīyate  / yacchrutiḥ

yo viśvacakṣuruta viśvatomukho viśvatohasta uta viśvataspāt  /
saṃbāhubhyāṃ namate saṃyajatrairdyāvāpṛthivī janayandeva ekaḥ  //

iti  / tathā svapnaḥ tejo 'vasthā brahmaṇaḥ  / kuta ityāha prakāśamāhātmyāt iti  / svapne bahiṣkaraṇāni śabdādau viṣaye tāvanna pragalbhante nāpi tatra bāhyaṃ śabdādikaṃ nāma kiñcitparamārthasadvidyate nāpi bāhyatayādhyavasāyasyāvidyādi kiñcidbhinnamabhinnaṃ vā kāraṇāntaramupalabhyate yuktyā vicāryamāṇaṃ vopapadyate 'tha ca svapne sarvaṃ prakāśate 'ta idamarthabalādāyātaṃ yatsa eva bhagavānsvasvabhāvo devastattatpramātṛtāṃ samāviṣṭaḥ svapnāyamānaḥ svātmānameva prakāśasvātantryādgṛhanagarāṭṭālādyanekapramātṛvaicitryarūpatayā pravibhajya pratipramātṛ svapne 'sādhāraṇameva viśvaṃ prakāśayatyeveti brahmaṇaḥ svātantryaṃ svapna eva brahmavādibhirabhyupagatam  / yato vedānteṣvidamuktam

pravibhajyātmanātmānaṃ sṛṣṭvā bhāvānpṛthagvidhān  /
sarveśvaraḥ sarvamayaḥ svapne bhoktā prakāśate  //

iti prakāśamāhātmyamevātra heturataḥ svapno brahmaṇastejo 'vastheti  / tathā prājñaḥ suptāvasthā iti  / sarvapramātṝṇāṃ yā suptāvasthā suṣuptaṃ sā prājñaḥ iti brahmaṇaḥ prājñāvastheti  / yataḥ sarvapramātṝṇāṃ grāhyagrāhakaprapañcavilayānmahāśūnyatvarūpe grāhyādivilaye saṃskāraśeṣe suṣupte viśvasya bījabhūtasya brahmaṇa eva prajñā brahma prajñātṛtayāntaratamamavaśiṣyata iti yāvat  / sarvasya pramāturnīlasukhādiviśvavaicitryaprathāyāḥ saiṣā saṃskārabhūmistataḥ prabuddhasya prāganubhūtavadvyavahāradarśanādanyathā yadyasyāṃ bhūmau sthiraṃ prajñātṛsvabhāvaṃ sarvakroḍīkāreṇa brahma na prākāśiṣyata kutastata utthitasya pramātuḥ prāganubhūtavastunastathaiva tadityanubhūtacaratvena smṛtirudapatsyata nāpi sukhamahamasvāpsaṃ duḥkhamahamasvāpsaṃ gāḍhamūḍho 'hamāsamityanubhavaḥ prādurabhaviṣyaditi  / tathā ca bhaṭṭadivākaravatsaḥ

sarve 'nubhūtā yadi nāntararthāstvadātmasātkārasurakṣitāḥ syuḥ  /
vijñātavastvapratimoṣarūpā kācitsmṛtirnāma na saṃbhavettat  //

iti  / itthaṃ suṣuptaṃ cinmayameva brahmaṇaḥ prājñāvastheti gīyate  / kutaḥ jñānaghanatvāt iti  / suṣuptaturyayoḥ sādhāraṇo 'yaṃ heturityubhayatra yojyam  / eṣā suṣuptabhūmiḥ jñānaghanā prakāśamūrtiḥ kevalaṃ viśvapralayasaṃskāreṇa dhyāmalā satī śuddhacinmayī na bhavatīti  / yaduktaṃ spandaśāstre

jñānajñeyasvarūpiṇyā śaktyā paramayā yutaḥ  /
padadvaye vibhurbhāti tadanyatra tu cinmayaḥ  //

iti  / tathā tataḥ paraṃ turyam iti  / tasmātsuṣuptāt param anyanniḥśeṣapāśavavāsanāsaṃskāraparikṣayācchuddhapūrṇānandamayaṃ brahmaṇasturīyaṃ rūpamanuguṇaṃ nāma  / yadatra nāmānvarthaṃ na kiñcidupapadyate 'to vyākhyātasyāvasthātrayasya viśrāmabhūtaṃ sarvāntaratamatvenānusyūtamiti catuḥsaṃkhyāpūraṇena turyam iti pūraṇapratyayena saṃkhyāvyapadeśo 'tra kṛtaḥ kathamavasthātrayasyānusyūtamapi tataḥ parametadityāha jñānaghanatvāt iti  / yato jāgradādayo 'vasthāḥ sarvā bhedapravaṇatvātpramātṝṇāmajñānamayyasturīyaṃ grāhyagrāhakakṣobhapralayasaṃskāraparikṣayājjñānaghanaprakāśānandamūrtyatastadantaḥsthamapi tābhyo 'vasthābhyaścinmayatayā samuttīrṇatvāt param anyaditi  / evamavasthāvicitraṃ paramādvayasvabhāvaṃ svatantraṃ brahmaiva pūrṇaṃ vijṛmbhate  // 35  //

evamapi śuddhasya paramātmanaḥ sarvapramātranusyūtatvena sthitatvādavaśyaṃ pramātṛgaṇagatamakhyātimālinyamāyātīti yattanneti dṛṣṭāntenāvedayati  /

jaladharadhūmarajobhirmalinīkriyate yathā na gaganatalam  /
tadvanmāyāvikṛtibhiraparāmṛṣṭaḥ paraḥ puruṣaḥ  // 36  //

yathā meghadhūmadhūlīpuñjairākāśasthairapyākāśapṛṣṭhamamalinaṃ svabhāvato na malinaṃ vidhīyate nāpi nityatāvaitatyakhaṇḍanāṃ vā nīyate kevalaṃ darpaṇapratibimbavattattadavasthāvicitraṃ gaganaṃ gaganameva sarvadā tathātvena pratyabhijñānāt  / tadvat iti tathaiva māyāvikṛtibhiḥ akhyātisamutthairvikārairnānāpramātṛgatairjanmamaraṇādyanekavicitrāvasthāmayairbhagavatsthairapi bhagavānna parāmṛṣṭaḥ na tairapahṛtasvarūpo yataḥ sa eva paraḥ puruṣaḥ iti sarvapuruṣāṇāṃ jīvānāmādya ullāso viśrāntisthānaṃ ceti sarvānubhavitṛtayā sadā sphuratīti paraśabdena nirdiṣṭaḥ  / tasmānna svasamutthairaprakāśarūpairmāyāvikārairaindrajālikavadbhagavataḥ kācitkhaṇḍaneti parameṣṭhinā nareśvaraviveke 'pyuktam

yadyapyarthasthitiḥ prāṇapuryaṣṭakaniyantrite  /
jīve niruddhā tatrāpi paramātmani sā sthitā  //
tadātmanaiva tasya syātkathaṃ prāṇena yantraṇā  /

iti  // 36  //

nanvekacinmātraparamārthā api puruṣā viśiṣṭasukhaduḥkhamohajanmamaraṇādinānāvasthāvaicitryabhājaśceti kathametaditi dṛṣṭāntamāha

ekasminghaṭagagane rajasā vyāpte bhavanti nānyāni  /
malināni tadvadete jīvāḥ sukhaduḥkhabhedajuṣaḥ  // 37  //

yathā ekasmin kumbhākāśe dhūlipuñjasamācchādite nānyāni ghaṭākāśāni vimalānyapyākāśatvāviśeṣāt malināni rajasācchāditāni bhavanti  / vimalaṃ vyāpakaṃ nityamapi hi gaganaṃ yādṛśaṃ ghaṭasaṃkocasaṃkucitaṃ bhavettādṛśameva tasyaiva ghaṭasya tadbhavati na punaḥ sarvāṇi ghaṭapaṭādyākāśāni kṛṣṇāgurudhūpitāni mṛgamadādhivāsitāni viṭhiragandhīni vaikākāśasvarūpatvātsaṃkīryante svagataghaṭādikṛtavicchedātpratyutaikatrāpi gagane vāstave sthite ghaṭādayaḥ svagatabhittisaṃkocaniyantritāḥ santo nānāgaganavaicitryaṃ prathayanti  / itthaṃ ghaṭakṛtaḥ saṃkoca eva gaganatayā tathā viśiṣyate tathātvenārthakriyākāritvānna punaḥ gaganasyaitatkiñcidghaṭagataṃ mālinyādi svarūpatirodhānāya bhavati nāpi parasparaṃ ghaṭādyavacchinnānāṃ gaganānāṃ vyāmiśraṇeti  / tadvat iti tathaivāmī jīvāḥ puruṣā ekacinmātraparamārthā api pārameśvaryā māyāśaktyāṇavamāyīyaprākṛtakośatrayāveṣṭanena pūrṇaṃ vyāpi cidānandaikaghanaṃ svarūpamapahastya parimitīkṛtā yenaikacaitanyātmāno 'pi svagatakośatrayavicchedadaurātmyātparasparaṃ bhinnāśceti ghaṭapaṭādyākāśavat  / itthaṃ māyīyakośakṛto viccheda eva jīvatayā vyavahriyate na punaḥ parameśvare 'navacchinnacidānandaikaghane jīvāḥ puruṣā ātmāno 'ṇava iti darśanāntaradṛṣṭaḥ kaścidvyavahāraḥ  / evamāṇavādikośāvacchinnā jīvā anādivicitrakarmamalavāsanādhivāsitanānādehā nānāśayā nānāpuṇyapāpasvarganarakasukhaduḥkhajanmamaraṇādidvandvabhedabhājaśca santo nānyonyaṃ saṃkīryante yathā nānādravyādhivāsitāni ghaṭādyavacchinnāni ghaṭākāśānīti sūpapannamekacinmātraparamārthā api svavicchedādanyonyabhedajuṣaśceti  // 37  //

itthaṃ jīvamaṇḍalagatā avasthāviśeṣā ye te kevalaṃ bhagavatyupacaryante na punastattvataste 'tra kecidityāha

śānte śānta ivāyaṃ hṛṣṭe hṛṣṭo vimohavati mūḍhaḥ  /
tattvagaṇe sati bhagavānna punaḥ paramārthataḥ sa tathā  // 38  //

tattvagaṇe indriyavarge śānte uparate sati tadgataḥ paramātmā śāntaḥ naṣṭa ivābhimanyata evaṃ tasmin hṛṣṭe sāhlāde sati sa tathaivopacaryate paraṃ tamomaye mūḍhe mohavāniti yathā sthāvarayonau na punaḥ paramārthataḥ vastuvṛttena saḥ parameśvaraḥ tathā tena prakāreṇa bhavati  / sarvo hi jaḍabhāga utpādyaḥ saṃhāryo vā bhavati na punarnitye bhagavati bodhasvabhāve māyādikañcukagate vināśotpattī syātāmiti bhagavānsadā samaḥ  // 38  //

samutpattikrameṇāgatā bhrāntiḥ punarjñaptikrameṇa sutarāṃ samunmūlitā bhavatītyatra svātmana eva svātantryamityāha

yadanātmanyapi tadrūpāvabhāsanaṃ tatpurā nirākṛtya  /
ātmanyanātmarūpāṃ bhrāntiṃ vidalayati paramātmā  // 39  //

anātmani acetanalakṣaṇe dehādau kṛśaḥ sthūlaścāsmītyādi yat tadrūpāvabhāsanam anātmanyātmatayā parāmarśanam tat tasmin purā ādāveva nirākṛtya ahaṃ cidānandaikaghano 'navacchinnasvabhāvaḥ svatantraścetyakṛtrimāhantāsphuraṇayā kṛtrimadehādipramātṛtāmapahṛtyātmaiva vigalitadehabandhaḥ paramātmatāṃ yātaḥ san ātmani asminsphuradrūpe viśvapadārthe prakāśavapuṣi svāṅgakalpe 'pi yā bhrāntiḥ dehādipramātṛtābhimānajanitā bhedaprathā tām vidalayati ahamevaiko viśvātmanā sphurāmītyevaṃ cūrṇīkaroti  / idamatra tātparyaṃ yāvadanātmani dehādāvātmābhimāno na galitastāvatsvātmaprathārūpe 'pi jagati bhedaprathāmoho na vilīyate 'taścānātmanyātmābhimānabhramavināśādātmanyanātmābhimānabhrāntiṃ paramātmaiva svātmamaheśvaro bhagavāneva vināśayati nānyasyātra sāmarthyam  // 39  //

evaṃ bhrāntidvayāpasāraṇātparameśvarībhūtasya yogino na kiñcitkāryamavaśiṣyata ityāha

itthaṃ vibhramayugalakasamūlavicchedane kṛtārthasya  /
kartavyāntarakalanā na jātu parayogino bhavati  // 40  //

ittham kārikārthapratipāditaprakāreṇa bhrāntidvayasya prarohavidāraṇe kṛtārthasya svasvātantryasya parijñapteraśeṣasaṃkocavidalanāt kṛtaḥ prāptaḥ arthaḥ puruṣārthalābho yena tasya prakṛṣṭayogayuktasya na kadācit kartavyāntarasya tīrthāṭanakṣetraparigrahadīkṣājapadhyānavyākhyāśravaṇādirūpasya kāryaśeṣasya kalanā manovyāpāro 'pi na vidyate

ayameva paro dharmo yadyogenātmadarśanam  /

iti hi svātmayogasya prādhānyamatastatprāptyā nānyatra pariśramaḥ pūrṇayoginaḥ  / yaduktam

yadā te mohakalilaṃ buddhirvyatitariṣyati  /
tadā gantāsi nirvedaṃ śrotavyasya śrutasya ca  //

iti gītāsu  // 40  //

saṃprati pṛthivyādimāyāntasya bhedāvabhāsabhājo viśvasya bhedābhedamayaśāktabhūmikāveśena pūrṇaprakāśānandaghanaśāṃbhavapadasamāpattyā bhedavilāyanena tadabhedamayatāprāptimabhidhāya tataḥ śāṃbhavātpadātsaṃpūrṇasudhāmbhodhikalpānmahāpravāhadeśīyaśāktaprasarollāsapramukhaṃ tattattaraṅgabhaṅgirūpatāmabhidadhatsvānubhavasiddhaṃ mahāmantravīryasāraṃ samastabhedavilāyanaparamādvayodayaṃ naraśaktiśivasāmarasyātmakaṃ parasaṃviddhṛdayaṃ krameṇonmīlayiṣyanviśvasyāgamikāṇḍatrayātmatāsaṃkalanayuktyaikīkāraṃ tāvadāha

pṛthivī prakṛtirmāyā tritayamidaṃ vedyarūpatāpatitam  /
advaitabhāvanabalādbhavati hi sanmātrapariśeṣam  // 41  //

pārthivaprākṛtamāyīyāṇḍātmakaṃ yatsthūlasūkṣmapararūpaṃ trividham vedyarūpatāpatitam jñānagocaratāṃ prāptam

tattadrūpatayā jñānaṃ bahirantaḥ prakāśate  /
jñānādṛte nārthasattā jñānarūpaṃ tato jagat  //
nahi jñānādṛte bhāvāḥ kenacidviṣayīkṛtāḥ  /
jñānaṃ tadātmatāṃ yātametasmādavasīyate  //

iti śrīkālikākramoktanyāyena yat advaitabhāvanam tadbalāttatprakarṣāt sanmātrapariśeṣam prakāśamānatātmakasattāmātrātmakam bhavati  / hiḥ yasmādarthe  // 41  //

etadeva bhedasyāvāstavatvapratipādanābhiprāyeṇopapādayati

raśanākuṇḍalakaṭakaṃ bhedatyāgena dṛśyate yathā hema  /
tadvadbhedatyāge sanmātraṃ sarvamābhāti  // 42  //

yathā kila sauvarṇaṃ raśanādyābharaṇaṃ suvarṇārthino raśanāviśeṣaparihāreṇa hemamātratayaivābhāti hemarajatakāṃsyatāmranāgādi tāvanmātrārthino loharūpatayā tadvatsarvam idaṃ tyaktahānādānādivikalpakalaṅkasyāvikalpapratibhāsamātraniṣṭhasya yogino

rūpādiṣu pariṇāmāttatsiddhiḥ  /

iti bhaṭṭaśrīkallaṭoktanītyā bhedatyāge sati sanmātram sattāmātrātmakam ābhāti  // 42  //

tadiyataḥ sarvasya saṃkocāvabhāsaparityāgādāgamaprasiddhyā nararūpasya śāktarūpopārohaṃ mantrasaṃpradāyaṃ kaṭākṣayannabhidadhāti

tadbrahma paraṃ śuddhaṃ śāntamabhedātmakaṃ samaṃ sakalam  /
amṛtaṃ satyaṃ śaktau viśrāmyati bhāsvarūpāyām  // 43  //

tat etatsattāmātrātmakaṃ sarvaṃ bṛhattvāt brahma  / yadāhuḥ śrutyantavidaḥ

sadevedaṃ somya agra āsīt  /

iti pūrṇatvāt param heyopādeyābhāvāt śuddham pṛthaktvopaśamāt śāntam ata eva abhedātmakam prakarṣāpakarṣābhāvāt samam

pradeśo 'pi brahmaṇaḥ sārvarūpyamanatikrāntaścāvikalpyaśca  /

iti sthitvā sakalam ata eva amṛtam avināśi

satyāsatyau tu yau bhāgau pratibhāvaṃ vyavasthitau  /
satyaṃ yattatra sā jātirasatyā vyaktayaḥ sthitāḥ  //

iti  /

yadādau ca yadante ca yanmadhye tasya satyatā  /

iti tatrabhavadbhartṛharinirūpitanītyā satyam tadeva sattāmātrātmakametatsarvam bhāsvarūpāyām icchājñānakriyārūpaśaktisāmarasyātmikāyāṃ parasyāṃ śaktau viśrāmyati

saṃvinniṣṭhā viṣayavyavasthitiḥ  /

iti sthityā tanmayībhavatīti  /

atha ca śāntam śakārasyānte yanmūrdhanyarūpaṃ tataḥ paraṃ yat amṛtam amṛtabījātmakaṃ brahma sanmātrātmakaṃ sādākhyapadasparśāt śuddham ata evāhamidaṃ sarvamiti sarvasamarasīkaraṇāt samaṃ sakalam cātaścākhyātigalanāt satyam  / yadādiṣṭaṃ bhagavatā

tṛtīyaṃ brahma suśroṇi  ...  /

iti śrītriṃśikāyām  / tadidamamṛtībhāvāvamṛṣṭaṃ sadāśivapadopārūḍhametadviśvātmakaṃ brahma prāguktāyāṃ śaktau viśrāmyati  // 43  //

kriyājñānecchāmukhena paraśaktau yanna viśrāmyati tanna kiñcidityāha

iṣyata iti vedyata iti saṃpādyata iti ca bhāsvarūpeṇa  /
aparāmṛṣṭaṃ yadapi tu nabhaḥprasūnatvamabhyeti  // 44  //

yadvastu vastuvṛttena bahirvidyamānapi tadyadīcchājñānakriyāmukhena bhāsvareṇaitacchaktitrayasāmarasyātmakaparāśaktisphāramayena bodhena na parāmṛṣṭam tatprakhyopākhyāvikalaṃ gaganapuṣpatulyam  / anena ca sadvṛttyūrdhvavartitriśūlātmakavṛttivīryaṃ sūcitam  // 44  //

etacchāktapadāveśamasyānuvadañśāṃbhavapadasamāpattyā tanmayībhāvamāvirbhāvayati

śaktitriśūlaparigamayogena samastamapi parameśe  /
śivanāmani paramārthe visṛjyate devadevena  // 45  //

tadittham samastamapi etatsattāmātrarūpatvātproktabrahmaparamārtham śaktitriśūlaparigamayogena nirṇītayuktyā parāśaktisamāpattikrameṇa śivanāmani paramārthe anavacchinnacidānandaikaghane parameśvare svasminsvabhāve visṛjyate antarmukhavimarśanaprakarṣāttatsamāveśena tanmayībhāvamāpādyate devānām brahmādisadāśivāntānāṃ sarvaprakāśānāmindriyāṇāṃ ca devena prabhuṇā paramaśivenaiva nahyatrānyasya kasyacitkartṛtvaṃ ghaṭate nāpyetadvyatirikto 'nyaḥ kaścitpramātāstyasyaiva ca bhagavatastattadbhūmikādhirohiṇastattadrudrakṣetrajñādipramātṛrūpatayā sphuraṇamiti devadevena iti yuktaivoktiḥ  / taditthaṃ visargavṛttirdarśitā  // 45  //

evamiyatā bhedātmano nararūpasya jagato bhedābhedātmakaśāktapadādhyāroheṇābhinnacidghanaśivasāmarasyāpattimupasaṃhāradṛśā pradarśyedānīṃ cidekaghanaḥ śiva eva śaktirūpatayollāsya narātmakaviśvarūpatayā sphurati na tu śivavyatiriktaṃ śaktinarayoḥ kimapi rūpaṃ śiva eva tvitthaṃ nijarasāśyānatayā sphuratīti prasarayuktiṃ mahāmantrasphāramayīṃ darśayati

punarapi ca pañcaśaktiprasaraṇakrameṇa bahirapi tat  /
aṇḍatrayaṃ vicitraṃ sṛṣṭaṃ bahirātmalābhena  // 46  //

cidānandecchājñānakriyākhyaśaktipañcakasāmarasyātmā yaḥ paramaśivastena cidādiśaktiprādhānyaprathanātmakaśivaśaktisadāśiveśvaraśuddhavidyākhyabhūmikonmīlanayuktyā tadaṇḍatrayaṃ vicitram iti tattadbhuvanādirūpam sṛṣṭaṃ bahirātmalābhena iti bāhyābhāsātmatayā svātmanaḥ pradarśanena  / punarapi ityanenaitaddarśayati yatparamaśiva eva svatantraḥ sadā svabhittau viśvaprapañcollāsanavilāpanakrīḍāṃ svātmānatiriktāmapyatiriktāmivādarśayannevamitthaṃ sthito na tu tadvyatiriktaṃ kimapyastīti  // 46  //

itthaṃ viśvollāsanavilāyanakrīḍāśīlo bhagavānyaḥ śivaḥ iti vyapadiśyate sa katamaḥ kutra tiṣṭhati katamena vā pramāṇena prasiddha ityāśaṅkya sarveṣāṃ svātmaiva śivaḥ sarvatrādisiddhatayā sphuransargādīnvidadhātītyasmacchabdavācakaiḥ śabdaiḥ pratipādayati

iti śakticakrayantraṃ krīḍāyogena vāhayandevaḥ  /
ahameva śuddharūpaḥ śaktimahācakranāyakapadasthaḥ  // 47  //

mayyeva bhāti viśvaṃ darpaṇa iva nirmale ghaṭādīni  /
mattaḥ prasarati sarvaṃ svapnavicitratvamiva suptāt  // 48  //

ahameva viśvarūpaḥ karacaraṇādisvabhāva iva dehaḥ  /
sarvasminnahameva sphurāmi bhāveṣu bhāsvarūpamiva  // 49  //

draṣṭā śrotā ghrātā dehendriyavarjito 'pyakartāpi  /
siddhāntāgamatarkāṃścitrānahameva racayāmi  // 50  //

iti vyākhyātena prakāreṇa cidādiśaktipañcakākṣipto 'nanto yaḥ śaktisamūhastadeva yantraṃ krīḍāyogena vāhayan araghaṭṭaghaṭīyantranyāyena sṛṣṭyādyunmajjananimajjanahelākrameṇa viparivartayannahameveti devaḥ sarvaprāṇinām aham ityanāhato nādātmā parāhantācamatkārasāraḥ svātmaparāmarśaḥ sa eva sarvasyānapahnavanīyo 'yaṃ svātmaiva devaḥ krīḍanaśīlaḥ sphuratīti  / anena svasvarūpaniṣṭha eva śiva iti pratipāditam  / tathā śuddharūpa iti kalpanātikrāntagocaraḥ  / anyacca śaktīnāṃ karaṇadevatānāṃ viṣayāharaṇatyāgādivyavahārasvātantryadātṛ yat mahācakranāyakapadam tatra tiṣṭhati tatsthaḥ  / yataḥ karaṇaśaktīnāṃ caitanyaviśrāntiṃ vinā svarūpasattā na vidyate 'taḥ śaktimantameva svarūpāsādanāyānavarataṃ bhajanta ityanena sarvapramātṛhṛdayādhiṣṭhātṛtvādbhagavato niyatabhuvanādhiṣṭhātṛtvaṃ parihṛtam  / tathā yadidaṃ kiñcidviśvatayābhimataṃ tatsarvamādarśapratibimbanyāyena mayyeva bhāti vyākhyātarūpāsmadarthaviśrāntamevāvabhāsate 'hantāsārameva sphuratīti yāvat  / tathā mattaḥ iti pūrṇādahamiti rūpātsvātmanaḥ sakalaṃ niḥśeṣamidaṃ viśvam prasarati pramātrapekṣayāpahṛtatayā sphurati  / kathamityāha svapnavicitratvamiva suptāt iti yathā nidritātpramātuḥ svapnāvasthāyāṃ bāhyapadārthābhāve 'pi puraprākāradevagṛhādi nānāścaryaṃ svapnapadārthavaicitryamavidyādiparikalpitakāraṇāntarābhāvātsvasaṃvidupādānameva prasarati tathaiva tīrthāntaraniyamitakāraṇāntarānupapatteranavacchinnacidānandaikaghanādahamiti rūpādviśvamiti  / tathā ahameva viśvarūpaḥ ityādi  / ahamityeva yaḥ pūrṇaścaitanyaparāmarśa eṣa evāsmi nānādehādipramātṛtāsamāpanno viśvarūpa āgopālabālāṅganādiṣvantarabhedena sphuraṇādviśvāni mamaiva rūpāṇīti yāvat  / ka iva karacaraṇādisvabhāvo deha iva yathā sāmānyena sarveṣāmeko dehaḥ karacaraṇādisvabhāvaḥ pratipramātṛ svālakṣaṇyena nānārūpastathaivaikaścaitanyalakṣaṇaḥ padārthaḥ sarvāvāsatvādviśvarūpa
iti  / tathā sarvasmin pramātṛpramāṇaprameyarūpe 'smin ahameva sphurāmi sarvasya svātmānubhavitṛtvena prakāśanāt  / kathaṃ bhāveṣu bhāsvarūpamiva iti yathā nānāvastuṣu bhāsvarūpam atiśayena dyotanaśīlaṃ vastu dedīpyate tathaiva jaḍe 'smiñjagatyekaścidrūpo 'hamiti  / ataśca draṣṭā ityādi dehendriyavarjito 'pi cinmūrtatvādahameva paśyāmi śṛṇomi jighrāmi rasayāmi spṛśāmīti sarvatra pūrṇāhantāviśrānteḥ kṛtakṛtyatā  / dehendriyavargo hi paśyāmītyādi manyate paraṃ svāpādyavasthāsu draṣṭṛtvādyabhāvāttasmāddehendriyādivargasamullāsakastadvarjito 'pi cidānandaikaghanaḥ sarvabhūtahṛdayāntaracārī viṣayopabhogabhoktāsmacchabdavācyaḥ paraḥ puruṣa eva  / tathā ca śrutiḥ

apāṇipādo javano grahītā paśyatyacakṣuḥ sa śṛṇotyakarṇaḥ  /
sa vetti vedyaṃ na ca tasyāsti vettā tamāhuragryaṃ puruṣaṃ mahāntam  //

iti  / tathākartāpi siddhāntāgama ityādi svayamavidhātāpi devamunimanuṣyādyāśayāviṣṭaḥ saṃkṣepavistāravivakṣayāntaḥpratibhāsvarūpo 'haṃ siddhāntādīnnānāścaryānkaromi na punarjaḍānāṃ loṣṭasthānīyānāṃ dehendriyāṇāṃ tatkaraṇaṃ śakyamiti tattadvyavadhānenāhameva sarvapramāṇanirmāteti  / anena parāhantāsvarūpasya svātmamaheśvarasya sattāyāṃ na pramāṇopayoga upapadyata upayujyate vetyuktaṃ syāt  / evamanapahnavanīyaḥ aham ityevamanubhavitṛtayā sarveṣāṃ svātmaiva śivaḥ sarvatrāvasthitaḥ sarvapramāṇeṣvādisiddha iti  // 47  // 48  // 49  // 50  //

tadevaṃ vyākhyātena krameṇa sarvo mamāyaṃ vibhava iti dārḍhyena svātmānaṃ pratyavamṛśanparabrahmasvarūpo yogī bhavatītyāha

itthaṃ dvaitavikalpe galite pravilaṅghya mohanīṃ māyām  /
salile salilaṃ kṣīre kṣīramiva brahmaṇi layī syāt  // 51  //

anena prakāreṇa sarvāhaṃbhāvapariśīlanayuktyā dvaitavikalpe galite bhedaprathāyāṃ vilīnāyāṃ mohanīṃ māyāṃ pravilaṅghya anātmanyātmābhimānarūpāmakhyātiṃ bhedaprathāhetumahameva viśvātmeti saṃkocāpasaraṇena samutsṛjya jñānī brahmaṇi bṛṃhaṇātmake pūrṇe cidānandaikaghane svarūpe layī syāt saṃkocavilayādbrahmatādātmyaṃ yāyāt  / kiṃ yathetyāha salile ityādi  / yathā salilam uddhṛtaṃ nānāghaṭādibhirjalaṃ kṣīraṃ vā vividhaśāvaleyabāhuleyādyanekagosahasrasaṃbhinnaṃ punarapi ghaṭaśāvaleyādikṛtabhedasaṃkocaparikṣayāt salile salilaṃ praviṣṭam kṣīre kṣīram vetyekameva tadvastu na tatra bhedaḥ sphurati tathaiva dehaprāṇapuryaṣṭakaśūnyātmakapratyavamarśabhaṅgādbrahmaiva saṃpadyate  / yathāha bhaṭṭadivākaravatsaḥ

jāte dehapratyayadvīpabhaṅge prāptaikadhye nirmale bodhasindhau  /
avyāvartya tvindriyagrāmamantarviśvātmā tvaṃ nitya eko 'vabhāsi  //

iti kakṣyāstotre  // 51  //

evaṃ brahmasattāmadhirūḍhasya yogino dvandvābhibhavo 'pi brahmamaya eva na svarūpavipralopāya pragalbhata ityāha

itthaṃ tattvasamūhe bhāvanayā śivamayatvamabhiyāte  /
kaḥ śokaḥ ko mohaḥ sarvaṃ brahmāvalokayataḥ  // 52  //

evaṃ nirṇītena prakāreṇa galitakañcukabandhasya yoginaḥ tattvasamūhe bhūtaviṣayendriyavrāte bhāvanayā sarvamidamekā svasaṃviditi dṛḍhapratipattyā śivamayatvam prāpte paramādvayarūpatāṃ yāte śokamohopalakṣitā dvandvābhibhavāḥ sarvam idaṃ tattvavrātam brahma paśyato 'sya na kecanaiva te brahmamayatvātsarve svarūparūpā iti na khedāya prabhavanti  // 52  //

nanu paramādvayarūpasyāpi jñānino 'vaśyaṃ sthite śarīre 'pi taddhetukaśubhāśubhakarmaphalasaṃcayaḥ kimiti na syāditi pariharati

karmaphalaṃ śubhamaśubhaṃ mithyājñānena saṃgamādeva  /
viṣamo hi saṅgadoṣastaskarayogo 'pyataskarasyeva  // 53  //

aśvamedhabrahmahananādirūpapuṇyāpuṇyakarmaphalapracayasaṃcayo 'pi mithyājñānena saṃgamādeva prādurbhavatyahaṃ śarīrīdamaśvamedhādi mamopāyatayāstviti yadanātmanyātmābhimānalakṣaṇaṃ vaiparītyena jñānaṃ tena yo 'bhiṣvaṅgastasmādeva paśoḥ śubhāśubhakarmaphalasaṃcayo yenānavaratādhivāsitaḥ saṃsārakleśabhājanaṃ bhavati  / nanu brahmātmakasyāpi pramātuḥ kimityetāvatā paśutvamāyātītyatrārthāntaramupakṣipati viṣamo hi ityādi  / yasmāt saṅgadoṣaḥ sarvathāviṣahyo yathāsādhuyogo 'tyantasādhorapi svagatadoṣasamarpaṇaṃ kurute tathaiva śuddhasyāpi pramāturakhyātijanito mohayogaḥ paśutvamāpādya śubhāśubhakarmasaṃbandhaṃ dadāti  // 53  //

janmamaraṇādyapi na brahmarūpasya yogino 'pi tu māyāpramātṝṇāmevetyāha

lokavyavahārakṛtāṃ ya ihāvidyāmupāsate mūḍhāḥ  /
te yānti janmamṛtyū dharmādharmārgalābaddhāḥ  // 54  //

ye pramātāro dehātmamānino bhūtvā phalakāmanākaluṣitā lokācārarūpāṃ puṇyāpuṇyamayīm avidyām bhedaprathārūpāṃ māyāṃ jagati svarganarakādiphalaprāptyupāyatvena sevante te mūḍhāḥ ajñāḥ puṇyāpuṇyanigaḍabaddhāstatphalopabhogāya punaḥ punarjāyante mriyante cetyanavaratasaṃsārakleśabhājo bhavanti na punaḥ prakṣīṇamohāvaraṇo vigalitadharmādharmabandho brahmasvabhāvo yogī jāyate mriyate veti  // 54  //

evamavidyopārjitānyapi karmāṇi jñānāvirbhāvādeva kṣīyante nānyathetyāha

ajñānakālanicitaṃ dharmādharmātmakaṃ tu karmāpi  /
cirasaṃcitamiva tūlaṃ naśyati vijñānadīptivaśāt  // 55  //
ajñānakāle kṛtrimapramātṛtābhimānāvasare puṇyāpuṇyarūpam karma anuguṇaphalaprārthanayā yat nicitam svīkṛtaṃ tat vijñānadīptivaśāt viśiṣṭajñānadīptivaśāt naśyati ahameva paraṃ brahmeti kṛtrimapramātṛtādāhasamarthaṃ vijñānaṃ tasya yā paunaḥpunyena pratyavamarśanātprabhā tatsāmarthyāttadadarśanaṃ yāti  / kimivetyāha cirasaṃcitaṃ tūlamiva yathā cirasaṃcitaṃ tūlaṃ haṃsaroma vahnipradīptivaśājjhaṭityeva bhasmasādyāti tathaiva sarvaḥ karmaphalapracayo vijñānavahnisāmarthyātkṣaṇamadhye pralayamupagacchatīti  / gītāsu

yathaidhāṃsi samiddho 'gnirbhasmasātkurute 'rjuna  /
jñānāgniḥ sarvakarmāṇi bhasmasātkurute tathā  //

iti  // 55  //

na kevalaṃ prākkṛtaṃ karma jñānaprasādādupalīyate yāvadidānīntanamapi karma jñāneddhayā dṛṣṭyā na phalopabhogāya paryavasyatītyāha

jñānaprāptau kṛtamapi na phalāya tato 'sya janma katham  /
gatajanmabandhayogo bhāti śivārkaḥ svadīdhitibhiḥ  // 56  //

ātmamaheśvarapratyavamarśaprarūḍhau kṛtamapi śubhāśubhādikam karma kṛtrimapramātṛtābhimānābhāvānnānuguṇaphaladānāya pragalbhata iti karmaphalābhāvāttadupabhogayogyasya janmanaḥ kena prakāreṇa sattā syānna bhavedyoginaḥ punarjanmetyarthaḥ  / nanu sa piṇḍapātātpunarna jāyate cettarhi kīdṛśaḥ syādityāha gatajanma ityādi  / gato janmarūpasya bandhasya yogaḥ saṃbandho yasya sa evamiti prakṣīṇamohāvaraṇaḥ svadīdhitibhiḥ cinmarīcinicayaiḥ saḥ śivarūpo 'rko bhāti sphurati na punastīrthāntaraparikalpito 'sya mokṣaḥ kutracitprāptiriti kevalaṃ māyādikañcukakṛtasaṃkocavināśātsvaśaktivikasvaratāmāpadyata iti  // 56  //

atraiva yuktimāha

tuṣakambukakiṃśārukamuktaṃ bījaṃ yathāṅkuraṃ kurute  /
naiva tathāṇavamāyākarmavimukto bhavāṅkuraṃ hyātmā  // 57  //

yathā kiṃśārukatuṣakambukebhyaḥ pṛthakkṛtaṃ śālibījaṃ bījasvabhāvakiṃśārukādisāmagryabhāvātkṣitijalātapamadhyavartyapi naiva aṅkurajananalakṣaṇakārye heturbhavati tathaiva kambukasthānīyena āṇavena malena tuṣasthānīyena māyāmalena kiṃśārukasthānīyena kārmamalena ca muktaḥ pṛthagbhūtaḥ ātmā caitanyaṃ malatrayarūpasāmagryabhāvānna punaḥ bhavāṅkuram saṃsāraprarohaṃ vidadhāti kevalaṃ viśvagatanānāpadārthasārthaprādurbhāvavināśavaicitryaṃ svātmani parāmṛśanmaheśvara eva bhavati  // 57  //

evaṃ jñānāgnidagdhakañcukabījasya jñānino na kiñcicchaṅkāsthānaṃ heyopādeyaṃ vetyata āha

ātmajño na kutaścana bibheti sarvaṃ hi tasya nijarūpam  /
naiva ca śocati yasmātparamārthe nāśitā nāsti  // 58  //

yaḥ ātmajñaḥ svātmamaheśvarasvātantryavitsaḥ na kutaścana bibheti na sa kasmādapi rājñaḥ śatroḥ prāṇibhyo vā bhayamādatte  / kuta etadityāha sarvaṃ hi tasya nijarūpam iti yataḥ tasya svātmamaheśvarādvayavedinaḥ sarvam padārthajātamidaṃ viśvam nijasya svātmano mahāprakāśaikavapuṣa eva rūpam ākāraḥ sarvatra prakāśānugamāditi prakāśa eva svātantryātsvaparātmanā prakāśate 'ta eva bhayasthānaṃ loke yatkiñcitpratibhāti tattasya tathaiva svāṅgakalpameva kathaṃ bhayajanakaṃ syādyaduta svātmano vyatiriktaḥ padārtho bhayaheturbhavetkaḥ punaḥ sarvataḥ paripūrṇasyāvadhibhūto bhinno yamādirasti yasmājjñānyapahastitadehātmamānitvo 'pi bibhiyāditi sarvatra nijarūpopalabdheḥ saṃsārasthito 'pyekako vigalitasvaparavibhāgatayā niḥśaṅkaṃ vicaratyeva  / yathoktaṃ parameṣṭhipādaiḥ

yo 'vikalpamidamarthamaṇḍalaṃ paśyatīśa nikhilaṃ bhavadvapuḥ  /
svātmamātraparipūrite jagatyasya nityasukhinaḥ kuto bhayam  //

iti  / granthakāro 'pi

ekako 'hamiti saṃsṛtau janastrāsasāhasarasena khidyate  /
ekako 'hamiti ko 'paro 'sti me itthamasmi gatabhīrvyavasthitaḥ  //

iti  / anyacca naiva ca śocati ityādi  / nāpyātmajñaḥ śocati yathā dhanadārādikaṃ mama naṣṭaṃ rikto 'smi vyādhinākrānto 'haṃ mriye vetyādi yato vyākhyātena krameṇa paramārthe tāttvike vastuni caitanyarūpe 'ntarmukhe pramātṛmātre nāśitā kṣayadharmitvaṃ na vidyate  / sarvaṃ hyabhimānasāraṃ kāryatvena pratibhāsamānamidantāvācchinnamutpadyate kṣīyate ca na punaḥ saṃvinmayasyātmano 'hantāsārasyākṛtrimasya svatantrasya kāryonmukhaprayatnānupalabdheḥ  / na caitāvatā svarūpavipralopaḥ syāditi vimṛśato yogino dehasthasyāpi taddhetukaḥ śokādyāvirbhāvaḥ svarūpācchādakatvena na bhavediti  // 58  //
nāpi svātmamaheśvarasvarūpapariśīlanadārḍhyādasya jñāninaścetasyapūrṇatvādidoṣaḥ syāditi pratipādayati

atigūḍhahṛdayagañjaprarūḍhaparamārtharatnasaṃcayataḥ  /
ahameveti maheśvarabhāve kā durgatiḥ kasya  // 59  //

atigūḍham atiśayena guptam hṛdayameva gañjam sarvaparamārthasvasvarūpaviśrāntisthānasvabhāvaṃ bhāṇḍāgāraṃ tatra yo 'titīvratamasamāśvāsaprarūḍhaḥ paramārthaḥ sadgurūpadiṣṭaḥ svātmajñānasatattvaḥ sa eva sarvavibhūtihetutvādratnasaṃcaya iva ratnasaṃcayaḥ tena hetunā ahameveti sarvamidamasmīti ya āvirbhūtaḥ pūrṇaḥ parāhantāviśrāntilakṣaṇaḥ maheśvarabhāvaḥ śarīriṇo 'pi svātmaprakāśasvātantryaṃ tasminsthite sati kā nāma varākī durgatiḥ daridrabhāvastadupalakṣito vā kaścitkṛtrimo vibhūtyādyatiśayaḥ syāt  / ābhāsasārā hi sarve padārthā yadaivābhāsante tadaiva yoginaḥ svātmakalpāḥ santaḥ kathamutkarṣāpakarṣādau pragalbhanta iti na kiñciddaurgatyādikaṃ bhavet  / kasya veti ko vāsyā durgateḥ samāśrayo dehādyātmābhimānino hyasyā durgateḥ samāśrayā bhavantu yataste vyatiriktasyaiṣaṇīyasya prāptyeśvarāstadapahārādriktā iti  / yaḥ punarakṛtrimāhantāpratyavamarśaparamārtho jñānī sarvamasmītyavyatiriktenaiṣaṇīyena maheśvaraḥ sa kathaṃ vyatiriktaprāptyaprāptyabhāvāddaurgatyāderbhājanaṃ syādata eva gañjaśabdasya ratnasaṃcayaśabdasyeśvaraśabdasya ca hṛdayaprarūḍhaparamārtho mahānityakṛtrimārthavācakāni viśeṣaṇānyupapāditāni  // 59  //

idānīṃ mokṣasvarūpamāha

mokṣasya naiva kiñciddhāmāsti na cāpi gamanamanyatra  /
ajñānagranthibhidā svaśaktyabhivyaktatā mokṣaḥ  // 60  //

mokṣasya parāhantācamatkārasārasya kaivalyasya dhāma vyatiriktaṃ sthānaṃ na vidyata eva deśakālākārāvacchedābhāvādata eva na cāpyanyatra kutracidvyatirikte gamanam layo mokṣo yathā bhedavādināṃ matenotkrāntyā cakrādhārādibhedanādūrdhvaṃ dvādaśānte laya eṣaiva muktiriti  / yaduktam

vyāpinyāṃ śivasattāyāmutkrāntyā kiṃ prayojanam  /
avyāpini pare tattve hyutkrāntyā kiṃ prayojanam  //

iti  / evaṃvidhāpyanye tīrthāntaraparikalpitā bahavo mokṣabhedāḥ santi te pratanyamānā granthagauravabhayamānayantīti neha pratanyanta iti sarvatra dvaitamalasya saṃbhavādamokṣe mokṣalipsā mokṣābhāsa eva  / kiṃ punarmokṣalakṣaṇamityāha ajñāna ityādi  / ajñānam akhyātijanita ātmanyanātmābhimānapūrvo 'nātmani dehādāvātmābhimānalakṣaṇo mohaḥ sa eva pūrṇasvarūpasaṃkocadāyitvādgranthiriva granthiḥ svasvātantryalakṣaṇasya nijasya vyāpitvāderdehādyabhimānatayā bandhastasya bhit bhedanaṃ nijapūrṇasvātmasvātantryapariśīlanadārḍhyāddehādyabhimānalakṣaṇasya granthervidāraṇaṃ tena hetunā svaśaktibhiḥ svātmasvātantryalakṣaṇairdharmaiḥ abhivyaktatā svātmaśaktivikasvarataiṣa eva niratiśayaḥ mokṣaḥ iti  / ayamāśayo yathā sahajanityavyāpakatvādidharmayuktamākāśamapi ghaṭādibhittibandhasaṃkucitaṃ tata eva tadevāvyāpakatvādidharmayuktaṃ ghaṭākāśamityucyata ākāśādbhinnamiva prathate punarapi ghaṭādibhittikṛtasaṃkocabhaṅgāttadeva ghaṭādyākāśaṃ tadaiva vyāpakatvādidharmayuktaṃ syānna punastasya ghaṭādibhaṅgānnūtanaḥ kaściddharmāvirbhāva āyātīti  / tathaiva dehādyabhimānakṛtasaṃkocasaṃkucitaṃ caitanyaṃ baddhamivetyucyate tadeva punaḥ svasvarūpajñānābhivyakterdehādipramātṛtābandhasaṃkṣayātsvaśaktivivekavikasvaraṃ muktamivetyabhimānamātrasārau parimitapramātrapekṣayā bandhamaukṣau na punaḥ paramārthe saṃvittattva evaṃ kiñcitsaṃbhavatīti  / tasmānmuktau nūtanaṃ na kiñcitsādhyate nijameva svarūpaṃ prathate  / etadeva viṣṇudharmeṣvapyuktam

yathodapānakaraṇātkriyate na jalāmbaram  /
sadeva nīyate vyaktimasataḥ saṃbhavaḥ kutaḥ  //
bhinne dṛtau yathā vāyurnaivānyaḥ saha vāyunā  /
kṣīṇapuṇyāghabandhastu tathātmā brahmaṇā saha  //

iti  // 60  //

evaṃ prakṣīṇājñānabandho jñānī parānugrahārthaṃ śarīramapi dhārayanmukta ityāvedayati

bhinnājñānagranthirgatasaṃdehaḥ parākṛtabhrāntiḥ  /
prakṣīṇapuṇyapāpo vigrahayoge 'pyasau muktaḥ  // 61  //

śarīrasaṃbandhe 'pi svātmajñānaviccharīrādyabhimānābhāvājjīvannapi muktaḥ vikasvaraśaktirbhavet  / nanu vigrahayoga eva bandhaḥ kathaṃ tatsaṃbandhe 'pyasau muktaḥ syādityāha bhinna ityādi  / bhinno vidārito 'jñānarūpo granthirapūrṇatvakhyātisamuttho dehādyabhimānarūpo bandho yena sa evam  / tathā gatasaṃdehaḥ ityata eva naṣṭasaṃśayaḥ  / parākṛtā nyakkṛtā paramādvayajñānalābhāt bhrāntiḥ dvayarūpo bhramo yena sa tatheti  / evaṃ pariśīlanena prakṣīṇāni puṇyāpuṇyāni vigalitasaṃskārāṇi dehātmamānitvābhāvāddharmādharmāṇi yasya sa evaṃvidha iti  / anena ajñānameva bandhaḥ iti pratipāditam  / tacca vigrahayoge 'pi yasya prakṣīṇaṃ sa tadaiva jīvanneva mukto na punaḥ śarīrayogo bandhastadapagamo muktiriti kiṃ tu dehapātātpūrṇo mokṣa iti  // 61  //
jīvanmuktasya karmahetau śarīre sthite 'pi śarīrayātrāmātrārthaṃ jñāneddhaṃ kurvāṇasya karma na phalāya tasya bhavatītyatropapattimāha

agnyabhidagdhaṃ bījaṃ yathā prarohāsamarthatāmeti  /
jñānāgnidagdhamevaṃ karma na janmapradaṃ bhavati  // 62  //

vahninirbhṛṣṭaṃ śālibījaṃ kṣitisalilātapamadhyavartyapi sāmagrīvaikalyādyathāṅkurādijanane 'śaktatāṃ yāti tathaiva jñānāgninā dagdham paramādvayabodhadīptyā pluṣṭam karma yathāhamevetthaṃ viśvātmanā sphurāmītyevaṃrūpeṇa dehādyātmamānitvahānerheyopādeyabuddhiparityāgena yatkiñcicchubhāśubhaṃ karma kriyamāṇaṃ taddagdhavīryaṃ na punarjñāninaḥ piṇḍapātādanantaraṃ janmaphalapradam bhavati dehanirmāṇahetuḥ saṃpadyate dagdhaṃ bījamivāṅkure  / tasmānna sarvāhaṃbhāvarūpāyāścitiśakteraphalābhisaṃdhānatayā kṛtaṃ karma bhūyo janma dātuṃ prabhavatīti  // 62  //

evaṃ punarvikasvarāpi citiśaktiḥ kathaṅkāraṃ dehavatī syādityāha

parimitabuddhitvena hi karmocitabhāvidehabhāvanayā  /
saṅkucitā citiretaddehadhvaṃse tathā bhāti  // 63  //

yasmāt parimitabuddhitvena akhyātijanitena dehādyabhimānavāsanāpūrvakakāmanākāluṣyaniścayena yatkṛtam karma yathāhamaśvamedhena yakṣya ihāmutra ca sukhī bhūyāsaṃ mā kadācana duḥkhyahaṃ bhūyāsamamunā karmaṇā vaindraṃ padaṃ prāpnuyāmityevaṃ vāsanāviśiṣṭasya karturevamanuguṇaṃ karma tasya manovāsanālabdhaprarūḍheḥ karmaṇaḥ ucitaḥ tadanuguṇaphalabhoktṛtāyogyo 'sau bhāvī dehaḥ prārabdhakarmaphalabhoktṛśarīrādhikāraparikṣayādyaduttaratra bhaviṣyaccharīraṃ tasya yā bhāvanā amunāśvamedhādikarmaṇā sāmrājyādyāpnuyāmityabhimatakarmaphalavāsanādhirūḍhistayā karmocitabhāvidehabhāvanayeyaṃ sarvataḥ pūrṇāpi citiśaktirāṇavamāyīyamalamūlena kārmamalenāghrātā saṃkucitā vyāpinyapi ghaṭākāśavatkarmānuguṇaphalabhoktṛśarīravāsanāvacchedavatī saṃpannā satī etaddehadhvaṃse tathā bhavati iti  / etasya prārabdhasya karmaphalasya yo bhoktā dehaḥ tasya bhogaparikṣayādyaḥ dhvaṃsaḥ mṛtistasmindehadhvaṃse sati sā citiḥ udbhūtakarmavāsanā tathā bhavati yenāśayena pūrvakarmaphalamupārjitaṃ tatkarmaphalabhoktā yo dehastadvatī saṃpadyate yadvaśāccitirapi svarganarakādibhogabhājanaṃ syāt  / tasmāccharīrībhūtvā parimitaphalalaulyādyatkṛtaṃ karma tatphalabhoktṛ janma dātumavaśyaṃ prabhavati  / yatpunaraśarīrībhūtvā sarvaṃ brahmāsmīti saṃvidrūpatayā kṛtaṃ tadvāsanāprarohānāsādanātkathaṃ vyāpinyāścitiśakterjanmane syāditi tātparyārthaḥ  // 63  //


evamanātmatayā samucitaṃ karma saṃsaraṇāya pramāturbhavatīti cettarhyātmasvarūpaṃ vaktavyaṃ yena saṃsārī na syāditi pratipāditamapi śiṣyajanahṛdayaṅgamīkartuṃ punaḥ kathayati

yadi punaramalaṃ bodhaṃ sarvasamuttīrṇaboddhṛkartṛmayam  /
vitatamanastamitoditabhārūpaṃ satyasaṃkalpam  // 64  //

dikkālakalanavikalaṃ dhruvamavyayamīśvaraṃ suparipūrṇam  /
bahutaraśaktivrātapralayodayaviracanaikakartāram  // 65  //

sṛṣṭyādividhisuvedhasamātmānaṃ śivamayaṃ vibudhyeta  /
kathamiva saṃsārī syādvitatasya kutaḥ kva vā saraṇam  // 66  //

yadi punaḥ paraśaktipātaviddhahṛdayaḥ pramātā dehādipramātṛtābhimānamadhaspadīkṛtya svātmānaṃ śivamayaṃ vibudhyeta cidānandaikaghanaṃ vijānīyātsa parijñātasvātmamaheśvarabhāvaḥ kathamiva kena prakāreṇa saṃsārī saṃsaraṇaśīlo bhavenna syāditi yāvadyataścidacidrūpapuryaṣṭakātmā kārmamalasaṃbandhena saṃsarati yaḥ punaścidekamūrtiḥ śivamayaḥ prakṣīṇāṇavādimalakañcukaḥ sa kathaṃ saṃsārīti tātparyam  / nanu cidekamūrtiḥ syātsaṃsārī ca bhavediti kiṃ duṣyedityevamāśaṅkyāha vitata ityādi  / vitatasya anavacchinnadeśakālākārasya pramāturdehādyabhimānapūrvasvakṛtavāsanāparikṣayātpūrṇasya tasya kutaḥ saraṇam sarvavyāpitvāttadatiriktaṃ kimasti yadvastvapekṣya tato viśliṣṭo 'nyatra bhinne saṃsaraṇaṃ gamanaṃ kuryādyato dehādipramātṛtābhimānāvacchinnasya kilāpādānādhikaraṇādikārakasaṃbhavo yaḥ punaścidekaghano brahmabhūto 'navacchinnadeśakālaḥ pramātā tasya saṃsaraṇe vācoyuktirapi na bhavediti  / kīdṛśaṃ śivarūpamātmānaṃ vibudhyetetyāha amalaṃ bodham ityādi  / apagata āṇavādimalapracayo yasya tamata eva vaimalyāt bodham śuddhacaitanyam  / tathā sarvasamuttīrṇam niratiśayaṃ jñānakriyāsvātantryaṃ prakṛtaṃ yasyeti tam vitatam deśādikṛtavicchedābhāvādvyāpinam  / tathāvidyamāne 'stodite pralayodayau yasyāḥ bhāsaḥ bodhadīpteḥ saiva rūpam deho yasya tam  / anyacca satyāḥ paramārthāḥ saṃkalpāḥ svecchāvihārā yasya yadyadicchati tattathaiva bhavatīti tamevaṃvidham  / tathā dikkālākārakalanābhiścarcābhirvirahitaṃ vyāpitvanityatvadharmayogādata eva dhruvam kūṭastham avyayam avināśinam  / tathā īśvaram svatantram  / anyacca tathā suparipūrṇam suṣṭhu nirākāṅkṣam  / tadanu bahutarāṇi prabhūtāni śabdarāśisamutthāni brāhmyādiśaktyadhiṣṭhitāni ghaṭapaṭādiśaktivrātāni teṣāṃ layotpattividhau svatantram  / anyacca sṛṣṭyādividhisuvedhasam supravīṇam vedhasam vidhātāramityevamādiviśeṣaṇaiḥ
sarvataḥ paripūrṇaṃ svātmamaheśvaraṃ jānāno yatkiñcidapi kurvāṇo dagdhakarmabījo na punaḥ saṃsārabhāgjīvanneva vimukto bhavediti yāvat  // 64  // 65  // 66  //

evaṃ svātmapratyavamarśopapattyā jñāninā vigalitakarmaphalābhilāṣeṇa kṛtamapi karma na phalāyetyāvedayansvānubhavasiddhaṃ lokadṛṣṭāntamāha

iti yuktibhirapi siddhaṃ yatkarma jñānino na saphalaṃ tat  /
na mamedamapi tu tasyeti dārḍhyato nahi phalaṃ loke  // 67  //

ahameva cidghanaḥ svatantraḥ sarvapramātrantaratamatvena sarvakarmakārī nāhaṃ vā kartā pārameśvarī svātantryaśaktiritthaṃ karotīti mama śuddhacaitanyarūpasyaitāvatā kimāyātam iti yuktibhiḥ prākpratipāditasvarūpābhirupapattibhirvyākhyātasvātmasvarūpavidaḥ pramāturubhayathā dehādyahaṃbhāvābhāvāddheyopādeyaśūnyatvena yatsiddhaṃ karma niṣpannamapi kṛtam na saphalam na tatphalena yujyate tasyātmajñāninaḥ pratipāditavadubhayathā kṛtrimatvābhāvātkṛtamapi karma kutra phalena yogaṃ kuryāddehādipramātṛtābhimānasvabhāvāśrayābhāvānna kutraciditi yāvat  / kṛtasya karmaṇo yā pramātuḥ phalābhimānarūḍhireṣa evāśrayo jñāninastvabhimānābhāvātsvasminrūpa eva prakṣīṇaṃ karma na phalena saṃbadhyata iti  / nanvabhimānādeva karma phalena yujyata iti kutra yathetyāha na mamedamapi ityādi  / dṛṣṭaṃ caitannāpūrvaṃ yathā na mamedam yajñādikaṃ karma api tu tasya kasyāpyarthavato yajamānasya iti anayā buddhyā kṛtamapi yajñādikaṃ karma loke mūlyārthitayā phalābhimānābhāvānna yatastatkarma pāralaukikena phalena yuktaṃ kalpate  / tathā hi yajanti yājakā yajate yajamāna iti nyāyena yajatāmṛtvijāṃ yajñakarma svayaṃ kṛtavatāmapīdamaśvamedhādikaṃ yajñakarma nāsmākaṃ kiñcidapi tu dīkṣitasya puṇyavato vayaṃ kileha yajñakarmāṇi niyamitamūlyamātrārthino 'tra na kecanaiva yajamānaḥ punaramunā karmaṇā svargādiphalabhāgapīti teṣāṃ karmaphalābhimānābhāvānna svayaṃ kṛtamapi karma tadīyena svargādinā phalena yujyate yajamānastu tatra yajñakarma svayamakurvāṇa ṛtviṅnirvartyakarmamukhaprekṣyapi mamedamaśvamedhādikaṃ yajñakarma madīyena dhanenāmyṛtvijaḥ karmaṇi pravṛttā iti mamaiva svargādiphalaṃ dehapātādavaśyaṃbhāvītyakurvāṇasyāpi yathā samīhitakarmaphalābhimānadārḍhyāttattasya karma phalena yujyate  / ata eva kartrabhiprāye kriyāphala iti dīkṣitātkarturyajate yajamāna ityātmanepadaṃ kartranabhiprāye tu parasmaipadaṃ yajanti yājakā iti  / iyānmahimā durlaṅghyo vikalpasvātantryasya yatsvayaṃ kṛtamapi karma phalābhimānābhāvāttatphalena
na yujyate 'nyaiḥ kṛtamapi karma mamedam ityabhimānadārḍhyātphalayuktaṃ syāttasmādṛtvigvyāpāravatkriyamāṇaṃ yoginā karma phalābhimānābhāvānna tatsaphalaṃ bhavediti  // 67  //

evaṃ sarvakarmasu heyopādeyakalpanākalaṅkaparityaktabuddhirjñānī dīptaḥ syādityāha

itthaṃ sakalavikalpānpratibuddho bhāvanāsamīraṇataḥ  /
ātmajyotiṣi dīpte juhvajjyotirmayo bhavati  // 68  //

ittham vyākhyātena prakāreṇa yā bhāvanā ahameva caitanyamaheśvaraḥ sarvātmanā sarvadaivaṃ sphurāmīti yātmani vimarśarūḍhiḥ saiva śanaiḥ prasarantī samīraṇaḥ vāyuriva tena jñānī pratibuddhaḥ bhasmacchanno vāyunā pratibodhito vahniryathā sakalavikalpān paśurasmi karmabandhabaddho deharūpī mamedaṃ putradārādyamunā karmaṇā svargo nirayo vā bhaviṣyatītyādisarvāḥ kalpanā ahamevedaṃ sarvamiti parāmarśaśeṣībhūtāḥ ātmajyotiṣi caitanyakṛśānau dīpte parāhantācamatkārasāre juhvat avikalpakasaṃvidrūpānupraveśena samarpayansa jyotirmayo bhavati dāhyavikalpendhanaparikṣayāddāhakākāraścidagnireva saṃpadyate parapramātrekavapurasāvavaśiṣyata iti yāvat  // 68  //

evaṃ vyākhyātena prakāreṇa yaḥ prakṛṣṭajñānayogābhyāsarataḥ sa śeṣavartanayā kathaṃ kālamativāhayatītyāha

aśnanyadvā tadvā saṃvīto yena kenacicchāntaḥ  /
yatra kvacana nivāsī vimucyate sarvabhūtātmā  // 69  //

yatkiñcitpuraḥ patitamadanayogyaṃ padārtham aśnan camatkurvanna punarniyamenedaṃ pavitramidamapavitramidaṃ kadannamidaṃ miṣṭānnamiti heyopādeyakalpanāvirahādayatnenāpatitaṃ yadapi tadapi samāharan  / tathā saṃvito yena ityādi  / kanthayā carmaṇā valkalena vā tūlapaṭādinā divyātmavastrairvā samācchādita ityubhayathotkarṣāpakarṣābhāvāccharīrācchādanārthakriyārthī bhūtvā nāpi kiñciddveṣṭi nāpi stautīti  / kathametadyataḥ saḥ śāntaḥ sukhaduḥkhādivikalpanātikrānta iti  / tathā yatra kvacana nivāsī iti  / yatra kvacana yādṛśe tādṛśe sthāne svapariśrayamātrārthī na punastasya kṣetrāyatanatīrthādi pavitratvātsvīkāryaṃ bhavati nāpi śmaśānaśvapacasadanādyapavitratvātparihāryaṃ syādayatnena yadyatsthānamāpatitaṃ tattadadhivasati pavitrāpavitrakalpanākalaṅkavirahāt  / vimucyate ityevamapi śeṣavartanayā parānugrahārthapravṛttaḥ kālamativāhayan vimucyate paramaśivībhavati  / uktaṃ ca yena kenacidācchanno yena kenacidāśitaḥ  / yatra kvacana śāyī yastaṃ devā brāhmaṇaṃ viduḥ  //

iti  / mokṣadharmeṣvapi

aniyataphalabhakṣyabhojyapeyaṃ vidhipariṇāmavibhaktadeśakālam  /
hṛdayasukhamasevitaṃ kadaryairvratamidamājagaraṃ śuciścarāmi  //

iti  / kathamevamapi kurvañjñānī svayaṃ mucyetetyāha sarvabhūtātmā iti  / yataḥ sa jñānī sarvabhūtātmā sarveṣāṃ bhūtānāmātmā sarvāṇi ca bhūtāni tasyātmeti kṛtvā na kiñcidbandhakatayā bhavati sarvaṃ vimuktaye 'sya saṃpadyata iti  // 69  //

nāpyevaṃrūpasya nirabhimānasya yatkiñcitkurvato 'pi puṇyapāpasaṃbhava ityāha

hayamedhaśatasahasrāṇyapi kurute brahmaghātalakṣāṇi  /
paramārthavinna puṇyairna ca pāpaiḥ spṛśyate vimalaḥ  // 70  //

ya evam paramārthavit svātmamaheśvarasvabhāvasatattvajñaḥ so 'śvamedharājasūyāptoryāmādiyajñānniḥsaṃkhyānphalakāmanābhimānavirahātkartavyatāmātramidamityevaṃ kṛtvā krīḍārthaṃ yadi kadācidvihitāni karmāṇi vidadhātyathavā brahmahananasurāpānastainyādīni pramādopanatāni mahāpātakānyavihitānyapyaśarīratayā cetyubhayathāhaṃ mametyabhimānābhāvātparameśvarecchaivetthaṃ vijṛmbhate mama kimāyātamiti buddhyā na puṇyaiḥ śubhaphalairnāpi pāpaiḥ aśubhaiḥ sa jñānī spṛśyate malinīkriyata iti  / kathametadityāha vimalaḥ iti  / yatastasya vigatāḥ prakṣīṇā āṇavamāyīyakārmamalāḥ saṃsaraṇahetava iti  / evaṃ malinasya hi pramāturvicchinnadehādipramātṛtayātmātmīyābhimānabhāvo yena mamedaṃ karma śubhamidamaśubhamityabhimānadaurātmyātpuṇyapāpasaṃcayayogaḥ syādyasya karmaphalasaṃcayo mamatvaheturmalapracayo vigataḥ syāttasyābhimānābhāvātkathaṃ puṇyapāpasparśaḥ  / yathā śrībhagavadgītāsu

yasya nāhaṅkṛto bhāvo buddhiryasya na lipyate  /
hatvāpi sa imāṃllokānna hanti na nibadhyate  //

iti  // 70  //

evaṃvidhasya jñānino niyatacaryāṃ parāmṛśannāha

madaharṣakopamanmathaviṣādabhayalobhamohaparivarjī  /
niḥstotravaṣaṭkāro jaḍa iva vicaredavādamatiḥ  // 71  //

madaḥ dehapramātṛtābhimānaḥ harṣaḥ alabdhasya lābhātpramodaḥ kopaḥ krodhaḥ manmathaḥ saṃbhogābhilāṣaḥ viṣādaḥ iṣṭaviyogānmūḍhatvam bhayam śatroḥ siṃhavyāghrādervā daraḥ lobhaḥ kārpaṇyam mohaḥ bhūteṣvātmātmīyabhāva ityetāndehasaṃskārapratyavamarśānmadhye madhye samāyātānapi sarvaṃ brahmāsmīti parivarjayatyavikalpakasaṃvidrūpānupraveśena svātmapratyavamarśaśeṣībhūtānsaṃpādayati  / tathā nirgataḥ stotravaṣaṭkārebhyo yaḥ sa evaṃ stutyasya vyatiriktasyābhāvānna tasya stotrādyupayogo nāpi vaṣaḍādimantrasaṃśrayo bhinnasya devatāviśeṣasya virahātkevalaṃ saḥ jaḍa iva vicaredavādamatiḥ iti  / pūrṇatvādākāṅkṣāvirahācconmatta ivetikartavyatārūpe śāstrīye karmaṇi pramāṇopapanne vā prameyasatattve pramātṛbhiḥ sahedamupapannamidaṃ neti vicārabahiṣkṛtabuddhirnāpi svātmanyupadeśamapekṣate parānupadeṣṭuṃ vā prameyamupanyasyatīti dāntaprāyo bhūtvā sarvaṃ brahmāvalokayankrīḍārthaṃ viharedeveti jaḍatvena nirūpitaḥ  // 71  //

evamapi parivarjyamānenāpi madādivargeṇa vayamiva jñānī sati śarīre kimiva na spṛśyata ityatra kāraṇamāha

madaharṣaprabhṛtirayaṃ vargaḥ prabhavati vibhedasaṃmohāt  /
advaitātmavibodhastena kathaṃ spṛśyatāṃ nāma  // 72  //
samanantarakārikāvyākhyātaḥ madādivargo 'yaṃ vibhedasaṃmohāt ityātmātmīyarūpo yaḥ vibhedasaṃmohaḥ apūrṇatvakhyātistataḥ prabhavati paśupramātṛbhyo dvaitabhrāntyā heyopādeyatayā samutpadyate  / yaḥ punaḥ sarvaṃ brahmāsmīti paramādvayātmabodhaḥ prakṛṣṭajñānyākāśakalpaḥ saḥ tena madādivargeṇa kathaṃ nāma spṛśyatām kena prakāreṇābilīkriyatāṃ bhinnaṃ vastu bhinnasya hi kadācitsvarūpamarpayatāṃ brahmabhūtatvena gṛhīto madādivargo brahmabhūtasya jñāninaḥ samānajāteḥ kathaṃ virodhāya syāditi  // 72  //

bāhyastavanahavanavargo 'pi dvaitasamāśraya eva na tasya paritoṣāyālamityāha

stutyaṃ vā hotavyaṃ nāsti vyatiriktamasya kiñcana ca  /
stotrādinā sa tuṣyenmuktastannirnamaskṛtivaṣaṭkaḥ  // 73  //

stutyam kiñciddevatārūpam hotavyaṃ vā kiñcijjñānino 'dvayabodharūpasya na vyatiriktam bhinnarūpaṃ vidyate yatstūyate hūyate veti  / nāpi kartavyam ityevaṃrūpatayā ca stotrādinā saḥ ātmajñaḥ paritoṣaṃ yātyabhedabodhasaṃbhogena hi nityānandamayatvātkṛtrimamānandaṃ nādriyate tasmānnirgato namaskṛtivaṣaṭkebhyo yaḥ sa eva muktaḥ vedānteṣu evaṃstuta iti  // 73  //

na ca tasya bhinnena devagṛheṇopayogaḥ svaśarīramevātmadevatādhiṣṭhānaṃ saṃvidāśrayo vā nānyaḥ kaściditi no bhinnaṃ devagṛhamasyetyāha

ṣaṭtriṃśattattvabhṛtaṃ vigraharacanāgavākṣaparipūrṇam  /
nijamanyadatha śarīraṃ ghaṭādi vā tasya devagṛham  // 74  //

tasya jñānino nijaḥ parakīyo vā deha eva devatāveśma svātmadevatāyā bhogyādhāratvāt  / bāhyastu mervādiprāsādastadā devagṛhībhavati yadā guruṇā śarīravyāptyā ṣaṭtriṃśattattvakalanarūpayā parikalpitaḥ syāttadgato bāhyo 'pi devaḥ svātmavyāptyā cidghanatvena parigṛhītaścettadā so 'pi tatra devo bhavedanyathobhayametajjaḍaṃ śilāśakalakalpameva kathaṃ bhaktānuddharenmṛtānsāmīpyādi vā nayedityevaṃ mukhyayā vṛttyā śarīram saṃvidāśrayatvāddevagṛhaṃ tadgataḥ sarveṣāmapi svātmā deva iti deha eva saṃprabuddhasya devagṛham  / kīdṛśaṃ tadityāha ṣaṭtriṃśattattva iti  / bāhyaṃ ṣaṭtriṃśattattvavyāptyā parikalpyate paraṃ dehadevagṛhaṃ punaḥ sākṣātṣaṭtriṃśatā tattvaiḥ bhṛtam poṣitam  / bāhyadevagṛhe gavākṣaracanā bhavatīdaṃ tu vigraharacanāgavākṣaparipūrṇam iti vigrahe śarīre racanā indriyadvāraparipāṭiḥ saiva tamorikalpanā tayā paripūrṇam akṣuṇṇamiti bāhyadevagṛhasadṛśam  / na kevalaṃ sarīraṃ saṃvida āśraya iti kṛtvā devagṛhaṃ yāvadyatkiñcidvā saṃvidadhiṣṭhitaṃ tatsarvaṃ tasya devagṛhamityāha ghaṭādi vā iti  / ghaṭādyupalakṣitaṃ viṣayapañcakamidaṃ bhogyarūpaṃ cakṣurādidvāreṇa saṃvidādhiṣṭhitam

bhoktaiva bhogyabhāvena sadā sarvatra saṃsthitaḥ  /

iti spandaśāstropadeśadṛśā saṃvinmayameva jñānino bhūtaśarīravadghaṭādi viśvaṃ bhāvaśarīramiti kṛtvā tadapyabhinnaṃ svaśarīravat devagṛham devasya krīḍāvataḥ svatantrasya svātmamaheśvarasya gṛham bhogyādhiṣṭhānamiti  // 74  //

bāhyadevagṛhe kila bhaktaḥ puṣpādyāharaṇapūrvaṃ devapūjāparo dṛṣṭo dehadevagṛhe punarjñānī kiṃ kurvannadhitiṣṭhatītyāha

tatra ca paramātmamahābhairavaśivadevatāṃ svaśaktiyutām  /
ātmāmarśanavimaladravyaiḥ paripūjayannāste  // 75  //

tasminsvadehadevagṛhe prakṛṣṭayogī paramaḥ sarvātiśāyī yaścaitanyalakṣaṇaḥ ātmā sa eva niḥśeṣaśabdādiviṣayopabhogavilāyanapragalbhatvāt bhairavaḥ bharaṇaravaṇavamanasvabhāvaḥ sa eva śivadevatā prakṛṣṭaśreyorūpo devastām paripūjayannāste anavarataṃ vakṣyamāṇena krameṇa tāṃ tarpayanparisphuret  / nanu bāhyadevatā parivārayutā bhavatyetāṃ kiṃparivārayutāṃ samarcayedityāha svaśaktiyutām iti  / svāḥ caitanyaraśmirūpāścinnirvṛtīcchājñānakriyāśaktīnāṃ vibhavātmikāścakṣurādikaraṇaśaktayastābhiḥ yutām samantādāvṛtām  / kaiḥ paripūjayannāsta ityāha ātmāmarśana ityādi  / svātmaivedaṃ sarvamiti yat āmarśanam sarvapadārthānāṃ saṃvidrūpatayā pūrṇāhantāviśrāntilakṣaṇo yaḥ parāmarśastena dvaitakāluṣyakalaṅkaparikṣayāt vimalāni yāni śabdādiviṣayapañcakarūpāṇi pūjārtham dravyāṇi jāḍyāpagamena viśuddhāni tairātmāmarśanavimaladravyairiti  / ayamāśayo jñānī heyopādeyabhedakalaṅkaparityāgenāyatnopanataṃ śabdādiviṣayapañcakaṃ śrotrādikaraṇadevībhiḥ samāhṛtyāntaścamatkurvansvātmanābhedamāpādayatītyevamanavarataṃ prativiṣayasvīkārakāle yo 'ntarabhedena camatkāraḥ pūrṇāhantāsphuraṇametadeva svātmadevatāpūjanamata eva śabdādayo viṣayāḥ pūjopakaraṇamityavadhānavatā viṣayagrahaṇakāle pratikṣaṇaṃ svātmadevatāpūjakena bhāvyamiti rahasyavidaḥ  / etadeva stutidvāreṇa rājānakarāmo dṛbdhavānyathā

nityoddāmasamudyamāhṛtajagadbhāvopahārārpaṇa-
     vyagrābhistava taijasīprabhṛtibhiryacchaktibhistarpyate  /
tanmāṃsāsravasāsthikūṭakalile kāye śmaśānālaye
     rūpaṃ darśaya bhairavaṃ bhavaniśāsaṃcāravīrasya me  //

iti  // 75  //

pūjānte tāvadagnihavanena bhāvyamiti jñāninaḥ kathaṃ tadityāha

bahirantaraparikalpanabhedamahābījanicayamarpayataḥ  /
tasyātidīptasaṃvijjvalane yatnādvinā bhavati homaḥ  // 76  //

tasya evaṃvidhasya svātmadevatāpūjakasya atidīpte parāhantācamatkārabhāsvare caitanyāgnau yatnādvinā tilājyendhanādisvīkārakadarthanāyā ṛte homaḥ vahnitarpaṇaṃ saṃpadyate  / kiṃ kurvata ityāha bahirantara ityādi  / bahiḥ nīlādau prameye yatsvaparapramātṛkalpanamantargrāhye sukhādau ca yatsaṃkalpanamityevaṃrūpo yaḥ bhedaḥ bāhyābāhyayoḥ pramātṛprameyayorniścayasaṃkalpanābhimānavṛttisvabhāvaṃ nānātvametadeva mahābījam pramātṛprameyayostataḥ samutpattestasya kalpanārūpasya bhedabījabhūtasya nicayaḥ bhedasyānantyādrāśistam arpayataḥ paramādvayadṛṣṭyāvikalpakasaṃvidrūpānupraveśena svātmavahnau juhvata iti  / ayamāśayaḥ parabrahmātmakasya yogino dehādipramātṛtābhimānābhāvādyaḥ svarasasiddhaḥ svaparapramātṛprameyakalanaparikṣayaḥ sa evākṛtrimo homaḥ  / yathāha bhaṭṭaśrīvīravāmanaḥ

yatrendhanaṃ dvaitavanaṃ mṛtyureva mahāpaśuḥ  /
alaukikena yajñena tena nityaṃ yajāmahe  //

iti  // 76  //

evaṃrūpasya yājakasya dhyānamāha

dhyānamanastamitaṃ punareṣa hi bhagavānvicitrarūpāṇi  /
sṛjati tadeva dhyānaṃ saṃkalpālikhitasatyarūpatvam  // 77  //

niyatākāracintitasyākārasyānyatra manovṛttergamanātkṣayo 'stīdam punaranastamitaṃ dhyānam yasmāt eṣa bhagavān apyanantaḥ svātmarūpo maheśvaraḥ kriyāśaktisvabhāvavikalpasvātantryeṇa yāni vicitrāṇi rūpāṇi sṛjati anavarataṃ nānāpadārthānvikalparūpānākārānbuddhidarpaṇe samullikhati tadeva astodayavarjitametasya dhyānam cintanaṃ nāto 'nyatkiñcit  / itaratra tu devatāviśeṣe nānāvaktrāṅgaparikalpanayā naiyatyaṃ syāt  / sarvo manovyāpāraḥ parāśaktisphārapallavabhūta iti jānānasyānavacchinnamidaṃ sarvaṃ parameśvarībhūtam  / tathā saṃkalpena manasā ālikhitam saṃvidbhittau citrīkṛtam satyarūpatvam paramārthatā yasya dhyānasya tat  / evaṃ yataḥ sarvamidaṃ prakāśamānaṃ vikalpollikhitaṃ manovyāpārarūpamapi prakāśānatiriktaṃ satyaṃ sarvatra saṃvidanugamāditi  / taduktaṃ śrīmatsvacchandaśāstre

yatra yatra mano yāti tatra tatraiva dhārayet  /
calitvā kutra gantāsi sarvaṃ śivamayaṃ yataḥ  //

iti  / tathā śaivopaniṣadi

yatra yatra mano yāti bāhye vābhyantare priye  /
tatra tatra śivāvasthā vyāpakatvātkva yāsyati  //

iti  / tasmātsvarasoditametadyogino dhyānamiti  // 77  //

japaścāsya kīdṛśaḥ syādityāha

bhuvanāvalīṃ samastāṃ tattvakramakalpanāmathākṣagaṇam  /
antarbodhe parivartayati ca yatso 'sya japa uditaḥ  // 78  //

vakṣyamāṇena krameṇa yo viśvasya pratikṣaṇamabhedena parāhantāpratyavamarśaḥ saḥ ayam asya japa uditaḥ akṛtrimatvena kathitaḥ  / ko 'sāvityāha bhuvanāvalīṃ samastām ṣaṭtriṃśattattvasamūhāntarvartinīṃ caturviṃśatyuttaraśatadvayasaṃkhyātāṃ prākārapaṅktiṃ niḥśeṣāṃ tathā tattvakramakalpanām iti tattvakramasya ātmavidyāśivākhyasya parikalpanām paricchedam athākṣagaṇam ityantarbahiṣkaraṇarūpamindriyasamūhaṃ ceti antarbodhe madhyamaprāṇaśaktyakṣasūtrabhūtāyāṃ svasaṃvittau nādabindupravāhakrameṇa yatparivartayati araghaṭṭaghaṭīyantravatpratiprāṇavikṣepasṛṣṭisthitisaṃhārakrameṇa sarvametatsvasaṃvittau paribhramayati pratikṣaṇaṃ nādātmanā parāmṛśatīti yāvat  / sa eva pūrṇāhantāviśrāntilakṣaṇo 'kṛtrimo 'sya japaḥ  / ayamāśayo japaḥ kila vācyarūpāyā devatāyā vācakasya mantrasyoccāraḥ sa cākṣamālayā prāṇaśaktivyāptikayākṣaparivartanakrameṇa saṃkhyeyaḥ  / paramādvayayoginastu svā prāṇaśaktistantubhūtā madhyamaprāṇe pravāhakrameṇa nadantī svarasoditā sarvākṣakroḍīkāreṇa sahajaivākṣamālocyate yataḥ sarvamidaṃ vācyaṃ ṣaṭtriṃśattattvātmakaṃ viśvaṃ prāṇaśaktāveva pratiṣṭhitaṃ satpratiprāṇavikṣepamudayavyayakrameṇa parāsvabhāvā bhagavatī prāṇasvarūpamāśritya vimṛśantī pratiprāṇaspandamavadhānavato yogino japamakṛtrimaṃ sādhayati  / atra japasaṃkhyā

ekaviṃśatsahasrāṇi ṣaṭśatāni divāniśam  /
japo devyāḥ samuddiṣṭaḥ sulabho durlabho jaḍaiḥ  //

iti śaivopaniṣadi  / śivasūtreṣu

kathā japaḥ  /

iti  / evameṣa vandyacaraṇānāmavadhānavatāmeva gocara iti  // 78  //

idaṃ vratamasyetyāha

sarvaṃ samayā dṛṣṭyā yatpaśyati yacca saṃvidaṃ manute  /
viśvaśmaśānaniratāṃ vigrahakhaṭvāṅgakalpanākalitām  // 79  //

viśvarasāsavapūrṇaṃ nijakaragaṃ vedyakhaṇḍakakapālam  /
rasayati ca yattadetadvratamasya sudurlabhaṃ ca sulabhaṃ ca  // 80  //

evam yat vakṣyamāṇam etadevāsya jñāninaḥ vratam svātmadevatāsamārādhanāya niyamaḥ  / kīdṛśaṃ tadāha sudurlabhaṃ ca sulabhaṃ ca iti  / suṣṭhu kṛtvā duḥkhenākhyātiparikṣayādanyopāyaparihārarupeṇa parameśvarānugraheṇa labhyata ityataḥ sudurlabham tathā sukhena bāhyāsthibhasmādyābharaṇāhāraniyamādisvīkārakadarthanāṃ ca vinā labhyata ityataḥ sulabhaṃ ca  / kiṃ tadvratamityāha sarvam ityādi  / yatsarvamidaṃ prātītikaṃ bhedāvabhāsarūpaṃ yuktyāgamānubhavapariśīlanenābhedadṛśā sarvamidamekaḥ sphurāmīti samīkṣate  / yathā śrībhagavadgītāsu
sarvabhūtasthamātmānaṃ sarvabhūtāni cātmani  /
īkṣate yogayuktātmā sarvatra samadarśanaḥ  //

ityādi  / evamabhedabuddhidārḍhyameva vratam  / anyacca yadviśvaśmaśānaniratāṃ saṃvidaṃ manute tadapi vratam  / yathā viśvam grāhyagrāhakasvabhāvaṃ ghaṭadehādijaḍalakṣaṇapadārthaśavaśatasamākrāntamiti kṛtvā tadeva śmaśānam pitṛvanaṃ yataḥ saṃvidekā bhagavatyajaḍāto 'nyattadullāsitaṃ sarvamidaṃ śavasthānīyaṃ jaḍamato viśvasya śmaśānena sādṛśyam  / tasmin viśvaśmaśāne niḥśeṣeṇa ratāṃ saṃvidam samutpattinidhanatayā mahābhīṣaṇe madhyavartinīm manute avabudhyate  / vratī kila śmaśāne vasatyayaṃ punaralaukiko vratī sarvatrāhamevaikacittatattvaparamārtha iti matvā jaḍaiḥ paśupramātṛbhirghaṭādibhiḥ prameyaiśca paretasthānīyaiḥ sahonmattavatkrīḍāṃ kurvāṇaḥ saṃsārabhuvamimāṃ sarvapramātṛprameyanidhanatayā bhīṣaṇaṃ śmaśānasvabhāvāmadhyāste  / anyacca yat vigrahakhaṭvāṅgakalpanākalitāṃ saṃvidaṃ manute iti  / vigrahaḥ śarīraṃ sa eva khaṭvāṅgakalpanā kaṅkālavidhiryoginaḥ kila svaśarīrapramātṛtābhimānadurgrahasaṃkṣayāccharīrātītaṃ svātmānaṃ manyamānasya saṃskāraśeṣībhūto vigrahaḥ śavaprāyamevetyavadhārayataḥ svaśarīrameva kaṅkālamudrākalpanā tayā kalitām bhogyādhāratvena mudritām  / vīravratino hi śmaśānasthasya khaṭvāṅgamudrayā bhāvyamato 'sya svasaṃvidvapuṣaḥ svaśarīramapi vedyatayā bhinnamavadhārayataḥ saiva khaṭvāṅgamudretyetadapyasya vratam  / tathā vedyakhaṇḍakakapālaṃ rasayati carvayati yacca śabdādiviṣayapañcakalakṣaṇaṃ sarvabhogyarūpamidam vedyam jñeyatvakāryatvābhyāṃ paricchinnamiti khaṇḍakam karparaprāyaṃ tadeva kapālam śiro 'sthiśakalam yadrasayati sārāharaṇakrameṇa pūrṇāhantāviśrāntyā camatkuruta ityetadapi vratam  / vratinā kila kapālasthaṃ vīrapānaṃ rasyata ityāha viśvarasa iti  / viśvasmin vedyaśabdādiviṣayapañcakarūpe kapālakhaṇḍe yo 'sau sārabhāgaścarvaṇāmṛtamayoṃ'śaḥ sa eva paramānandadāyitvāt rasāsavaḥ uttamaṃ pānaṃ
tena pūrṇam nirbharam  / etaduktaṃ syādviśvasya yaḥ śalkasthānīyaḥ kaṭhinoṃ'śaḥ pātrakalpaḥ sa eva kapālaṃ tadgataḥ sārabhāgaścamatkārakṣama āhlādadāyitvātpānamiti  / kapālaṃ tu vratinaḥ karagataṃ bhavatītyāha nijakaragam iti  / nijāḥ svātmīyā ye karāḥ cakṣurādikaraṇadevīsvabhāvāścinmarīcayasteṣu bhogyatayā tadvedyakhaṇḍakaṃ viṣayatvaṃ gacchatīti nijakaragaṃ yathā pāṇisthena kapālena pānaṃ pīyate tathaiva vedyakhaṇḍakakapālena viśvarasāsavaścakṣurādisaṃvitkarairyoginā samāhṛtyāsvādyate  / ayamāśayo yogī sarvadaiva yathopanataṃ viṣayapañcakaṃ karaṇadevībhirāhṛtya yuktyā svacaitanyabhairavaviśrāntimavyucchinnāṃ bhajamānaścaramakṣaṇaparyantaṃ yathopadiṣṭamadvayadṛśā nirvāhayatītyetadevāsya pariśīlitasadgurucaraṇapaṅkajasya vratamato 'nyaccharīraśoṣaṇamātramiti  // 80  //

prākpratipāditaṃ saṃkalayannasyopadeśasyotkṛṣṭatvamāha

iti janmanāśahīnaṃ paramārthamaheśvarākhyamupalabhya  /
upalabdhṛtāprakāśātkṛtakṛtyastiṣṭhati yatheṣṭam  // 81  //

iti samanantaroktena prakāreṇa yatpratipāditaṃ rahasyam paramārthamaheśvarākhyam tāttvikaṃ maheśvaram upalabhya svātmani dṛḍhapratipattyā samyaganubhūya  / kīdṛgāha janmanāśahīnam iti yenādhigatenotpattimaraṇe na syātāmiti yāvat  / kṛtakṛtyastiṣṭhati yatheṣṭam iti yogyetatprāpya kartavyatāntarasyābhāvānniṣpannaparapuruṣārthaḥ yatheṣṭam svecchātikramaṃ vinā svātantryeṇa cakrabhramavaddhṛtaśarīraḥ tiṣṭhati kālamativāhayannāste  / kathamāha upalabdhṛtāprakāśāt iti  / etadrahasyapariśīlanena sarvāsu daśāsvanubhavitṛtayā prakāśaḥ parisphuraṇaṃ tasmāditi śarīrastho 'pi pūrṇānandamaya iti yāvat  // 81  //

itthaṃ svātmānaṃ yaḥ kaścijjīvatāṃ madhyājjānānaḥ sa sarvastadrūpaḥ syādityadhikāriniyamābhāvamāha

vyāpinamabhihitamitthaṃ sarvātmānaṃ vidhūtanānātvam  /
nirupamaparamānandaṃ yo vetti sa tanmayo bhavati  // 82  //

ittham ityuktena prakāreṇa vyāpinam anavacchinnacidānandaikaghanaṃ śivam abhihitam yuktyāgamānubhavapariśīlanakrameṇāveditam yo vetti yaḥ kaścideva prāṇiprāyo jānāti saḥ sarvastyaktasaṃkocaḥ tanmayaḥ śiva eva syāditi  / atra svātmajñāne nādhikāriniyamo yato ye kecana janmamaraṇādidoṣāghrātāstiryañco 'pi vā te sarve svātmamaheśvarapratyabhijñānāttanmayā bhavantīti yacchabdasya parāmarśaḥ  / kīdṛśaṃ ca sarvātmānam iti  / sarveṣāṃ pramātṛprameyāṇāmātmā sarvāṇi vā pramātṛprameyāṇi yasyātmā taṃ sarvottīrṇaṃ sarvamayamiti yāvat  / ata eva vidhūtam nyakkṛtaṃ sarvadā sarvatra cidrūpatayā sphuraṇāt nānātvam bhedānantyaṃ yena tamevamākāṅkṣāvirahāt nirupamaḥ viśeṣaṇarahitaḥ prakṛṣṭaḥ ānandaḥ yasya tamevaṃvidhamiti svātmānaṃ jānānaḥ sarvaḥ śivarūpī syāditi  // 82  //

evamadhigatasvātmamaheśvaraḥ svaśarīrādhikāraparikṣaye kutra śarīraṃ parityajetkiṃ vā yātītyādisaṃśayaṃ pariharati

tīrthe śvapacagṛhe vā naṣṭasmṛtirapi parityajandeham  /
jñānasamakālamuktaḥ kaivalyaṃ yāti hataśokaḥ  // 83  //

evaṃ pariśīlitasvasvarūpo jñānī sarvamidaṃ svātmaprakāśasvātantryamiti paramādvayadṛśā gāḍhaṃ samāśvastahṛdayaḥ tīrthe prayāgapuṣkarakurukṣetrādau mahāpuṇye sthāne 'tha vā śvapacasadane 'ntyajanagṛhopalakṣite 'tipāpīyasi śarīraṃ muñcannityubhayathā svīkāraparityāgakadarthanāvirahito 'pyātmajñānādeva kaivalyaṃ yāti kalevaraparikṣayātpradhānādikāryakāraṇavargebhyo 'nyāṃ cidānandaikaghanāṃ turyātītarūpāṃ kevalatāṃ yātīti yāvat  / yato 'sya sarvamidaṃ viśvaṃ svātmanā pūrṇaṃ samadṛśā parameśvarādhiṣṭhitaṃ paśyato na kṣetrākṣetrapravibhāgo 'ta eva hataḥ parākṛto vikalpaśaṅkāsamutthaḥ śokaḥ yena sa evam  / yathoktam

himavati gaṅgādvāre vārāṇasyāṃ kurau prayāge vā  /
veśmani caṇḍālādeḥ śivatattvavidāṃ samaṃ maraṇam  //
iti śrīnirvāṇayogottare  / nāpyasya dehapātāvasare smṛtyupayoga ityāha naṣṭasmṛtirapi iti  / āstāṃ saṃsmṛtirityapiśabdārthaḥ  / yadi vā sa jñānī śarīratyāgakāle tadutthavātapittaśleṣmābhibhavāt naṣṭasmṛtiḥ kāṣṭhapāṣāṇatulyatvādvigatasvātmasaṃbodhaḥ kalevaramavaśo bhutvā tyajati tathāpi prāgadhigatasvātmajñānaḥ kaivalyamavaśyaṃ yāti tato na svātmajñānādhigame pramayasamaye smaraṇāsmaraṇe viśeṣo 'sti  / nanu tīrthātīrthapravibhāgo 'sya svātmajñānavido mā bhūdyatpunarantakāle sa naṣṭasmṛtirapīti yadeva svātmajñānamupāyatayā gṛhītaṃ tasya dehapātāvasare tu vismaraṇaṃ cettarhi kathaṃ sa muktaḥ syāt  / yaduktam

antakāle 'pi māmeva smaranmuktvā kalevaram  /
yaḥ prayāti sa madbhāvaṃ yāti nāstyatra saṃśayaḥ  //

iti śrīgītāsu  / evamapyatra smaraṇasyaiva upayogo yadapi parameśvarasmaraṇābhāve 'pyantakāle tadbhāvāpattiḥ syāttarhi sarvaḥ paśujanaḥ pramayasamaye mūḍho 'pi viśeṣābhāvātparameśvarasamāpattiṃ yāyādvākyāni caivamādīnyapramāṇāni syurna caivamityatrottaramāha jñānasamakālamuktaḥ iti  / satyaṃ nāsya smaraṇenopayogaḥ kintu sadguruṇā yadaiva tasya karṇamūle svātmamaheśvarajñānopadeśaḥ kṛtastasminneva kāle 'hameva sarvamidamityadhirūḍhasvātmajñānaparamārtho vigalitamāyādikañcukabhāvo nānyatkiñcidapekṣate kevalaṃ saṃskāraśeṣatayā cakrabhramavaccharīraṃ vahamāna iti na punastasyottarakāle smaraṇāsmaraṇakadarthanā yasmādajñānajanitāṇavamāyīyakañcukasaṃbandhe sati dehakañcukaṃ prabhavati svātmajñānopadeśenājñānajanitakañcukakṣayātkathaṃ dehakañcukaṃ vinaṣṭaprāyaṃ paryante jñānino yantraṇāṃ kartumalamiti svātmajñānakathanāvasara eva sa jīvanneva muktaḥ syāt  / yathoktaṃ kularatnamālikāyāṃ sāhastrikāyām

yadā guruvaraḥ samyakkathayettadasaṃśayam  /
muktastatraiva kāle 'sau yantravatkevalaṃ vaset  //

iti  / śrīmanniśāṭane 'pi

godohamiṣupātaṃ vā nayanonmīlanātmakam  /
sakṛdyuktaḥ pare tattve sa mukto mocayetparān  //
yasmātpūrvaṃ pare nyasto yenātmā brahmaṇi kṣaṇam  /
smaraṇaṃ tu kathaṃ tasya prāṇānte samupasthite  //

iti  / atha vātmavidaḥ paryantakṣaṇaḥ svānubhavaikasākṣī kenānubhūyate yadvaśāttasya smaraṇamasmaraṇaṃ vā parikalpyate yāvatā tatrārvāgdṛśāṃ nāsti gocara iti sarvajñāstratra praṣṭavyā na punaḥ śarīraceṣṭāmātrānmaraṇāvasare 'dhigataparamārthasyāpi dehatyāgakṣaṇaḥ śubhāśubhatvenānumātuṃ śakyaḥ  / tasmādavaśyameva sadā tadbhāvabhāvitaṃ svātmajñānavidaṃ svātmastha eva parameśvaro maraṇāvasare svaṃ svarūpaṃ kāṣṭhapāṣāṇatulyamapi smārayati  / yaduktam

svasthaceṣṭāśca ye martyāḥ smaranti mama nārada  /
kāṣṭhapāṣāṇatulyāṃstānantakāle smarāmyaham  //

tathā

sthire cetasi susvasthe śarīre sati yo naraḥ  /
dhātusāmye sthire smartā viśvarūpaṃ ca māmakam  //
tatastu mriyamāṇaṃ taṃ kāṣṭhapāṣāṇasaṃnibham  /
ahaṃ smarāmi madbhaktaṃ nayāmi paramāṃ gatim  //

iti bhagavatā lakṣmīsaṃhitāyāmuktam  / evamatra sadātadbhāvabhāvitatvameva heturanyathā pūrvānubhavasaṃskāradārḍhyaṃ vinā kathamante smṛtirapi syāditi na kenacidapi jñānino maraṇāvasare samupayoga iti  // 83  //

yadi punastīrthādyāśrayaṇamuktaprakāreṇa na kutracidapyaṅgabhāvaṃ yāti tarhi kimiti vidvadbhistatsamāśrīyata iti viṣayavibhāgamāha

puṇyāya tīrthasevā nirayāya śvapacasadananidhanagatiḥ  /
puṇyāpuṇyakalaṅkasparśābhāve tu kiṃ tena  // 84  //

yeṣāṃ viduṣāmapi dehādipramātṛtāgrahaḥ sāmprataṃ na vigalitaḥ svātmajñānacarcāyāṃ ca na tathā samāśvāsasteṣāmiṣṭāpūrtādidharmasaṃgrahaṃ kurvatāmadharmasaṃgrahaṃ vā prayāgāditīrthasevā maraṇāvasare kṣetraparigrahaḥ puṇyāya uttamalokaprāptaye niścitaṃ syādeva  / tathaiva śvapacasadananidhanagatiḥ iti  / śvapākādigṛhopalakṣite pāpīyasi sthāne nidhanagatiḥ pramayaprāptiḥ nirayāya avīcyādinarakapātāyaiteṣāṃ kimiti na bhaveddehapramātṛtāgrahasya vidyamānatvāt  / maraṇasthānānuguṇaṃ bhogamapi bhuktvā śubhāśubheṣu deheṣu jāyante punarmriyante cetyanavaratajanmamaraṇaprabandhā dehādyātmamānina evaṃprāyāḥ syuḥ  / yasya punaḥ svātmajñānapratyavamarśadārḍhyāddehādipramātṛtābhimāno niḥśeṣeṇa vigalitastasya cinnabhaḥsvabhāvasya dharmādharmasvabhāvavāsanāsparśaprakṣaye vṛtte sati kiṃ tena  / evaṃ sa yataḥ śubhāśubhakarmabhājāṃ tīrthādiparigrahastatastena tīrthasevādinā vimalasya jñānino nāstyupayogaḥ  / yaduktaṃ mānave dharmaśāstre

yamo vaivasvato rājā yastavaiṣa hṛdi sthitaḥ  /
tena cedavivādastvaṃ mā gaṅgāṃ mā gayāṃ gamaḥ  //

iti  / atra dehātmamānitaiva hṛdayavartinī yamaḥ sā yaiḥ pūrṇasvātmamaheśvarasvabhāvamupalabhya saṃbhakṣitā teṣāṃ kathameṣa tīrthādisevāprayāsa iti siddhāntaḥ  // 84  //

nanu prākpratipāditaṃ yathā jñānadagdhāṇavamāyīyakārmamalasvarūpa ātmā piṇḍapātātsvasvarūpastha eva na punarbhavaprarohaṃ vidhatte dagdhabījamivāṅkuramiti  / svātmajñānāvirbhāvasamakālaṃ dehakañcukabhaṅgaścettarhyuttaratra mā vidhattāṃ yatpunarvidyamāne dehādikañcukabandhe kathaṅkāraṃ sa tadgatadharmācchurito na syāttadācchuritaḥ sanmṛtaḥ kathaṃ na saṃsārīti codyamapavadati

tuṣakambukasupṛthakkṛtataṇḍulakaṇatuṣadalāntarakṣepaḥ  /
taṇḍulakaṇasya kurute na punastadrūpatādātmyam  // 85  //

tadvatkañcukapaṭalīpṛthakkṛtā saṃvidatra saṃskārāt  /
tiṣṭhantyapi muktātmā tatsparśavivarjitā bhavati  // 86  //

tuṣakambukābhyām suṣṭhu pṛthakkṛtaḥ viśliṣṭo yaḥ taṇḍulakaṇaḥ tasya yaḥ tuṣadalāntarakṣepaḥ prāgiva punastatraiva vinyāsaḥ sa yathā tuṣadalāntaraprakṣepaḥ taṇḍulakaṇasya tadrūpatayāṅkurajananakṣamatvena sthito 'pi tādātmyam gāḍhāvaṣṭambham na kurute ayaḥśalākāvadbhinnāveva tuṣataṇḍulau tiṣṭhato na punarekakāryajananavyagrau bhavataḥ tadvat tathaiva jñānina iyam saṃvit cetanā kañcukapaṭalyāḥ āṇavamalādikañcukasamūhāt pṛthakkṛtā ahameva svātmamaheśvarasvabhāvo viśvātmanā sarvadā sarvatra sphurāmīti svātmajñānapariśīlanadārḍhyātsamuddhṛtā atra ityasyāṃ kañcukapaṭalyāṃ kañcitkālaṃ śeṣavartanayā dehabhāvena tiṣṭhantyapi sthitā satī vimuktātmā pradhvastabandhā tatsparśavivarjitā bhavati tasyā dehādikañcukapaṭalyāḥ sparśaḥ puṇyāpuṇyarūpakārmamalopajanito ya uparāgaḥ saṃsārāṅkurajananakṣamastena vivarjitā parihṛtā saṃpadyate yathā tuṣadalāntare kṣiptastaṇḍulo 'ṅkuraprarohasparśarahito bhavatīti  / idamuktaṃ syādajñānakāraṇakastāvatsaṃsārastatra samuditasvātmajñānadalitakañcukasya yoginaḥ saṃvinna punaḥ saṃsāraheturajñānajanitasāmagrīvaikalyānnāpi tasya śeṣavartanayā saṃskāravaśena tiṣṭhannayaṃ dehakañcukabandho jñānāgnidagdhājñānamūlaḥ svagatadharmāvirbhāvena saṃsāraprarohaṃ dātumalamiti jñānī jīvanneva turīyarūpo dehābhāvātturyātītarūpa ityubhayathā punarna kācitsaṃsāraśaṅketi  // 86  //

nanu śeṣavartanayā yāvaccharīrāvasthitaṃ yogisaṃvedanamadhigatasvasvarūpamapi dehopādhikṛtamālinyasyāpi tāvadaṃśena vidyamānatvādaśuddhameveti dṛṣṭāntena pariharati

kuśalatamaśilpikalpitavimalībhāvaḥ samudgakopādheḥ  /
malino 'pi maṇirupādhervicchede svacchaparamārthaḥ  // 87  //

evaṃ sadguruśāsanavimalasthiti vedanaṃ tanūpādheḥ  /
muktamapyupādhyantaraśūnyamivābhāti śivarūpam  // 88  //

yathā maṇiḥ atipravīṇavaikaṭikasamuddyotitanairmalyaḥ sansamudgakaviśliṣṭatvāt malino 'pi dhūsaraprāyo bhavati sa eva punaḥ samudgakopādhivicchede āvaraṇaviśeṣābhāve svacchaparamārthaḥ yathāvannirmalasvarūpaḥ saṃpadyate  / evam anenaiva prakāreṇedam vedanaṃ sadguruśāsanavimalasthiti paripūrṇasvātmajñānavidyo daiśikapravarastasya yat śāsanam svātmajñānarahasyamukhāmnāyastasya pariśīlanena vigatā kālikārūpasyāṇavamalasya māyīyakārmamalabhittibhūtasya sthitiryasya tadevaṃvidhaṃ maulikamalaprakṣayānnabhorūpamapi vedanam tanūpādheḥ tanuḥ śarīraṃ tallakṣaṇā upādhiḥ viśeṣaṇaṃ tataḥ muktam pṛthakkṛtaṃ viśeṣaṇāntarābhāvāttat śivarūpamābhātyeva dehabhaṅgātparamaśivatvena bhāsata iti yāvat  / yathā samudgakopādhivirahānmaṇiḥ svasvarūpo bhāti tathaiva svasvarūpāvabodhādvimalamapyaśuddhābhimataśarīropādhikṣayādviśuddhameva saṃvedanaṃ bhāsate  / nanu maṇiryathā samudgakopādhervimukto 'pi punaranyatamopādhiparigrahātsamalaḥ saṃpadyate tathaiva tanūpādhermuktamapi saṃvedanaṃ maṇivadupādhyantaraṃ cedgṛhṇāti tarhi punarapi sopādhitvādaśuddhameveti pariharati upādhyantaraśūnyamapi iti  / na dṛṣṭāntadārṣṭāntikayoḥ sarvathā sāmyaṃ yataḥ piṇḍapātāttasya mahāprakāśavapuṣaḥ paramādvayarūpasya sarvamidamupādhyabhimataṃ svāṅgakalpaṃ bhāsate 'to vyatiriktasyopādhyantarasyābhāvānna punastadupādhyantareṇa viśiṣyata iti na maṇinopādhigrahaṇasya sāmyam  / ajñānamūlaṃ kila śarīropādhigrahaṇaṃ taccetsvātmajñānakuṭhāreṇa dalitaṃ kathaṅkāraṃ punarupādhisaṃśrayo bhavet  / yaduktam

ajñānenāvṛtaṃ jñānaṃ tena muhyanti jantavaḥ  //
jñānena tu tadajñānaṃ yeṣāṃ nāśitamātmanaḥ  /
teṣāmādityavajjñānaṃ prakāśayati tatparam  //

iti śrīgītāsu  / tasmātsvasvarūpajñānādyoginaḥ svasaṃvedanaṃ sadaiva śuddhameveti  // 88  //

sarvopādhyutpattau yathāvatpariśīlitavyāpāramanaḥsaṃskāraprarūḍhireva nimittaṃ na punarnūtanatvena kimapyāyātītyāvedayati

śāstrādiprāmāṇyādavicalitaśraddhayāpi tanmayatām  /
prāptaḥ sa eva pūrvaṃ svargaṃ narakaṃ manuṣyatvam  // 89  //

āgamaprāmāṇyādgurūpadeśapāramparyakathanādyuktipariśīlanātprāgvāsanāprarūḍhayā śraddhayā vā svātmajñāna iṣṭāpūrte pāśave karmaṇi vā kṛtābhyāsaḥ pramātā tadaiva tatsaṃskārapraroheṇa tanmayatām tattadabhyastavastusvarūpatām prāptaḥ sannuttaratra dehapātādvāsanānuguṇyena svargam niratiśayāṃ prītim narakam avīcyādiduḥkham manuṣyatvam sukhaduḥkhobhayarūpaṃ manuṣyabhāvaṃ prāpnoti na punaranabhyastavāsanasyāpi puruṣasya dehapātādeva yatkiñcidāpatati  / yataḥ sarvaḥ pramātā yenāśayena yadabhyasyati tadaiva sa tadrūpo bhavati kintu maraṇasamaye sphuṭatayā yadabhilaṣitaṃ vastu tatpramāturabhivyaktiṃ yātīti nābhyastavastunaḥ kadācidviparyayaḥ syānnāpyanabhyastavastusvarūpaṃ kiñcidapūrvatvenāpatediti sarvatra pūrvābhyāsa eva kāraṇamiti bhāvaḥ  // 89  //

evaṃ sadā tadbhāvabhāvitatvaṃ svātmavido dehatyāgāvasare pūrṇaprathāheturna punarlokaparidṛśyaḥ puṇyāpuṇyarūpo maraṇāvasaraḥ kaścitsvarganirayādikāraṇaṃ parikalpanīyamityāha

antyaḥ kṣaṇastu tasminpuṇyāṃ pāpāṃ ca vā sthitiṃ puṣyan  /
mūḍhānāṃ sahakārībhāvaṃ gacchati gatau tu na sa hetuḥ  // 90  //

ye 'pi tadātmatvena viduḥ paśupakṣisarīsṛpādayaḥ svagatim  /
te 'pi purātanasaṃbodhasaṃskṛtāstāṃ gatiṃ yānti  // 91  //

evaṃpratipādite jñānini antyaḥ kṣaṇaḥ caramo dehavināśasahabhāvī kālo dhātudoṣavaśena duṣṭavyādhyanubhavādvā samīpasthitaiḥ pramātṛbhiranumitām puṇyāṃ pāpamayīṃ vā sthitiṃ puṣyan sevamānaḥ san mūḍhānām dehātmamānināṃ pramātṝṇāmeva sahakārībhāvam kāraṇatvaṃ gacchati  / gacchatu varāko naitāvatā nirbhagnadehātmamānitve sadā svātmamaheśvaranibhālanacature tasmin yogini saḥ antyaḥ kṣaṇaḥ gatau dehāddehāntaraprāptau hetuḥ kāraṇaṃ bhavet  / kuta etadāgatamiti nidarśayannāha ye 'pi iti  / ye 'pi kenāpyāśayavaśena śāpādinā vā pāpayonayaḥ paśurūpatāmapi prāptāḥ svagatim ātmasthitiṃ maraṇāvasare ātmatvena jānīyuḥ te mūḍhāḥ santo 'pi prāgabhyastasvātmajñānavāsanāprabodhānugṛhītāḥ svātmasthitiṃ ca labhante  / gajendramokṣaṇādau yathā hastinā paśusvabhāvenāpi satā prākpariśīlitaparameśvarabhaktisaṃskāraprabuddhena viṣṇuṃ bhagavantaṃ stutvā samyakkañcukaṃ vihāya svasvarūpamupalabdhaṃ kastatra smaraṇe heturabhūt  / ayaṃ bhāvaḥ śarīrādyutthadhātudoṣavaśātkāṣṭhapāṣāṇaceṣṭo jñānī puṇyaṃ pāpamālabiḍālādikaṃ vā yatkiñcitpralapandehaṃ tyajati naitāvatā svasthaceṣṭatayā yadabhyastaṃ jñānādikaṃ tasya vipralopaḥ syāccharīrādigatā dharmāḥ śarīrādāveva nipatanti na punaḥ sadā bhāvitaṃ vastu sthagayituṃ prabhavantītyāmaraṇakṣaṇaṃ sarvatra prarūḍhireva paramārthaḥ  / yathā gītāsūktam

yaṃ yaṃ vāpi smaranbhāvaṃ tyajatyante kalevaram  /
taṃ tamevaiti kaunteya sadā tadbhāvabhāvitaḥ  //

tathā

teṣāṃ satatayuktānāṃ bhajatāṃ prītipūrvakam  /
dadāmi buddhiyogaṃ taṃ yena māmupayānti te  //

iti bhāvitāntaḥkaraṇataiva paryantagatidānahetuḥ  // 91  //

evaṃ darśitadṛṣṭyā yataḥ sadā tadbhāvabhāvitatvamapahastya nūtanatvena śarīravināśe nāpūrvaṃ kiñcitsamāpatejjñānino yato deha eva vināśī kevalaṃ sa eva vinaśyati na punarvāsanāpraroha iti darśayannāha

svargamayo nirayamayastadayaṃ dehāntarālagaḥ puruṣaḥ  /
tadbhaṅge svaucityāddehāntarayogamabhyeti  // 92  //

evaṃ jñānāvasare svātmā sakṛdasya yādṛgavabhātaḥ  /
tādṛśa eva tadāsau na dehapāte 'nyathā bhavati  // 93  //

tat tasmāccharīraghaṭādiniviṣṭaḥ puruṣaḥ sarvasya kārmamalādhivāsita ātmā svargādyabhiprāyapūrvakṛtakarmaphalavāsanāvāsitāntaḥkaraṇaḥ svargamayaḥ prarūḍhasvargaphalavāsanāviśiṣṭatvātsvargaphalabhokteti yāvat  / evaṃ duṣkṛtapūrvakarmavāsanāprarūḍho narakaphalabhoktā kevalaṃ deha ubhayakarmaphalabhogāyatanam tadbhaṅge svaucityāt iti tasmindehakṣaye svasya ātmano yathāhitavāsanānuguṇyādanyena bhogāyatanena śarīrāntareṇa samanantaraṃ saṃbandhamupayātyuttarakālaṃ yena viśiṣṭakarmavāsanādattaphalabhogabhāgī bhavati  / tathaiva jñānāvasare upadeśyasya gurūpadiṣṭasvātmaprakāśanakāle svātmā caitanyam sakṛt ekavāraṃ yādṛk yādṛśaḥ avabhātaḥ upadeśakramānusāreṇa paripūrṇasvātantryalakṣaṇāṃ mitāṃ vā parāmarśadaśāṃ gataḥ tadṛśa eva sadā asau yenaiva svarūpeṇa jñāninā svātmā sarvakālaṃ parāmṛṣṭaḥ tādrūpyeṇa vāsanāprarohāttasya prathate na punaḥ dehapāte prakāśito 'pi svātmā jñāninaḥ anyathā samācchāditaḥ bhavati  / na hi bhātamabhātaṃ syādaparathā na kaścitkiñcidabhyasediti sarvavyavahāravipralopo bhavet

dharmeṇa gamanamūrdhvaṃ gamanamadhastādbhavatyadharmeṇa  /
jñānena cāpavargo viparyayādiṣyate bandhaḥ  //

ityādi sarvaṃ ca truṭyettasmānmaraṇakāle śarīraṃ yathāstu tathāstu kevalaṃ vāsanāprarohaḥ svātmagataḥ sarvasya bandhe mokṣe ca heturiti  // 93  //

yadi punardhātuvaiṣamyāccharīre maraṇavyathopalabdhiḥ syānnaitāvatābhyāsaprarohe kācitkṣatirityāveditāmeva sthitimupalabdhuṃ parighaṭayate

karaṇagaṇasaṃpramoṣaḥ smṛtināśaḥ śvāsakalilatā cchedaḥ  /
marmasu rujāviśeṣāḥ śarīrasaṃskārajo bhogaḥ  // 94  //

sa kathaṃ vigrahayoge sati na bhavettena mohayoge 'pi  /
maraṇāvasare jñānī na cyavate svātmaparamārthāt  // 95  //

karaṇagaṇasya bāhyāntargatasya trayodaśātmakasya samyak pramoṣaḥ svarūpavipralopo yathā cakṣurādīnīndriyāṇi rūpādiviṣayālocanāyāṃ na pragalbhante vāgādikarmendriyāṇyapyevaṃ vacanādānādau na pravartante nāpi buddhiryathārthamarthamadhyavasyati manaso 'navasthitirahaṅkāro 'pi madhye madhye saṃskāratayāste  / tathā smṛtināśaḥ anubhūtaviṣayasya saṃpramoṣo bandhubhirarthyamāno 'pi mumūrṣuḥ puro 'vasthitaṃ vastu śataśo 'nubhūtamapi na pratyabhijānātyata eva sadā tadbhāvabhāvitatvaṃ vinā brahmavidyādikathanamantakāle dānamanyatkiñcidvā tasyāmavasthāyāṃ nabhaścitramiva na citte prarohati kintu taditikartavyatāmātraṃ kāryamiti niyogaḥ  / tathā śvāsaḥ kaṇṭhyo vāyustasya kalilatā kaṇṭhadeśe gadgadikā hikkā vā  / anyacca marmasu chedaḥ asthisaṃdhiṣu troṭaḥ  / tathā rujāviśeṣāḥ jvarātīsāraprabhṛtaya iti  / evaṃ yaḥ śarīrasya bhūtakañcukasya vātapittaśleṣmadhātuvaiṣamyāt śarīrasaṃskārajo bhogaḥ dehajo duḥkhānubhavaḥ sa katham kena prakāreṇa vigrahayoge sati jñānino 'pi na bhavet syādeva  / tena hetunā jñānī sadādhaspadīkṛtadehādyabhimānaḥ samāviṣṭasvātmamaheśvarabhāvaśca maraṇakṣaṇajanitaśārīrājñānasaṃbandhe 'pi svātmaparamārthāt prarūḍhacaitanyapratyavamarśasatattvāt na cyavate nānyathābhāvaṃ yāti  / yato 'sau jñānī nyakkṛtadehasaṃbandho na tajjena bhogena tanmayīkartuṃ pāryate kevalaṃ lokavacceccharīrapātasamanantaraṃ kṣaṇaṃ nopalabhata ityetāvatā tasya svasthahṛdayasya svasaṃkalpitābhiprāyeṇa svasthaceṣṭatayābhyastabhagavadbhakterna kiñcidapūrvaṃ samāpatati tasmājjñānī svātmaprathāsamanantarameva mukto na śarīrasaṃskāro 'sya bandhadāyīti śataśaḥ prākpratipāditam  / yastu sadā dehātmamānī puṇyapāpamayaḥ sa kathaṃ dehasaṃskārodbhūtasukhaduḥkhādibhogajanitaṃ tanmayatvaṃ nāyāti  / yaduktam

yadā sattve pravṛddhe tu pralayaṃ yāti dehabhṛt  /
tadottamavidāṃ lokānamalānpratipadyate  //

ityādi  / sattvādayo guṇāḥ prakṛtidharmāstanmayasyaiva niyantraṇāṃ vidadhate yena punastato viviktatayā pariśīlitā na taṃ pratyete kecaneti jñānino 'nya eva panthāḥ  / ye punaradṛṣṭagurucaraṇāḥ paśupramātāraste svagatāndharmānanyatrāpādayanti  / yathā yadyayaṃ jñānī syātkimiti vyādhyādyupahataśarīro bhuṅkte paridadhāti ca yadi vā maraṇasamaye jāḍyamāyātaḥ smṛtamanena na kiñcidityevaṃ bahuprakāramavidyopahatatvādvivadamānāśca kena paryanuyujyantām  / yadyayaṃ jñānī syāddehadharmasaṃskārayuktaśca bhavetkimetāvatā ca tasya duṣyet  / jñāninaḥ svātmaprakāśastattadavasthāvicitro 'pi svātmaprakāśa eva na punastasya svātmānubhavitṛtayā vipralopaḥ syādyena jñānaṃ naśyet  / pūrṇaṣāḍguṇyamahimāpi bhagavānvāsudevaḥ kṛṣṇāvatāre vyādhaśarāghātajanitavyatho bhūtaśarīraṃ tyaktavānityevaṃ kṛtvā kiṃ tasya jagatprabhoḥ svasvarūpavipralopo 'bhūdityā kīṭātsadāśivāntasyāpi dehasaṃskāra etādṛśa eva kintvekaḥ svātmapratyavamarśamātrasanāthadeho 'parastu dehādyātmamānitāsatattva itīyānviśeṣaḥ  / tasmāccharīradharmā jñānyajñāninoḥ sadṛśā eva naitāvatā phalasāmyamityetadeva gītāsūktam

sadṛśaṃ ceṣṭate svasyāḥ prakṛterjñānavānapi  /
prakṛtiṃ yānti bhūtāni nigrahaḥ kiṃ kariṣyati  //

iti  // 95  //

idānīmakrameṇa krameṇa ca jñānayogapariśīlane vicitraparaśaktipātameva kāraṇaṃ pratipādayanphalabhedamāha

paramārthamārgamenaṃ jhaṭiti yadā gurumukhātsamabhyeti  /
atitīvraśaktipātāttadaiva nirvighnameva śivaḥ  // 96  //

yasminneva kāle janaḥ paścimajanmā gurumukhāt pravaradaiśikavaktrāt enam śataśaḥ pratipāditam paramārthamārgam pūrṇasvātantryalakṣaṇaṃ svātmasaṃbodhamukhāmnāyarahasyasaraṇiṃ yaḥ kaścit abhyeti samabhiyāti saḥ tadaiva tasminnevāvasare gurūpadeśasamanantaramevānantarāyaṃ kṛtvā śiva eva syāt  / śrīkule yathoktam

helayā krīḍayā vāpi ādarādvātha tattvavit  /
yasya saṃpātayeddṛṣṭiṃ sa muktastatkṣaṇātpriye  //

iti  / nanu kathamevaṃvidhaṃ mukhāmnāyarahasyamevopanayedityāha atitīvraśaktipātāt iti  / atiśayena tīvraḥ karkaśo yo 'sāvanugrahākhyāyāḥ pārameśvaryāḥ śakteḥ pātaḥ paśuhṛtkamalāvataraṇaṃ yena paśurapi gurvāmnāyavedanācchivībhavati jīvanneva mukta iti yāvat  / yathā tāmradravyaṃ siddharasapātātsuvarṇībhavati  / ayamarthaḥ parameśvarānugrahopāya eva svātmajñānalābha iti nātra niyatiśaktisamutthaṃ japadhyānayajñādikamupāyatayā kramate  / anugrahaśaktividdhahṛdayasya tu haṭhādevākramaṃ devatāmukhāmnāyarahasyaṃ hṛdayamāvarjayati yena jhaṭityeva parameśvarībhāvaṃ yātītyaparyanuyojyo vicitraḥ pārameśvaraḥ śaktipāta iti  // 96  //

yasya punarmadhyamandamandatarādibhedena pravṛttaḥ śaktipātastasya gurūpadeśamāmaraṇakṣaṇaṃ yāvadyogakrameṇa vimṛśataḥ piṇḍapātācchivatvaṃ syāditi pratipādayati

sarvottīrṇaṃ rūpaṃ sopānapadakrameṇa saṃśrayataḥ  /
paratattvarūḍhilābhe paryante śivamayībhāvaḥ  // 97  //

evaṃ kila śaktipātamandatvātpūrṇajñānopadeśānāsādanena sarvottīrṇaṃ rūpam sarvatattvaparyantavartisvabhāvam saṃśrayataḥ sākṣātkurvataḥ kathamityāha sopānapadakrameṇa iti  / kandanābhihṛtkaṇṭhalampikābindunādaśaktirūpāṇi sopānāni ūrdhvamākramaṇāya tīrthānyeva teṣāṃ padam āsādanaṃ tatra hānādānarūpaḥ kramaḥ śanaiḥśanaiḥ kande tato nābhau tato hṛdītyevamākramaṇaṃ teneti  / evaṃ yāvatparamārthaprarohopalabdhau piṇḍapātāvasare yoginastasya krameṇa śivatāsvabhāvā sthitirbhavatītyeṣā kramayuktiḥ kathitā  // 97  //

evamapi kramayogamabhyasyato yoginaḥ samāśvastasyāpi satastathārūḍhirna syādabhīṣṭaprāptāvantarāyo jāyate yadi paramanāsāditatattvasya maraṇaṃ syāttadā kiṃ bhavedityāśaṅkāṃ pariharati

tasya tu paramārthamayīṃ dhārāmagatasya madhyaviśrānteḥ  /
tatpadalābhotsukacetaso 'pi maraṇaṃ kadācitsyāt  // 98  //

yogabhraṣṭaḥ śāstre kathito 'sau citrabhogabhuvanapatiḥ  /
viśrāntisthānavaśādbhūtvā janmāntare śivībhavati  // 99  //

evamullaṅghanakrameṇa yogamabhyasyataḥ kenāpyantarāyeṇa madhyaviśrānteḥ kutraciccakrādhāre 'pyanubhavopalabdhestatraiva paritoṣaṃ gatasyāta eva paramārthamayīṃ dhārāmagatasya paratattvarūpāṃ pratijñātāṃ daśāṃ sarvādhvottīrṇāmaprāptavato yadi vā tatpadalābhotsukacetaso 'pi pratijñātaparamārthasattāsādanasābhilāṣasyāpi kadācit madhye vipattiḥ saṃbhāvyate tadaitasyālabdhalābhasyāpi piṇḍapātātkā gatirityāha yogabhraṣṭaḥ ityādi  / saḥ yogāt samādherubhayathā bhraṣṭaḥ calitaḥ śāstre āgamagranthe kathitaḥ uktaḥ  / kīdṛgbhavedityāha citra ityādi  / piṇḍapātādeva citrabhogāni nānāścaryastryannapānamālyavastrānulepanagītavādyādipradhānāni yāni bhuvanāni svaviśrāntyanuguṇāni tattveśvarasthānāni teṣu patiḥ īśvaro bhavati maraṇasamanantarameva divyairbhogairyujyata iti yāvat  / tadbhogādhikāraparikṣaye punarapi sa yogabhraṣṭaḥ kathaṃ syādityāha viśrānti ityādi  / viśrāntisthānasya kandādeḥ pradeśasya vaśāt tadabhyāsasaṃskāraprabodhasāmarthyātsaḥ janmāntare dvitīye janmani bhūtvā saṃsāre 'dhikāriśarīraṃ yogābhyāsayogyaṃ prāpya pūrvābhyastaṃ yogaṃ prayāsena svīkṛtya helayā paramārthamayīṃ prāgjanmapratijñātāṃ daśāmadhiruhya piṇḍapātācchiva eva bhavati  // 99  //

athābhyasyato 'pi yogaṃ yogino manaścāñcalyādviśrāntimekadeśe 'pi manāgapyalabhamānasya yogaṃ prati śraddhāvataśca piṇḍapātātkā gatiḥ syādityāha

paramārthamārgamenaṃ hyabhyasyāprāpya yogamapi nāma  /
suralokabhogabhāgī muditamanā modate suciram  // 100  //

viṣayeṣu sārvabhaumaḥ sarvajanaiḥ pūjyate yathā rājā  /
bhuvaneṣu sarvadaivairyogabhraṣṭastathā pūjyaḥ  // 101  //

enam iti śataśaḥ pratipāditaṃ svātmajñānasatattvaṃ panthānam abhyasya śraddhābhaktibhyāṃ sevitvāpi cittadoṣānavasthānena yathāvadyogalakṣaṇāṃ viśrāntiṃ janmamadhye 'pyanadhigataḥ sanmṛtaścettadā sa yogabhraṣṭo jñānayogaviṣayaprarūḍhaśraddhābhaktiprasādasāmarthyena devalokabhogabhāgī sāhlādacittaḥ suciram kālaṃ harṣaṃ prayāti surairapi bhuvaneṣu nijanijasthāneṣu pūjyo bhavati  / ka ivetyāha sārva ityādi  / yathā sārvabhaumo rājā saptadvīpeśvaro rājā cakravartī viṣayeṣu nānāmaṇḍaleṣu sarvajanaiḥ pūjyate samabhyarcyate tathaiva ayaṃ prakṣīṇapuṇyāpuṇyaviṣayaḥ samutpannavairāgyaḥ paścimajanmā vandyo 'smākaṃ yasya svātmani jijñāsārthaṃ prāgjanmanyudyamo 'bhūditi surairapi stūyata iti yāvat  // 100  // 101  //

tasya lokāntarabhogādhikāranivṛtteranantaraṃ kiṃ syādityāha

mahatā kālena punarmānuṣyaṃ prāpya yogamabhyasya  /
prāpnoti divyamamṛtaṃ yasmādāvartate na punaḥ  // 102  //

devalokeṣu yathānirdiṣṭeṣu bhogānbhuktvātidīrgheṇa kālena sa yogabhraṣṭaḥ saṃsāre 'sminmanuṣyabhāvamāgatya yogābhyāsasādhanayogyaṃ śarīramāsādya prāgjanmani manaścāñcalyādyo yogo duṣprāpo 'bhūttameva yogam prāgjātabhaktiśraddhāprarūḍhayogavāsanāsaṃskāraprabodhamanāyāsena prāpya samabhyasya ca dehānte divyamamṛtam paratattvasvarūpamupalabhate parasvarūpatādārḍhyaṃ gacchatīti yāvat  / ata eva tasmātpunarāvartanaṃ tasya na syāditi  / evaṃ mahati kalyāṇe svātmajñānaviṣaye manāgapi pratyavamarśaḥ saṃsārasaraṇāya na bhavati  / yaduktaṃ śrīgītāsu

nehābhikramanāśo 'sti pratyavāyo na vidyate  /
svalpamapyasya dharmasya trāyate mahato bhayāt  //

iti  / tathā

ayatiḥ śraddhayopeto yogāccalitamānasaḥ  /
aprāpya yogasaṃsiddhim  ...  //

ityādipraśnādārabhya

anekajanmasaṃsiddhastato yāti parāṃ gatim  //

ityuttaraparyanto grantho muninā pratipādito 'pi smartavya iti  // 102  //

evamanena jñānayogakrameṇa jantormanāgapi spṛṣṭasya sata iyānvibhūtyatiśayo yaḥ pravaktuṃ na pāryate tasmātsarvātmanā vivekārdrahṛdayairjananamaraṇanivṛttau sāvadhānairbhāvyamiti nirūpayati

tasmātsanmārge 'sminnirato yaḥ kaścideti sa śivatvam  /
iti matvā paramārthe yathātathāpi prayatanīyam  // 103  //

yata evaṃ svātmapratyavamarśābhyāsaḥ pratipāditakramavaśāduttamaphalalābhaḥ tasmāt etasminsuśobhane mārge prakṛṣṭamuktiprāpake pathi yaḥ kaścinnirataḥ ityadhikāriniyamābhāvaḥ pradarśitaḥ  / yaḥ kaścit jano jananamaraṇavyādhyādikleśaśataparipīḍitaḥ nirataḥ vivekabuddhyā niḥśeṣeṇa ratastatraiva śraddadhāno bhūtvā nimagnaḥ saḥ janturacirāllaghunaiva kālena śivatvameti sakalasāṃsārikakleśānavadhūya paraśreyorūpadaśāmekenaiva janmanā prāpnoti  / yathā śivadharmottare śāstre

ihaikabhaviko mokṣa eṣa tāvatparīkṣyatām  /
anekabhavikā muktirbhavatāṃ kena vāryate  //

iti  / iti matvā evaṃ vimṛśya tasmin paramārthe yathātathā yena tenāpi prakāreṇa prayatanīyam prakarṣeṇa samudyamaḥ kāryaḥ  / pradhāne yatnaḥ phalavāniti kṛtvātrārthe manāgapyavalepo na vidheyo yena yogābhyāsena svātmaprarūḍhiścetsamutpannā siddhaṃ naḥ samīhitaṃ na ceddivyalokāntaraprāptiḥ  / tato 'pi pratyāvṛttasya prāksamabhyastayogavāsanāprabodhabalena punarapi yogasaṃbandha iti śreyomārgapariśīlanānna viruddhaṃ kiñcitkartuḥ samāpatatīti paramapuruṣārthasādhanāyāṃ manāgapyavalepo na kārya iti śivam  // 103  //

evaṃ śāstrakāraḥ śeṣabhaṭṭārakoktaṃ paramārthasāropadeśaṃ śivādvayaśāsanakrameṇa yuktyanubhavāgamasanāthaṃ pratipādya svātmanaḥ paritoṣamātrārthitayā svābhidhānapradarśanapūrvakamayamevopadeśaḥ parapuruṣārthasādhanopāya iti nirūpayangranthārthopasaṃhāramāha

idamabhinavaguptoditasaṃkṣepaṃ dhyāyataḥ paraṃ brahma  /
acirādeva śivatvaṃ nijahṛdayāveśamabhyeti  // 104  //

idam prathamānaṃ vitatya pratipāditaṃ yat param prakṛṣṭam brahma bṛṃhakatvātparipūrṇānandamayaṃ svātmasvarūpam dhyāyataḥ anāyāsena svātmani pratyavamṛśato janasya acirāt śīghrameva na tu punarbahūnāṃ janmanāmanta iti  / tadevaṃvidhasya brahmabhūtasya śivatvamabhyeti niḥśreyasaprāptiḥ saṃbhavati  / katham nijahṛdayāveśam kṛtvā nijaṃ hṛdayam parāmarśasthānamāviśya  / kīdṛśaṃ tadbrahma kīrtanīyanāmnā abhinavaguptenoditaḥ prakāśitaḥ saṃkṣepaḥ tātparyaṃ yatra tadevaṃvidham  / atra ca nāmavyājenedamapyuktaṃ syādyathā abhinavaḥ yo 'nyairadṛṣṭaḥ parabrahmarahasyātiśayaḥ guptaśca avacchanna ivābhūtsa evaṃvidhaḥ uditaḥ prakāśitaḥ saṃkṣepaḥ yatra tadevaṃvidhaṃ brahmeti  / evamāvedayatā durlabhatopadeśasya pratipāditā syāt  // 104  //

granthaparimāṇaṃ nirūpayannasminprakaraṇe kartṛtvamāha
āryāśatena tadidaṃ saṃkṣiptaṃ śāstrasāramatigūḍham  /
abhinavaguptena mayā śivacaraṇasmaraṇadīptena  // 105  //

idaṃ śāstrasāram bahūnāṃ granthānāṃ yatprakṛṣṭaṃ satattvaṃ tat mayā saṃkṣiptam granthasahasrairapyupapādayitumaśakyaṃ tadeva laghunā vṛttaśataparimāṇena svīkṛtyoktamityanena pratibhākauśalamuktaṃ bhavet  / kīdṛśeṇa mayā śivacaraṇasmaraṇadīptena iti  / śivasya paraśreyaḥsvabhāvasya svātmasthasya cidānandaikamūrteryāni caraṇāni cidraśmayasteṣām smaraṇam śabdādiviṣayagrahaṇakāle nibhālanaṃ pratikṣaṇaṃ svānubhavāpramoṣastena dīptaḥ parāhantācamatkārabhāsvaro 'ta eva kīrtanīyābhidhānena  / anyathā kathaṃ dehādyātmamānino 'jñātasvātmamaheśvarasatattvasyeyati mahārthopadeśe 'sya kartṛtādhikāritvamupapadyate yato yo yatsvabhāvaḥ sa tatsvabhāvaṃ vivektuṃ pragalbhata ityupadeṣṭuḥ samāviṣṭamaheśvarasvabhāvo 'nena vākyenoktaḥ syāditi śivam  // 105  //

iti śrīmahāmāheśvarācāryābhinavaguptaviracitaḥparamārthasāraḥ  //

śrīmataḥ kṣemarājasya sadgurvāmnāyaśālinaḥ  /
sākṣātkṛtamaheśasya tasyāntevāsinā mayā  // 1  //
śrīvitastāpurīdhāmnā viraktena tapasvinā  /
vivṛtiryoganāmneyaṃ pūrṇādvayamayī kṛtā  // 2  //

saṃpūrṇeyaṃ paramārthasārasaṃgrahavivṛtiḥ kṛtistatrabhavatparamamāheśvaraśrīrājānakayogarājasya  //
